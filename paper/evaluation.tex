\section{Evaluation}
\label{sec:eval}

\Mar{I think the high-level goal should be something along the lines
  of understanding prevalence of long test runs and main sources of
  cost in open-source Java projects.  Intuitively, the problem should
  come before the solution :).  Based on that, we study
  parallelization of testing frameworks.  More precisely, we
  investigage how prevalent test parallelization is, the potential for
  improving execution cost, issues of flakiness that hinders use of
  parallelization, and how to address those issues.}

We evaluated the characteristics of regression tests in open-source
development from a sample set of Java projects from Github. We are
interested in opportunities to improve performance of test execution
with parallelism \Jbc{Should we use the term parallelism indistinctly
to refer to multi-process and multi-threaded execution?}\Mar{I think I
used the term low-level parallelism on intro to make this distiction.
Please confirm if that is correct and coherent with what you want to
say.  If yes, use this term.}, measure the
obtained gain (if any), and discuss the major concerns that affect
performance based on the obtained findings. We pose the followinga
research questions:

\newcommand{\RQONE}{How prevalent is the occurence of time-consuming
regression tests in open-source projects?}
\newcommand{\RQTWO}{What is the distribution of CPU and IO bound
regression tests from the sample set?}
\newcommand{\RQTHREE}{What is the proportion of \Fix{time-consuming?}
test cases per project?}
\newcommand{\RQFOUR}{How often developers use the parallelism features
from build manager systems to improve performance on test execution?}

\newcommand{\rqA}{\textbf{RQ1.} \RQONE}
\newcommand{\rqB}{\textbf{RQ2.} \RQTWO}
\newcommand{\rqC}{\textbf{RQ3.} \RQTHREE}
\newcommand{\rqD}{\textbf{RQ4.} \RQFOUR}

\begin{itemize}
    \item \rqA
    \item \rqB
    \item \rqC
    \item \rqD
\end{itemize}

The first research question addresses the distribution of subjects per
elapsed time to execute regression tests. We are interested on
identifying time-consuming test suites for further investigation. The
second research question addresses addresses \Fix{...}. The third
research question addresses the distribution of \Fix{...}. Finally,
the fourth research question investigates the relevance of supporting
parallel execution on builder tools.

\subsection{Subjects}
\label{sec:subjects}

We used the Github's Search API to fetch the top 1.000 projects in
Java with at least 100 stars. The number of stars indicates the
interest and appreciation from the community to a given project.
\Comment{(https://help.github.com/articles/about-stars/)} Although our
criteria is subjective, it is an approximation for relevant projects
to conduct our study. For each project, we detected the build manager
system based on the files located in the root directory (\eg,
\emph{pom.xml}) from the project in the following precedence: Maven,
Gradle, and Ant. From the 1.000 downloaded projects, we automatically
detected the build system from 806 subjects and selected the \Fix{154}
projects that we were able to compile.

\subsection{Setup and Replication}
\label{sec:setup}

To run our experiments we used a Core i7-4790 (3.60GHz) Intel
processor machine with 8 virtual CPUs (4 cores with 2 native threads
each) and 16GB of memory, running on Ubuntu 14.04 LTS (64-bit
version). \Fix{add Vagrantfile with software settings} All the source
artifacts, including supporting scripts (\eg, the script that test
subjects and generates the raw data), and the full list of projects
are publicly available at \Fix{...}.

\subsection{Answering research question RQ1}
\label{sec:rqone}

%% \begin{itemize}
%%     \item \RQONE
%% \end{itemize}

To evaluate the frequency of time-consuming regression tests, we first
compiled the project and test source files, later, we measured the
elapsed time to run the tests via the build system ignoring
non-related tasks (\eg, \emph{javadoc} generation).  We grouped the
resulting time from test executions into four different groups: tests
that ran in less than one minute (\emph{short execution}), one to five
minutes (\emph{normal execution}), five to ten minutes (\emph{long
execution}), and more than ten minutes to execute (\emph{very long
execution}).  Figure~\ref{fig:timecost-barplot} shows the distribution
of subjects per group. It is important to mention that
Figure~\ref{fig:timecost-barplot} represents a lower bound for elapsed
time: since the subjects were tested in a potentially unstable
revision, some tests may fail, interrupting the execution.

\begin{figure}[t!]
    \centering
    \includegraphics[width=0.5\textwidth]{plots/timecost-barplot/timecost-barplot.pdf}
    \caption{\label{fig:timecost-barplot} Subjects grouped by elapsed
    time on test execution ($t$): short execution ($t < 1m$), normal
    execution ($1m \leq t < 5m$), and long execution ($5m \leq t <
    10m$), very long execution ($10m \leq t$).}
\end{figure}

\Fix{Explain scatter plot}

\begin{figure}[t!]
    \centering
    \includegraphics[width=0.5\textwidth]{plots/teststime-scatter/timetests-scatter.pdf}
    \caption{\label{fig:timetests-scatter} \Fix{...}}
\end{figure}

\subsection{Answering research question RQ2}
\label{sec:rqtwo}

%% \begin{itemize}
%%     \item \RQTWO
%% \end{itemize}

\Fix{to appear...}

\subsection{Answering research question RQ3}
\label{sec:rqthree}

%% \begin{itemize}
%%     \item \RQTHREE
%% \end{itemize}

\Fix{to appear...}

\subsection{Answering research question RQ4}
\label{sec:rqfour}

%% \begin{itemize}
%%     \item \RQFOUR
%% \end{itemize}

\Fix{to appear...}

