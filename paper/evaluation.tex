\section{Objects of Analysis}
\label{sec:subjects}

We used \github{}'s search API~\cite{githubsearch} to identify
projects that satisfy the following criteria: (1) the primary language
is Java\footnote{In case of projects in multiple languages, the
  \github{} API considers the predominant language as the primary
  language.}, (2) the project has at least 100 stars, (3) the latest
update was on or after January 1st, 2016, and (4) the \emph{readme}
file contains the string \emph{mvn}.  We focused on Java for its
popularity.  Although there is no clearcut limit on the number of
\github{} stars~\cite{github-stars} to define relevant projects, we
observed that one hundred stars was enough to eliminate trivial subjects. The third
criteria serves to skip projects without recent activity. The fourth
criteria is an approximation to find Maven projects.\Comment{ The
  rationale is that if the string \emph{mvn} exists in the
  \emph{readme} file, it may represent a Maven call (\eg, to compile
or to test the project).} We focused on Maven for its popularity on
Java projects.  Important to highlight that, as of now, the
\github{}'s search API can only reflect contents from repository
statistics (\eg, number of forks, main programming language); it does
not provide a feature to search for projects containing certain files
(\eg{}, \emph{pom.xml}) in the directory structure.
Figure~\ref{fig:subject-query} illustrates the query to the \github{}
API as an HTTP request.   The result set is sorted
in descending order of stars.


\vspace{1ex}
\begin{figure}[t!]
\centering
\scriptsize
\lstset{
  escapeinside={@}{@},
  numbers=left,xleftmargin=1em,frame=single,framexleftmargin=0.5em,
  basicstyle=\ttfamily\scriptsize, boxpos=c, numberstyle=\tiny,
  deletekeywords={true}
}
\begin{lstlisting}
https://api.github.com/search/repositories?q=language:java
 +stars:>=100+pushed:>=2016+mvn%20in:readme&sort=stars
\end{lstlisting}
  \caption{\label{fig:subject-query} Query to the \github{} API for
  projects that (1) use Java, (2) contains at least 100
  stars, (3) has been updated on January 1st, 2016 (or later), (4) contains
  the string \emph{mvn} in the \emph{readme} file.}
  \vspace{-5ex}  
\end{figure}

We used the following methodology to select projects for
analysis. After obtaining the list of potential projects from GitHub, we filtered those
containing a \pomf{} file in the root directory.\Comment{  A Maven project may
contain several sub-modules with multiple \pomf{} files.}
Then, considering this set of Maven projects, we
executed the tests for \SubjectsReruns{} times to discard those projects with
issues
in the build file and non-deterministic results observed from sequential executions.
As of August 25th 2017, our search criteria returned a total of \SubjectsGithub{}
subjects.
From this set of projects,
\SubjectsGithubNotMaven{} projects were not Maven or did not have a
\pomf{} in the root directory, 
\SubjectsGithubNotTestable{} projects were not considered because of
environment incompatibility
(\eg, missing\Comment{ required web browser or database management system}~DBMS),
\SubjectsGithubFlaky{} projects were discarded because of
``flaky tests''~\cite{luo-etal-fse2014}. A ``flaky'' test is a test that passes or fails under
the same circumstances leading to non-deterministic results.
As some of our experiments consist of running tests on different
threads, we ignored these projects as it would be impractical
to identify whether a test failed due to a race condition or some
other source of flakiness.
From the remaining \SubjectsGithubConsistant{} projects with
deterministic results, we eliminated \SubjectsGithubTooManyFailures{}
projects with \SuiteFailingThreshold{} or more failing tests as to
reduce bias. For the
remaining projects with failing tests, we used the JUnit's
\CodeIn{@Ignore} annotation to ignore failing tests.
Our final set of subjects contains \numSubjs{} projects.
Figure~\ref{fig:subjects} summarizes our sample set.

\begin{figure}[ht]
  \centering
  \includegraphics[width=0.27\textwidth]{results/piechart-subjs.pdf}
  \caption{\label{fig:subjects}We fetched \SubjectsGithub{} popular projects
  hosted on \github{}. From this initial sample, we ignored
  \SubjectsGithubNotMaven{} projects without Maven support,
  \SubjectsGithubNotTestable{} with missing dependencies,
  \SubjectsGithubFlaky{} projects with flaky tests, and
  \SubjectsGithubTooManyFailures{} projects had at least
  \SuiteFailingThreshold{} of failing tests. We considered
  \numSubjs{} projects to conduct our study.}
  \vspace{-1ex}
\end{figure}

%\subsection{Setup and Replication}
\label{sec:setup}

To run our experiments, we used a Core i7-4790 (3.60 GHz) Intel
processor machine with eight virtual CPUs (four cores with two native
threads each) and 16GB of memory, running Ubuntu 14.04 LTS Trusty Tahr
(64-bit version).  We used\Comment{ Git,} Java 8 and Maven
3.3.9 to build projects and run test suites. To
process test results and generate plots
we used Python\Comment{ 3.4}, Bash, R and Ruby\Comment{ 2.3}.  All source artifacts are
publicly available for replication on our website~\cite{ourwebpage}.
This includes supporting scripts\Comment{ (\eg, the scripts to run the tests
and generate raw analysis data)} and the full list of projects.

\section{Evaluation}
\label{sec:eval}

%% We are interested in understanding the prevalence of time-consuming
%% test suites and main sources of execution cost. We want to understand
%% how the execution cost is distributed on test cases within a test
%% suite and how developers approach test execution. Based on that, we
%% study parallelization of testing frameworks and build systems.  More
%% precisely, we investigate how prevalent test parallelization is, the
%% potential for improving execution cost, issues of flakiness that
%% hinders the use of parallelization, and how to address those issues.
%% More specifically, we pose the following research questions:

%% The first research question addresses the prevalence of long-running
%% test suites. We are interested to know if costly test suites are
%% common in open-source projects.  The second research question
%% addresses the relationship of test cases and the overall execution
%% cost: we are interested to investigate how the execution time is
%% distributed among test cases.  In the third research question, we
%% investigate if developers consider low-level parallelism features
%% available out-of-the-box to amortize test execution (see
%% Section~\ref{sec:modes}). In addition, we want to identify what
%% configurations are often used and why they are more popular (if any).
%% The fourth research question addresses the impact of low-level
%% parallelism on test execution from projects in our sample set. We want
%% to identify subjects that already use test parallelization and compare
%% their performance in contrast to sequential execution. In addition, we
%% are interested in evaluating the performance of sequential test suites
%% with different parallelization settings.
%% Finally, the fifth research question discusses the limitations and
%% insights to overcome the pitfalls of parallelization.

%% \Comment{
%%     \Fix{distribution of execution time per test case. For each subject
%%     identified in the first research question, we investigate how
%%     balanced is the cost of the test suite in contrast to the cost of
%%     test cases and if there are subjects where the time cost is mostly
%%     dominated by a small fraction of test cases.} \Fix{The third research
%%     question addresses the distribution of regression tests according
%%     to the use of computational resources.  We are interested in
%%     investigating if regression test suites are CPU intensive and if there
%%     are opportunities to improve performance. The RQ4 addresses}
%%     \Fix{...elaborate...}

%%     The rationale is that if the time cost of a regression test is equally
%%     distributed among test cases, the execution cost could be potentially
%%     improved by running tests in parallel (in contrast to the scenario
%%     where only one test case dominates most of the execution time).
%% }

We pose the following research questions, organized by the dimensions
of analysis we presented in Section~\ref{sec:intro}.

%% Feasibility
\newcommand{\numRQFeasibilityOne}{RQ1}
\newcommand{\RQFeasibilityOne}{How prevalent are time-consuming
  test suites\Comment{ in open-source projects}?}

\newcommand{\numRQFeasibilityTwo}{RQ2}
\newcommand{\RQFeasibilityTwo}{How is time distributed across test cases?}

%% Adoption
\newcommand{\numRQAdoptionOne}{RQ3}
\newcommand{\RQAdoptionOne}{How popular is test suite
  parallelization\Comment{ in open-source projects}?}

\newcommand{\numRQAdoptionTwo}{RQ4}
\newcommand{\RQAdoptionTwo}{What are the main reasons that prevent developers
  from using test suite parallelization?}

%% Speedups
\newcommand{\numRQSpeedupOne}{RQ5}
\newcommand{\RQSpeedupOne}{What are the speedups obtained with parallelization
  (in projects that actually use it)?}

\newcommand{\numRQSpeedupTwo}{RQ6}
\newcommand{\RQSpeedupTwo}{How test execution scales with the number of
  available CPUs?}

%% Tradeoffs
\newcommand{\numRQIssuesOne}{RQ7}
\newcommand{\RQIssuesOne}{How parallel execution configurations affect testing
  costs and flakiness?}

\setlist[itemize]{leftmargin=1em}
\begin{itemize}
\item Feasibility
  \begin{itemize}
  \item \textbf{\numRQFeasibilityOne.} \RQFeasibilityOne
  \item \textbf{\numRQFeasibilityTwo.} \RQFeasibilityTwo    
  \end{itemize}
\item Adoption
  \begin{itemize}
  \item \textbf{\numRQAdoptionOne.} \RQAdoptionOne    
  \item \textbf{\numRQAdoptionTwo.} \RQAdoptionTwo
  \end{itemize}
\item Speedups
  \begin{itemize}
  \item \textbf{\numRQSpeedupOne.} \RQSpeedupOne
  \item \textbf{\numRQSpeedupTwo.} \RQSpeedupTwo
  \end{itemize}      
\item Tradeoffs
  \begin{itemize}
  \item \textbf{\numRQIssuesOne.} \RQIssuesOne    
  \end{itemize}
\end{itemize}

%%\newcommand{\RQFeasibilityTwo}{What is the distribution of CPU and IO bound
%%regression test suites from the sample set?}
%%
%%\newcommand{\RQAdoptionOne}{How uniformly distributed is the execution time
%%across test cases in costly projects?}
%%
%%\newcommand{\RQAdoptionTwo}{How often developers use the parallelism features
%%from build systems to improve runtime performance?}

\subsection{Feasibility}
\label{sec:rqA}
\label{sec:rqB}

\begin{itemize}
  \item \numRQFeasibilityOne{}. \textbf{\RQFeasibilityOne}
\end{itemize}

To evaluate prevalence of projects with time-consuming test suites, we
considered the \numSubjs{} projects, appearing in 
Figure~\ref{fig:subjects}.  Figure~\ref{fig:mvn-execution} illustrates
the script we used to measure time.

We took the following actions to isolate our environment from
measurement noise.
First, we observed that some test tasks called test-unrelated tasks
(\eg, \emph{javadoc} generation and static analyses) that could
interfere in our time measurements.
To address that potential issue, we inspected Maven execution logs
from a sample including a hundred projects prior to running the script
from Figure~\ref{fig:mvn-execution}.
The tasks we found were ignored from execution (lines 1-4).
Furthermore, we configured
our workstation to only run essential services as to avoid noise from unrelated OS events.
The machine was dedicated to our experiments and we
accessed it via SSH. In addition, we configured the \CodeIn{isolcpus}
option from the Linux Kernel \cite{linux-kernel} to isolate six
virtual CPUs to run our experiments, leaving the remaining CPUs to run
OS processes~\cite{isolcpus-use}.  The rationale for this decision is
to prevent context-switching between user processes (running the
experiment) and OS-related processes.  Finally, to make sure our
measurements were fair, we compared timings corresponding to the
sequential execution of tests using Maven with that obtained with
JUnit's default \CodeIn{JUnitCore} runner, invoked from the command
line.  Results were very close.
The main loop (lines 6-15) of the script in
Figure~\ref{fig:mvn-execution} iterates over the list of subjects and
invokes Maven multiple times\Comment{ to isolate cost of running
tests} (lines 8-11).  It first makes all dependencies available locally
(line 8), compiles the source and test files (line 9), and then runs
the tests in offline mode as to skip the package update task, enabled
by default (line 11). After execution, we used a regular expression on
the output log to extract the elapsed time (line 12-14).

\section{Evaluation}
\label{sec:eval}

We are interested in understanding the prevalence of time-consuming
test suites and main sources of cost. We want to understand how the
execution cost is distributed on test cases within a test suite and
how developers approach test execution. Based on that, we study
parallelization of testing frameworks.  More precisely, we investigate
how prevalent test parallelization is, the potential for improving
execution cost, issues of flakiness that hinders the use of
parallelization, and how to address those issues.  More specifically,
we pose the following research questions:

\newcommand{\numRQA}{RQ1}
\newcommand{\numRQB}{RQ2}
\newcommand{\numRQC}{RQ3}
\newcommand{\numRQD}{RQ4}
%%\newcommand{\numRQE}{RQ5}
\newcommand{\numRQF}{RQ5}

\newcommand{\RQA}{How prevalent is the occurence of time-consuming
  regression test suites in open-source projects?}
\newcommand{\RQB}{How the time cost is distributed across test cases?}
\newcommand{\RQC}{How often parallelization settings appear in build
  files?}
\newcommand{\RQD}{What is the impact of parallelization?}
%%\newcommand{\RQE}{What factors contribute to improve performance
%%  through parallelization?\Jbc{Previously as "When does
%%    parallelization works and when it does not so great?"}}
\newcommand{\RQF}{What are the limitations of low-level parallelism?}

\begin{itemize}
    \item \textbf{\numRQA.} \RQA
    \item \textbf{\numRQB.} \RQB
    \item \textbf{\numRQC.} \RQC
    \item \textbf{\numRQD.} \RQD
%%    \item \textbf{\numRQE.} \RQE
    \item \textbf{\numRQF.} \RQF
\end{itemize}

%%\newcommand{\RQB}{What is the distribution of CPU and IO bound
%%regression test suites from the sample set?}
%%
%%\newcommand{\RQC}{How uniformly distributed is the execution time
%%across test cases in costly projects?}
%%
%%\newcommand{\RQD}{How often developers use the parallelism features
%%from build systems to improve runtime performance?}

The first research question addresses the prevalence of long-running
test suites. We are interested to know if costly test suites are
common in open-source projects.  The second research question
addresses the relationship of test cases and the overall execution
cost: we are interested to investigate how the time cost is
distributed among test cases.  In the third research question, we are
in understanding if developers are aware of low-level parallelism
features available out-of-the-box through build systems (\eg,
configuring Maven to use JUnit with multiple threads to speedup test
execution). In addition, we want to identify which configurations are
often used and why they more popular (if any).  The fourth research
question addresses the impact of low-level parallelism on the
regression tests from our sample set. For subjects that use
parallelization settings, we are interested to compare the execution
performance when these settings are activated and deactivated (\ie,
tests run sequentially).
%%\Jbc{This RQ has to be reworked:} \Fix{The fifth research question
%%addresses the characteristics of regression tests from the previous
%%experiment.  More specifically, we want to investigate if there is a
%%relation between the balance of test execution (\ie, how uniformly
%%distributed is the execution time across tests cases) and the usage of
%%computational resources (\ie, if tests are mostly CPU or IO intense)
%%that impacts the effectiveness of parallelization}.
Finally, the fifth research question discusses the limitations and
insights to overcome the pitfalls of parallelization.

\Comment{
    \Fix{distribution of execution time per test case. For each subject
    identified in the first research question, we investigate how
    balanced is the cost of the test suite in contrast to the cost of
    test cases and if there are subjects where the time cost is mostly
    dominated by a small fraction of test cases.} \Fix{The third research
    question addresses the distribution of regression tests according
    to the use of computational resources.  We are interested in
    investigating if regression test suites are CPU intensive and if there
    are opportunities to improve performance. The RQ4 addresses}
    \Fix{...elaborate...}

    The rationale is that if the time cost of a regression test is equally
    distributed among test cases, the execution cost could be potentially
    improved by running tests in parallel (in contrast to the scenario
    where only one test case dominates most of the execution time).
}

\subsection{Subjects}
\label{sec:subjects}

We evaluated the characteristics of regression tests in open-source
development from a sample set of Java projects from \github{}.  We are
interested to evaluate non-trivial test suites on popular projects
that are in activity. We used the \github{}'s search
API~\cite{githubsearch} to fetch Java projects according to the
following criteria: (1) the primary language must be Java\footnote{In
case of projects in multiple languages, the \github{} API considers
the predominant language as the primary language.}; (2) the project
has at least 100 stars; (3) latest update on (or after) January 1st,
2016; (4) the \emph{readme} file contains the string \emph{mvn}.

On \github{}, when a user adds a star to a project, s/he demonstrated
appreciation and bookmarked it for later
reference~\cite{github-stars}.  Although there is not a specific range
for the number of stars, our criteria is an estimation to avoid
trivial subjects: we assumed that \github{} users are likely to
demonstrate interest on well-tested projects. The third criteria is a
constraint to skip projects without recent activity. The fourth
criteria is an approximation to find projects with Maven support. The
rationale is that if the string \emph{mvn} exists in the \emph{readme}
file, it may represent a Maven call (\eg, to compile or test the
project). We used Maven as a reference due to its popularity on Java
projects and to automate our evaluation scripts.
Figure~\ref{fig:subject-query} illustrates the query as an HTTP
request.

\begin{figure}[h!]
\centering
\scriptsize
\lstset{
    escapeinside={@}{@},
    numbers=left,xleftmargin=1em,frame=single,framexleftmargin=0.5em,
    basicstyle=\ttfamily\scriptsize, boxpos=c, numberstyle=\tiny,
    deletekeywords={true}
}
\begin{lstlisting}
https://api.github.com/search/repositories?q=language:java
        +stars:>=100+pushed:>=2016-01-01
        +mvn%20in:readme+sort:stars
\end{lstlisting}
    \caption{\label{fig:subject-query} Query to the \github{} API for
    projects with the following criteria: (1) Java, (2) at least 100
    stars, (3) updated on January 1st, 2016 (or later), (4) contains
    the string \emph{mvn} in the \emph{readme} file. Output is
    paginated in descending order of stars.}
\end{figure}

A Maven project may contain several sub-modules with multiple \pomf{}
files. We considered only projects with a \pomf{} file located in the
root directory.  As of March 25th 2017, our search criteria returned
\SubjectsGithub{} subjects.  From the \SubjectsGithub{} downloaded
projects, \SubjectsGithubNotMaven{} were not Maven projects (or did
not have a \pomf{} in the root directory) and
\SubjectsGithubNotTestable{} were in an untestable revision (\eg,
missing dependencies or incompatible testing environment). To ensure
that our sample set is stable, we retested the remaining subjects to
eliminate flaky tests (total of \SubjectsGithubFlaky{} subjects). Our
final set consists in \numSubjs{} testable subjects.

\subsection{Setup and Replication}
\label{sec:setup}

To run our experiments, we used a Core i7-4790 (3.60 GHz) Intel
processor machine with eight virtual CPUs (four cores with two native
threads each) and 16GB of memory, running Ubuntu 14.04 LTS Trusty Tahr
(64-bit version). Software settings include \Comment{the Linux
\emph{sysstat} package to measure performance, }git to fetch subjects,
Java 8 and Maven 3.3.9 to build and test subjects. Our evaluation
scripts depends on Python 3.4 and Bash to execute and R to analyze the
data. For replication, all source artifacts are publicly available at
\Fix{create gh-pages}, including supporting scripts (\eg, the script
that test subjects and generates the raw data), and the full list of
projects. \Comment{, and a \emph{Vagrantfile} to emulate our hardware
and all software dependencies.}

\subsection{Answering research question \numRQA{}}
\label{sec:rqA}

\begin{itemize}
    \item \emph{\RQA}
\end{itemize}

To evaluate the frequency of time-consuming regression tests, we
considered the \numSubjs{} subjects from Section~\ref{sec:subjects}
and compared the elapsed time to run their tests.
Figure~\ref{fig:mvn-execution} shows the commands used in our script
to test each subject. The main loop (lines 5-11) iterates over the
list of subjects and invokes Maven in separated steps (lines 7-9). To
avoid inflating the measured time we executed Maven in different
steps: we first compiled the source and test files (line 7), made all
dependencies available locally (line 8) and later, we ran the tests in
offline mode (line 9) to bypass package updates. After the execution,
we used a regular expression on the output log to extract the elapsed
time (line 10). Before collecting the time cost, we executed all
subjects and randomly selected 100 logs to inspect and identify
non-related tasks (\eg, \emph{javadoc} generation and static analysis)
to ignore during the experiment (lines 1-3).  Also, measured the
approximated overhead from the build system after skipping non-related
tasks. We identified the generated test reports and compared the
difference between the elapsed time reported from Maven and the sum of
all test cases executed.
To avoid noise from operating system events, we used a dedicated
server remotely via SSH with the operating system running only
essential services (\eg, the SSH server). In addition, we configured
the \CodeIn{isolcpus} option from the Linux Kernel \cite{linux-kernel}
to isolate six virtual CPUs to execute our experiment and the
remaining CPUs to run the operating system. By isolating a set of
virtual CPUs, we prevent context-switching from experiment processes
and OS-related processes.

\section{Evaluation}
\label{sec:eval}

We are interested in understanding the prevalence of time-consuming
test suites and main sources of cost. We want to understand how the
execution cost is distributed on test cases within a test suite and
how developers approach test execution. Based on that, we study
parallelization of testing frameworks.  More precisely, we investigate
how prevalent test parallelization is, the potential for improving
execution cost, issues of flakiness that hinders the use of
parallelization, and how to address those issues.  More specifically,
we pose the following research questions:

\newcommand{\numRQA}{RQ1}
\newcommand{\numRQB}{RQ2}
\newcommand{\numRQC}{RQ3}
\newcommand{\numRQD}{RQ4}
%%\newcommand{\numRQE}{RQ5}
\newcommand{\numRQF}{RQ5}

\newcommand{\RQA}{How prevalent is the occurence of time-consuming
  regression test suites in open-source projects?}
\newcommand{\RQB}{How the time cost is distributed across test cases?}
\newcommand{\RQC}{How often parallelization settings appear in build
  files?}
\newcommand{\RQD}{What is the impact of parallelization?}
%%\newcommand{\RQE}{What factors contribute to improve performance
%%  through parallelization?\Jbc{Previously as "When does
%%    parallelization works and when it does not so great?"}}
\newcommand{\RQF}{What are the limitations of low-level parallelism?}

\begin{itemize}
    \item \textbf{\numRQA.} \RQA
    \item \textbf{\numRQB.} \RQB
    \item \textbf{\numRQC.} \RQC
    \item \textbf{\numRQD.} \RQD
%%    \item \textbf{\numRQE.} \RQE
    \item \textbf{\numRQF.} \RQF
\end{itemize}

%%\newcommand{\RQB}{What is the distribution of CPU and IO bound
%%regression test suites from the sample set?}
%%
%%\newcommand{\RQC}{How uniformly distributed is the execution time
%%across test cases in costly projects?}
%%
%%\newcommand{\RQD}{How often developers use the parallelism features
%%from build systems to improve runtime performance?}

The first research question addresses the prevalence of long-running
test suites. We are interested to know if costly test suites are
common in open-source projects.  The second research question
addresses the relationship of test cases and the overall execution
cost: we are interested to investigate how the time cost is
distributed among test cases.  In the third research question, we are
in understanding if developers are aware of low-level parallelism
features available out-of-the-box through build systems (\eg,
configuring Maven to use JUnit with multiple threads to speedup test
execution). In addition, we want to identify which configurations are
often used and why they more popular (if any).  The fourth research
question addresses the impact of low-level parallelism on the
regression tests from our sample set. For subjects that use
parallelization settings, we are interested to compare the execution
performance when these settings are activated and deactivated (\ie,
tests run sequentially).
%%\Jbc{This RQ has to be reworked:} \Fix{The fifth research question
%%addresses the characteristics of regression tests from the previous
%%experiment.  More specifically, we want to investigate if there is a
%%relation between the balance of test execution (\ie, how uniformly
%%distributed is the execution time across tests cases) and the usage of
%%computational resources (\ie, if tests are mostly CPU or IO intense)
%%that impacts the effectiveness of parallelization}.
Finally, the fifth research question discusses the limitations and
insights to overcome the pitfalls of parallelization.

\Comment{
    \Fix{distribution of execution time per test case. For each subject
    identified in the first research question, we investigate how
    balanced is the cost of the test suite in contrast to the cost of
    test cases and if there are subjects where the time cost is mostly
    dominated by a small fraction of test cases.} \Fix{The third research
    question addresses the distribution of regression tests according
    to the use of computational resources.  We are interested in
    investigating if regression test suites are CPU intensive and if there
    are opportunities to improve performance. The RQ4 addresses}
    \Fix{...elaborate...}

    The rationale is that if the time cost of a regression test is equally
    distributed among test cases, the execution cost could be potentially
    improved by running tests in parallel (in contrast to the scenario
    where only one test case dominates most of the execution time).
}

\subsection{Subjects}
\label{sec:subjects}

We evaluated the characteristics of regression tests in open-source
development from a sample set of Java projects from \github{}.  We are
interested to evaluate non-trivial test suites on popular projects
that are in activity. We used the \github{}'s search
API~\cite{githubsearch} to fetch Java projects according to the
following criteria: (1) the primary language must be Java\footnote{In
case of projects in multiple languages, the \github{} API considers
the predominant language as the primary language.}; (2) the project
has at least 100 stars; (3) latest update on (or after) January 1st,
2016; (4) the \emph{readme} file contains the string \emph{mvn}.

On \github{}, when a user adds a star to a project, s/he demonstrated
appreciation and bookmarked it for later
reference~\cite{github-stars}.  Although there is not a specific range
for the number of stars, our criteria is an estimation to avoid
trivial subjects: we assumed that \github{} users are likely to
demonstrate interest on well-tested projects. The third criteria is a
constraint to skip projects without recent activity. The fourth
criteria is an approximation to find projects with Maven support. The
rationale is that if the string \emph{mvn} exists in the \emph{readme}
file, it may represent a Maven call (\eg, to compile or test the
project). We used Maven as a reference due to its popularity on Java
projects and to automate our evaluation scripts.
Figure~\ref{fig:subject-query} illustrates the query as an HTTP
request.

\begin{figure}[h!]
\centering
\scriptsize
\lstset{
    escapeinside={@}{@},
    numbers=left,xleftmargin=1em,frame=single,framexleftmargin=0.5em,
    basicstyle=\ttfamily\scriptsize, boxpos=c, numberstyle=\tiny,
    deletekeywords={true}
}
\begin{lstlisting}
https://api.github.com/search/repositories?q=language:java
        +stars:>=100+pushed:>=2016-01-01
        +mvn%20in:readme+sort:stars
\end{lstlisting}
    \caption{\label{fig:subject-query} Query to the \github{} API for
    projects with the following criteria: (1) Java, (2) at least 100
    stars, (3) updated on January 1st, 2016 (or later), (4) contains
    the string \emph{mvn} in the \emph{readme} file. Output is
    paginated in descending order of stars.}
\end{figure}

A Maven project may contain several sub-modules with multiple \pomf{}
files. We considered only projects with a \pomf{} file located in the
root directory.  As of March 25th 2017, our search criteria returned
\SubjectsGithub{} subjects.  From the \SubjectsGithub{} downloaded
projects, \SubjectsGithubNotMaven{} were not Maven projects (or did
not have a \pomf{} in the root directory) and
\SubjectsGithubNotTestable{} were in an untestable revision (\eg,
missing dependencies or incompatible testing environment). To ensure
that our sample set is stable, we retested the remaining subjects to
eliminate flaky tests (total of \SubjectsGithubFlaky{} subjects). Our
final set consists in \numSubjs{} testable subjects.

\subsection{Setup and Replication}
\label{sec:setup}

To run our experiments, we used a Core i7-4790 (3.60 GHz) Intel
processor machine with eight virtual CPUs (four cores with two native
threads each) and 16GB of memory, running Ubuntu 14.04 LTS Trusty Tahr
(64-bit version). Software settings include \Comment{the Linux
\emph{sysstat} package to measure performance, }git to fetch subjects,
Java 8 and Maven 3.3.9 to build and test subjects. Our evaluation
scripts depends on Python 3.4 and Bash to execute and R to analyze the
data. For replication, all source artifacts are publicly available at
\Fix{create gh-pages}, including supporting scripts (\eg, the script
that test subjects and generates the raw data), and the full list of
projects. \Comment{, and a \emph{Vagrantfile} to emulate our hardware
and all software dependencies.}

\subsection{Answering research question \numRQA{}}
\label{sec:rqA}

\begin{itemize}
    \item \emph{\RQA}
\end{itemize}

To evaluate the frequency of time-consuming regression tests, we
considered the \numSubjs{} subjects from Section~\ref{sec:subjects}
and compared the elapsed time to run their tests.
Figure~\ref{fig:mvn-execution} shows the commands used in our script
to test each subject. The main loop (lines 5-11) iterates over the
list of subjects and invokes Maven in separated steps (lines 7-9). To
avoid inflating the measured time we executed Maven in different
steps: we first compiled the source and test files (line 7), made all
dependencies available locally (line 8) and later, we ran the tests in
offline mode (line 9) to bypass package updates. After the execution,
we used a regular expression on the output log to extract the elapsed
time (line 10). Before collecting the time cost, we executed all
subjects and randomly selected 100 logs to inspect and identify
non-related tasks (\eg, \emph{javadoc} generation and static analysis)
to ignore during the experiment (lines 1-3).  Also, measured the
approximated overhead from the build system after skipping non-related
tasks. We identified the generated test reports and compared the
difference between the elapsed time reported from Maven and the sum of
all test cases executed.
To avoid noise from operating system events, we used a dedicated
server remotely via SSH with the operating system running only
essential services (\eg, the SSH server). In addition, we configured
the \CodeIn{isolcpus} option from the Linux Kernel \cite{linux-kernel}
to isolate six virtual CPUs to execute our experiment and the
remaining CPUs to run the operating system. By isolating a set of
virtual CPUs, we prevent context-switching from experiment processes
and OS-related processes.

\section{Evaluation}
\label{sec:eval}

We are interested in understanding the prevalence of time-consuming
test suites and main sources of cost. We want to understand how the
execution cost is distributed on test cases within a test suite and
how developers approach test execution. Based on that, we study
parallelization of testing frameworks.  More precisely, we investigate
how prevalent test parallelization is, the potential for improving
execution cost, issues of flakiness that hinders the use of
parallelization, and how to address those issues.  More specifically,
we pose the following research questions:

\newcommand{\numRQA}{RQ1}
\newcommand{\numRQB}{RQ2}
\newcommand{\numRQC}{RQ3}
\newcommand{\numRQD}{RQ4}
%%\newcommand{\numRQE}{RQ5}
\newcommand{\numRQF}{RQ5}

\newcommand{\RQA}{How prevalent is the occurence of time-consuming
  regression test suites in open-source projects?}
\newcommand{\RQB}{How the time cost is distributed across test cases?}
\newcommand{\RQC}{How often parallelization settings appear in build
  files?}
\newcommand{\RQD}{What is the impact of parallelization?}
%%\newcommand{\RQE}{What factors contribute to improve performance
%%  through parallelization?\Jbc{Previously as "When does
%%    parallelization works and when it does not so great?"}}
\newcommand{\RQF}{What are the limitations of low-level parallelism?}

\begin{itemize}
    \item \textbf{\numRQA.} \RQA
    \item \textbf{\numRQB.} \RQB
    \item \textbf{\numRQC.} \RQC
    \item \textbf{\numRQD.} \RQD
%%    \item \textbf{\numRQE.} \RQE
    \item \textbf{\numRQF.} \RQF
\end{itemize}

%%\newcommand{\RQB}{What is the distribution of CPU and IO bound
%%regression test suites from the sample set?}
%%
%%\newcommand{\RQC}{How uniformly distributed is the execution time
%%across test cases in costly projects?}
%%
%%\newcommand{\RQD}{How often developers use the parallelism features
%%from build systems to improve runtime performance?}

The first research question addresses the prevalence of long-running
test suites. We are interested to know if costly test suites are
common in open-source projects.  The second research question
addresses the relationship of test cases and the overall execution
cost: we are interested to investigate how the time cost is
distributed among test cases.  In the third research question, we are
in understanding if developers are aware of low-level parallelism
features available out-of-the-box through build systems (\eg,
configuring Maven to use JUnit with multiple threads to speedup test
execution). In addition, we want to identify which configurations are
often used and why they more popular (if any).  The fourth research
question addresses the impact of low-level parallelism on the
regression tests from our sample set. For subjects that use
parallelization settings, we are interested to compare the execution
performance when these settings are activated and deactivated (\ie,
tests run sequentially).
%%\Jbc{This RQ has to be reworked:} \Fix{The fifth research question
%%addresses the characteristics of regression tests from the previous
%%experiment.  More specifically, we want to investigate if there is a
%%relation between the balance of test execution (\ie, how uniformly
%%distributed is the execution time across tests cases) and the usage of
%%computational resources (\ie, if tests are mostly CPU or IO intense)
%%that impacts the effectiveness of parallelization}.
Finally, the fifth research question discusses the limitations and
insights to overcome the pitfalls of parallelization.

\Comment{
    \Fix{distribution of execution time per test case. For each subject
    identified in the first research question, we investigate how
    balanced is the cost of the test suite in contrast to the cost of
    test cases and if there are subjects where the time cost is mostly
    dominated by a small fraction of test cases.} \Fix{The third research
    question addresses the distribution of regression tests according
    to the use of computational resources.  We are interested in
    investigating if regression test suites are CPU intensive and if there
    are opportunities to improve performance. The RQ4 addresses}
    \Fix{...elaborate...}

    The rationale is that if the time cost of a regression test is equally
    distributed among test cases, the execution cost could be potentially
    improved by running tests in parallel (in contrast to the scenario
    where only one test case dominates most of the execution time).
}

\subsection{Subjects}
\label{sec:subjects}

We evaluated the characteristics of regression tests in open-source
development from a sample set of Java projects from \github{}.  We are
interested to evaluate non-trivial test suites on popular projects
that are in activity. We used the \github{}'s search
API~\cite{githubsearch} to fetch Java projects according to the
following criteria: (1) the primary language must be Java\footnote{In
case of projects in multiple languages, the \github{} API considers
the predominant language as the primary language.}; (2) the project
has at least 100 stars; (3) latest update on (or after) January 1st,
2016; (4) the \emph{readme} file contains the string \emph{mvn}.

On \github{}, when a user adds a star to a project, s/he demonstrated
appreciation and bookmarked it for later
reference~\cite{github-stars}.  Although there is not a specific range
for the number of stars, our criteria is an estimation to avoid
trivial subjects: we assumed that \github{} users are likely to
demonstrate interest on well-tested projects. The third criteria is a
constraint to skip projects without recent activity. The fourth
criteria is an approximation to find projects with Maven support. The
rationale is that if the string \emph{mvn} exists in the \emph{readme}
file, it may represent a Maven call (\eg, to compile or test the
project). We used Maven as a reference due to its popularity on Java
projects and to automate our evaluation scripts.
Figure~\ref{fig:subject-query} illustrates the query as an HTTP
request.

\begin{figure}[h!]
\centering
\scriptsize
\lstset{
    escapeinside={@}{@},
    numbers=left,xleftmargin=1em,frame=single,framexleftmargin=0.5em,
    basicstyle=\ttfamily\scriptsize, boxpos=c, numberstyle=\tiny,
    deletekeywords={true}
}
\begin{lstlisting}
https://api.github.com/search/repositories?q=language:java
        +stars:>=100+pushed:>=2016-01-01
        +mvn%20in:readme+sort:stars
\end{lstlisting}
    \caption{\label{fig:subject-query} Query to the \github{} API for
    projects with the following criteria: (1) Java, (2) at least 100
    stars, (3) updated on January 1st, 2016 (or later), (4) contains
    the string \emph{mvn} in the \emph{readme} file. Output is
    paginated in descending order of stars.}
\end{figure}

A Maven project may contain several sub-modules with multiple \pomf{}
files. We considered only projects with a \pomf{} file located in the
root directory.  As of March 25th 2017, our search criteria returned
\SubjectsGithub{} subjects.  From the \SubjectsGithub{} downloaded
projects, \SubjectsGithubNotMaven{} were not Maven projects (or did
not have a \pomf{} in the root directory) and
\SubjectsGithubNotTestable{} were in an untestable revision (\eg,
missing dependencies or incompatible testing environment). To ensure
that our sample set is stable, we retested the remaining subjects to
eliminate flaky tests (total of \SubjectsGithubFlaky{} subjects). Our
final set consists in \numSubjs{} testable subjects.

\subsection{Setup and Replication}
\label{sec:setup}

To run our experiments, we used a Core i7-4790 (3.60 GHz) Intel
processor machine with eight virtual CPUs (four cores with two native
threads each) and 16GB of memory, running Ubuntu 14.04 LTS Trusty Tahr
(64-bit version). Software settings include \Comment{the Linux
\emph{sysstat} package to measure performance, }git to fetch subjects,
Java 8 and Maven 3.3.9 to build and test subjects. Our evaluation
scripts depends on Python 3.4 and Bash to execute and R to analyze the
data. For replication, all source artifacts are publicly available at
\Fix{create gh-pages}, including supporting scripts (\eg, the script
that test subjects and generates the raw data), and the full list of
projects. \Comment{, and a \emph{Vagrantfile} to emulate our hardware
and all software dependencies.}

\subsection{Answering research question \numRQA{}}
\label{sec:rqA}

\begin{itemize}
    \item \emph{\RQA}
\end{itemize}

To evaluate the frequency of time-consuming regression tests, we
considered the \numSubjs{} subjects from Section~\ref{sec:subjects}
and compared the elapsed time to run their tests.
Figure~\ref{fig:mvn-execution} shows the commands used in our script
to test each subject. The main loop (lines 5-11) iterates over the
list of subjects and invokes Maven in separated steps (lines 7-9). To
avoid inflating the measured time we executed Maven in different
steps: we first compiled the source and test files (line 7), made all
dependencies available locally (line 8) and later, we ran the tests in
offline mode (line 9) to bypass package updates. After the execution,
we used a regular expression on the output log to extract the elapsed
time (line 10). Before collecting the time cost, we executed all
subjects and randomly selected 100 logs to inspect and identify
non-related tasks (\eg, \emph{javadoc} generation and static analysis)
to ignore during the experiment (lines 1-3).  Also, measured the
approximated overhead from the build system after skipping non-related
tasks. We identified the generated test reports and compared the
difference between the elapsed time reported from Maven and the sum of
all test cases executed.
To avoid noise from operating system events, we used a dedicated
server remotely via SSH with the operating system running only
essential services (\eg, the SSH server). In addition, we configured
the \CodeIn{isolcpus} option from the Linux Kernel \cite{linux-kernel}
to isolate six virtual CPUs to execute our experiment and the
remaining CPUs to run the operating system. By isolating a set of
virtual CPUs, we prevent context-switching from experiment processes
and OS-related processes.

\input{codes/evaluation}

\Comment{\Jbc{Ensure to describe that: subjects that could not run the tests
due to incompatible testing environment were removed (likely to happen
for subjects that rely on database systems and other external
factors), the exit status for all executions are consistent and I
removed flaky projs}}

We ran each subject's tests three times and grouped them according to
the time cost (in average): tests that run in less than a minute
(\shortg{} group), tests than run in one minute and less than five
minutes (\medg{} group), and tests that run in five (or more) minutes
(\longg{} group). We based our classification from a previous work
\cite{gligoric-etal-issta2015} and added the \medg{} group due to the
variability of the time cost from subjects out of the \shortg{} group.
\Jbc{these results will be updated $\rightarrow$}
Figure~\ref{fig:rq1-boxplot} shows the time cost distribution of each
group: the \shortg{} group is the most right skewed distribution with
outliers closed to the \medg{} group and 50\% of the subjects run in
less than 15 seconds; the median from the \medg{} group is nearly two
minutes and most of the subjects run in less than four minutes; most
of the \longg{} group runs in less than 25 minutes but has outliers
that require more than 50 minutes to execute.
Figure~\ref{fig:rq1-barplot} summarizes the proportion of subjects
from each group.

\begin{figure}[ht]
    \centering
    \begin{subfigure}{0.24\textwidth}
        \centering
        \includegraphics[width=\textwidth]{plots/barplot-timecost.pdf}
        \caption{\label{fig:rq1-barplot}}
    \end{subfigure}%
    ~
    \begin{subfigure}{0.24\textwidth}
        \centering
        \includegraphics[width=\textwidth]{plots/boxplot-timecost.pdf}
        \caption{\label{fig:rq1-boxplot}}
    \end{subfigure}%
    \caption{(a) Subjects grouped by time cost ($t$): short run ($t <
    1m$), medium run ($1m \le t < 5m$), and long run ($5m \le t$); (b)
    Distribution of time cost by group.}
\end{figure}

\Jbc{these results will be updated $\rightarrow$}
It is important to mention that Figure~\ref{fig:rq1-barplot}
represents a lower bound estimation for time cost because some
projects omit long-running tests on their default execution. For
instance, it is necessary to invoke Maven with additional parameters
to run integration tests from project \CodeIn{apache.maven-surefire}
(nearly 30 min in our environment).  Also, some tests may finish
earlier than expected due to existing bugs on the ongoing revision.
From the \numSubjs{} compiled projects, 286 successfully executed
all tests and 118 reported test failures.

\begin{center}
\fbox{
\begin{minipage}{8cm}
    \textit{Answering \numRQA{}:}~\emph{Given that
    \percentMedLongRunning{} of the projects are costly (\ie,
    \numMedLong{} projects from \medg{} and \longg{}), we concluded
    that time-consuming regression test suites are relatively frequent
    in open-source projects.}
\end{minipage}
}
\end{center}

\subsection{Answering research question \numRQB{}}
\label{sec:rqB}

\begin{itemize}
    \item \emph{\RQB}
\end{itemize}

\Jbc{pending updates! $\rightarrow$}
Section~\ref{sec:rqA} showed that medium and long-running projects are
not uncommon; they account to nearly \percentMedLongRunning{} of the
\numSubjs{} projects we analyzed.  However, parallelism by itself does
not assure speedups on those projects.  One sensible factor that is
important to the effectiveness of low-level parallelism is the
distribution of test costs in the test suite.  In the limit, if cost
is dominated by a single test, it is unlikely that parallelization
will be beneficial as a test method is the smallest working unit in
test frameworks (see Section~\ref{sec:frameworks}).

\begin{figure}[ht]
    \centering

    \begin{subfigure}{0.5\textwidth}
      \centering
      \includegraphics[width=\textwidth]{R/testcost-med.pdf}
      \caption{\label{fig:medtcost}Medium}
    \end{subfigure}\\
    
    \begin{subfigure}{0.5\textwidth}
      \centering
      \includegraphics[width=\textwidth]{R/testcost-long.pdf}
      \caption{\label{fig:longtcost}Long}
    \end{subfigure}%
    
    \begin{subfigure}{0.5\textwidth}
      \centering
      \begin{tabular}{rrrr}
        \toprule
        & $\sigma\leq1$ & $1<\sigma\leq5$ & $\sigma\ge5$ \\
        \midrule    
        Medium & 17 & 20 & 7 \\
        Long   &  9 & 14 & 5 \\
        \bottomrule
      \end{tabular}
      \caption{\label{fig:sd}...\Fix{Looks like there are missing
        projects...}}    
    \end{subfigure}%

    \caption{\label{fig:time-distributions}Distribution of time cost per project.}    
\end{figure}

\sloppy Figures~\ref{fig:medtcost} and~\ref{fig:longtcost} show the
time distribution of tests per project.  We observed that the average
median value of execution cost for a test was relatively small (dashed
horizontal red lines), namely 0.41s for \medg{} projects and 0.23s for
\longg{} projects.  Note that correlation between cost of test suites
and cost of their individual tests is not to be expected.  The
standard deviation ($\sigma$) associated with each ditribution was
relatively low.  Figure~\ref{fig:sd} shows the number of projects
within specific ranges of $\sigma$ values.  We also observed that
projects with very high standard deviations may or may not be
indicative of monopolization of execution by a small number of tests.
For example, the highest value of $\sigma$ ocurred in project
\CodeIn{wildfly-swarm}, whose test suite was classified as
long-running.  This project contains only 65 tests, 62 of which take
less than 0.5s to run, 1 of which takes nearly 3s to run, and 2 of
which take 40m to run.  This is an example where test suite
parallelization may not be very beneficial as $>$99\% execution cost
is dominated by only 3\% of the tests.  Without these two tests this
project would have been classified as short-running.  A closer
inspection in the data indicates that the project
\CodeIn{wildfly-swarm} was an extreme case; we did not find projects
with similar characteristics of time monopolization.  Project
\CodeIn{jenkinsci.gerrit-trigger-plugin} is also classified as
long-running and has the second highest value of $\sigma$.  For this
project, however, cost is distributed more smoothly across 529 tests,
of which 119 take more than 1s to run.  


\begin{center}
\fbox{
\begin{minipage}{8cm}
    \textit{Answering \numRQB{}:}~\emph{Overall, results indicate that
    projects with a very small number of tests monopolizing end-to-end
    execution time were rare.}
\end{minipage}
}
\end{center}

%% We are interested to know whether
%% most of the execution cost of a subject is dominated by a small subset
%% of test cases or if the cost is nearly equally distributed. 

%% We also evaluated the dispersion of time distributions (one
%% distribution per project) to answer research question \numRQB{}.  To
%% measure dispersion \emph{across} projects we used Relative Standard
%% Deviation (RSD)~\cite{everitt-book-stats-2010}.  Note that, if we were
%% to analyze each project in isolation, the standard deviation of a
%% distribution ($\sigma$) would suffice to quantify how dispersed the
%% (time) distribution is.  However, in our case, we would like to be
%% able to compare and summarize dispersion across projects.  The RSD,
%% which is obtained dividing the standard deviation by the mean ($\mu$)
%% of a distribution, provides such normalization effect.  This metric
%% provides a lower bound (zero) but not an upper bound (somewhere close
%% to 1).  The smaller (larger) the value of RSD the more (less) uniform
%% the distribution is.  Consequently, the lower the value of RSD the
%% more parallelizable a test suite should be.



%% \begin{figure}[h!]
%%   \centering
%%   \includegraphics[width=0.5\textwidth]{R/testcost.pdf}  
%%   \caption{\label{fig:relativesd}Distribution of RSD ($\sigma/\mu$)
%%     across projects.}
%% \end{figure}

%% Figure~\ref{fig:relativesd} shows the distribution of RSD across
%% medium and long-running projects.  Results show that the distribution
%% is skewed to the right indicating that test costs are relatively well
%% distributed in most costly projects we analyzed \Fix{$\leftarrow$
%%   confirm}.

%% analyzed the execution time
%% for the \numMedLong{} projects from the \longg{} and \medg{} groups
%% (see Section~\ref{sec:rqA}).
%% For each subject we calculated the
%% relative standard deviation of the test cases: we collected the
%% elapsed time of each individual test, calculated the standard
%% deviation, and divided by the mean. \Jbc{I need to clarify the
%%   relationship "well/bad-balanced" regression test and relative
%%   standard deviation}

%% Results indicated that \Fix{...elaborate...}. \Jbc{We may identify
%% different groups of subjects}\Fix{TODO: collect data + compute the
%% statistic, create a scatter plot to identify groups of subjects}

%% \Jbc{I shouldn't introduce parallelization arguments here but we have
%% to address this at some point:
%% Regression tests that are well distributed may benefit from
%% parallelism since more tests executes at the same time while the
%% opposite scenario may require a different approach. In the later
%% scenario, executing tests in parallel may have insignificant impact
%% since a small subset of test cases dominates the execution.}

\subsection{Answering research question \numRQC{}}
\label{sec:rqC}

\begin{itemize}
    \item \emph{\RQC}
\end{itemize}

To evaluate the presence of parallelization settings (see Section
\ref{sec:modes}), we considered all projects classified as \medg{} and
\longg{}-running, a total of \numMedLong{} subjects (see
Figure~\ref{fig:rq1-barplot}). To reduce noise we did not consider
projects with \shortg{}-running test suites as we observed that
\percentShortSequential{} of those projects do not use parallelism;
including them would favor sequential execution in our analysis.

To answer \numRQC{} we initially mined build files (\ie, \pomf{}) that
contain parallelization keywords, namely \CodeIn{forkMode},
\CodeIn{forkCount}, and \CodeIn{parallel}.  According to the Maven
surefire documentation\Fix{cite surefire doc}, any configuration to
run tests in parallel must contain one of these keywords. This mining
step, however, is not sound as a matching project may not indicate a
project with a valid configuration for running tests in parallel.
False positive can happen because of comments, for instance.  
To eliminate the cases of false positives and also to categorize true
positive cases, we complemented the initial mining step with a manual
inspection of files.


%% settings); the second step (inspection) consists in a manual
%% inspection to confirm the presence of parallelism settings in the
%% build file and classify them according to the parallelism level.
%% Figure \Fix{removed} describes the discovery step: we list the paths
%% of all build files and filter only the files that contain any of the

Figure~\ref{tab:inspection-table} summarizes our results. The first
column indicates the group of projects according to their time cost.
The second column indicates the number of build files per group.  The
third column is a subset from the previous column and represents the
number of build files found in the mining step (including
false-positives highlighted in parentheses). The last column indicates
the ratio of projects with parallelization settings.

From the \numMedLong{} subjects, we found \pomMedLong{} \pomf{} files
where \numPomMatched{} of those files match one of the keywords
mentioned above.  From these \numPomMatched{} build files, we
eliminated six false-positives: two were names of project modules
(\eg, \emph{camel-cookbook-parallel-processing}), one was part of a
comment, and three were commented configurations. The
\numPomMatchedValid{} build files with valid configurations are
distributed across \numProjectsPar{} projects from our sample.
\emph{From these results we found that $\sim$51\% of medium and
long-running projects do not use parallel features to run test
suites.}\Mar{please make it consistent with research
question}\Mar{explain this is over(under)-estimated...}

\begin{figure}[ht!]
    \centering
    \begin{tabular*}{0.48\textwidth}{@{\extracolsep{\fill}}cccc}
        \toprule
        \multirow{2}{*}{Group} %1st row, 1st cell
            & \multirow{2}{*}{\# \pomf{}}
            & \# \pomf{} matched
            & \# projects matched\\%
        % 2nd row - empty cell
            & % empty cell
            & (false-positives)
            & / total\\%
        \midrule%
        Long   & 1613 & 79 (1) & 19 / \numLong{}\\%
        Medium & 1238 & 30 (5) & 22 / \numMed{}\\%
        \midrule%
        Total % last row, first cell
            & \pomMedLong{}
            & \numPomMatched{} (6)
            & \numProjectsPar{} / \numMedLong{}\\%
        \bottomrule%
    \end{tabular*}
    \caption{Presence of parallelization settings in build files: the
    first column indicates the group of projects according to their
    time cost; the second column indicates the number of build files;
    the third column is the subset of files with parallelization
    keywords; the last column indicates the ratio of projects with
    parallelism support.}
    \label{tab:inspection-table} 
\end{figure}

\subsubsection{Distribution of parallel modes}

From the \numProjectsPar{} projects identified above, we investigated
further the \numPomMatchedValid{} build files with parallel settings.
We analyzed the support and distribution of parallel modes from this
subset of projects. To calculate the distribution of parallel modes,
we considered only the presence of the mode in at least one of the
project settings.  Recall that a build file may contain more than one
parallel setting and a project may contain several sub-modules with
build files.  In case the value of a parallel option is resolved
dynamically (\eg, via command-line argument or system variable) we
compute all modes related to the option. For instance, depending on
the value, the \CodeIn{parallel} option can be \Seq{} (\CodeIn{none}),
\ParClassSeqMeth{} (\CodeIn{classes}), \SeqClassParMeth{},
(\CodeIn{methods}), and \ParClassParMeth{} (\CodeIn{all}).
Figure~\ref{fig:freqmodes} summarizes our findings. \Jbc{Given that we
checked each configuration, we could elaborate how these settings are
used.} \Fix{Missing conclusion: Fork the most used configuration}

\begin{figure}[h!]
    \centering
    \includegraphics[width=0.32\textwidth]{plots/parallel-modes.pdf}
    \caption{\label{fig:freqmodes}Distribution of parallel modes
    identified in a subset of \numProjectsPar{} projects (unused modes
    omitted). A single project may have support to more than one
    parallel mode.}
\end{figure}

\subsection{Answering research question \numRQD{}}
\label{sec:rqD}

\begin{itemize}
    \item \emph{\RQD}
\end{itemize}

To answer \numRQD{}, we considered the \numProjectsPar{} subjects with
parallelization settings identified in \numRQC{} and compared their
time cost with sequential execution. \Jbc{should have a sentence to
motivate this RQ} 
\Jbc{explain methodology}.

\Fix{---------------------}

%%To evaluate the distribution of execution time per project, we sorted
%%the test cases by decreasing order of elapsed time and calculated the
%%number of tests executed in 90\% of the total time. Later, we reported
%%the \Fix{balance} of execution time by dividing the number of tests
%%that represents 90\% of the execution time by the number of tests
%%cases. For instance, a balance of 50\% indicates \Fix{...}.  \Fix{We
%%collected the elapsed time from test cases for each generated report.
%%Maven Surefire generates an XML report with execution information
%%(\eg, number of skipped tests and elapsed time) per test suite
%%\Jbc{Should I use the previous sentence as a footnote or should I
%%delete it?}. We noticed that some test cases reported an elapsed time
%%of zero: since the reported time is in milliseconds, some tests may
%%execute in a shorter time. \Fix{..to be continued...}}. Results
%%indicate that \Fix{...}.
%%
%%\begin{figure}[h!]
%%    \centering
%%    \includegraphics[width=0.4\textwidth]{results/plots/balance.pdf}
%%    \caption{\Fix{balance}}
%%\end{figure}

%% \subsection{Answering research question RQ3}
%% \label{sec:rqThree}
%% 
%% \begin{itemize}
%%     \item \RQB
%% \end{itemize}
%% 
%% To evaluate the distribution of CPU and IO intensive test suites from
%% the sample set, we used the command \emph{sar} to monitor the system
%% activity in background while tests ran. \emph{Sar} is a command that
%% collects and reports statistics (\eg, percentage of IO waiting and
%% usage of CPU in user mode) based on the kernel activity and it is
%% highly configurable to collect detailed information (\eg, usage of a
%% specific processor core or percentage of network interface
%% utilization). We configured \emph{sar} to report \Fix{...explain how
%% we executed and what fields we are interested}. \Fix{explain fields}.
%% Figure \Fix{A} shows the distribution of subjects grouped in intervals
%% of \Fix{W}\% of CPU utilization. Results indicates that \Fix{...}
%% 
%% \begin{figure}[h!]
%%     \centering
%%     \includegraphics[width=0.4\textwidth]{results/plots/cpuness.pdf}
%%     \caption{\Fix{cpu usage}}
%% \end{figure}
%% 
%% \Comment{we proposed the definition of \emph{cpuness}
%% computed as the follow: $((user\_t + system\_t) / elapsed\_t) * 100$,
%% where \emph{user\_t} is the elapsed time of execution in \emph{user
%% mode}, \emph{system\_t} is the elapsed in \emph{kernel mode}, and
%% \emph{elapsed\_t} is the elapsed time to finish the execution. We
%% measured the \emph{cpuness} of each regression test \Fix{...elaborate
%% the meaning of cpuness} \Fix{Describe how I measured user, system and
%% "wall" time}.  \Fix{Explain results}.  \Fix{show plots}}
%% 


\Comment{\Jbc{Ensure to describe that: subjects that could not run the tests
due to incompatible testing environment were removed (likely to happen
for subjects that rely on database systems and other external
factors), the exit status for all executions are consistent and I
removed flaky projs}}

We ran each subject's tests three times and grouped them according to
the time cost (in average): tests that run in less than a minute
(\shortg{} group), tests than run in one minute and less than five
minutes (\medg{} group), and tests that run in five (or more) minutes
(\longg{} group). We based our classification from a previous work
\cite{gligoric-etal-issta2015} and added the \medg{} group due to the
variability of the time cost from subjects out of the \shortg{} group.
\Jbc{these results will be updated $\rightarrow$}
Figure~\ref{fig:rq1-boxplot} shows the time cost distribution of each
group: the \shortg{} group is the most right skewed distribution with
outliers closed to the \medg{} group and 50\% of the subjects run in
less than 15 seconds; the median from the \medg{} group is nearly two
minutes and most of the subjects run in less than four minutes; most
of the \longg{} group runs in less than 25 minutes but has outliers
that require more than 50 minutes to execute.
Figure~\ref{fig:rq1-barplot} summarizes the proportion of subjects
from each group.

\begin{figure}[ht]
    \centering
    \begin{subfigure}{0.24\textwidth}
        \centering
        \includegraphics[width=\textwidth]{plots/barplot-timecost.pdf}
        \caption{\label{fig:rq1-barplot}}
    \end{subfigure}%
    ~
    \begin{subfigure}{0.24\textwidth}
        \centering
        \includegraphics[width=\textwidth]{plots/boxplot-timecost.pdf}
        \caption{\label{fig:rq1-boxplot}}
    \end{subfigure}%
    \caption{(a) Subjects grouped by time cost ($t$): short run ($t <
    1m$), medium run ($1m \le t < 5m$), and long run ($5m \le t$); (b)
    Distribution of time cost by group.}
\end{figure}

\Jbc{these results will be updated $\rightarrow$}
It is important to mention that Figure~\ref{fig:rq1-barplot}
represents a lower bound estimation for time cost because some
projects omit long-running tests on their default execution. For
instance, it is necessary to invoke Maven with additional parameters
to run integration tests from project \CodeIn{apache.maven-surefire}
(nearly 30 min in our environment).  Also, some tests may finish
earlier than expected due to existing bugs on the ongoing revision.
From the \numSubjs{} compiled projects, 286 successfully executed
all tests and 118 reported test failures.

\begin{center}
\fbox{
\begin{minipage}{8cm}
    \textit{Answering \numRQA{}:}~\emph{Given that
    \percentMedLongRunning{} of the projects are costly (\ie,
    \numMedLong{} projects from \medg{} and \longg{}), we concluded
    that time-consuming regression test suites are relatively frequent
    in open-source projects.}
\end{minipage}
}
\end{center}

\subsection{Answering research question \numRQB{}}
\label{sec:rqB}

\begin{itemize}
    \item \emph{\RQB}
\end{itemize}

\Jbc{pending updates! $\rightarrow$}
Section~\ref{sec:rqA} showed that medium and long-running projects are
not uncommon; they account to nearly \percentMedLongRunning{} of the
\numSubjs{} projects we analyzed.  However, parallelism by itself does
not assure speedups on those projects.  One sensible factor that is
important to the effectiveness of low-level parallelism is the
distribution of test costs in the test suite.  In the limit, if cost
is dominated by a single test, it is unlikely that parallelization
will be beneficial as a test method is the smallest working unit in
test frameworks (see Section~\ref{sec:frameworks}).

\begin{figure}[ht]
    \centering

    \begin{subfigure}{0.5\textwidth}
      \centering
      \includegraphics[width=\textwidth]{R/testcost-med.pdf}
      \caption{\label{fig:medtcost}Medium}
    \end{subfigure}\\
    
    \begin{subfigure}{0.5\textwidth}
      \centering
      \includegraphics[width=\textwidth]{R/testcost-long.pdf}
      \caption{\label{fig:longtcost}Long}
    \end{subfigure}%
    
    \begin{subfigure}{0.5\textwidth}
      \centering
      \begin{tabular}{rrrr}
        \toprule
        & $\sigma\leq1$ & $1<\sigma\leq5$ & $\sigma\ge5$ \\
        \midrule    
        Medium & 17 & 20 & 7 \\
        Long   &  9 & 14 & 5 \\
        \bottomrule
      \end{tabular}
      \caption{\label{fig:sd}...\Fix{Looks like there are missing
        projects...}}    
    \end{subfigure}%

    \caption{\label{fig:time-distributions}Distribution of time cost per project.}    
\end{figure}

\sloppy Figures~\ref{fig:medtcost} and~\ref{fig:longtcost} show the
time distribution of tests per project.  We observed that the average
median value of execution cost for a test was relatively small (dashed
horizontal red lines), namely 0.41s for \medg{} projects and 0.23s for
\longg{} projects.  Note that correlation between cost of test suites
and cost of their individual tests is not to be expected.  The
standard deviation ($\sigma$) associated with each ditribution was
relatively low.  Figure~\ref{fig:sd} shows the number of projects
within specific ranges of $\sigma$ values.  We also observed that
projects with very high standard deviations may or may not be
indicative of monopolization of execution by a small number of tests.
For example, the highest value of $\sigma$ ocurred in project
\CodeIn{wildfly-swarm}, whose test suite was classified as
long-running.  This project contains only 65 tests, 62 of which take
less than 0.5s to run, 1 of which takes nearly 3s to run, and 2 of
which take 40m to run.  This is an example where test suite
parallelization may not be very beneficial as $>$99\% execution cost
is dominated by only 3\% of the tests.  Without these two tests this
project would have been classified as short-running.  A closer
inspection in the data indicates that the project
\CodeIn{wildfly-swarm} was an extreme case; we did not find projects
with similar characteristics of time monopolization.  Project
\CodeIn{jenkinsci.gerrit-trigger-plugin} is also classified as
long-running and has the second highest value of $\sigma$.  For this
project, however, cost is distributed more smoothly across 529 tests,
of which 119 take more than 1s to run.  


\begin{center}
\fbox{
\begin{minipage}{8cm}
    \textit{Answering \numRQB{}:}~\emph{Overall, results indicate that
    projects with a very small number of tests monopolizing end-to-end
    execution time were rare.}
\end{minipage}
}
\end{center}

%% We are interested to know whether
%% most of the execution cost of a subject is dominated by a small subset
%% of test cases or if the cost is nearly equally distributed. 

%% We also evaluated the dispersion of time distributions (one
%% distribution per project) to answer research question \numRQB{}.  To
%% measure dispersion \emph{across} projects we used Relative Standard
%% Deviation (RSD)~\cite{everitt-book-stats-2010}.  Note that, if we were
%% to analyze each project in isolation, the standard deviation of a
%% distribution ($\sigma$) would suffice to quantify how dispersed the
%% (time) distribution is.  However, in our case, we would like to be
%% able to compare and summarize dispersion across projects.  The RSD,
%% which is obtained dividing the standard deviation by the mean ($\mu$)
%% of a distribution, provides such normalization effect.  This metric
%% provides a lower bound (zero) but not an upper bound (somewhere close
%% to 1).  The smaller (larger) the value of RSD the more (less) uniform
%% the distribution is.  Consequently, the lower the value of RSD the
%% more parallelizable a test suite should be.



%% \begin{figure}[h!]
%%   \centering
%%   \includegraphics[width=0.5\textwidth]{R/testcost.pdf}  
%%   \caption{\label{fig:relativesd}Distribution of RSD ($\sigma/\mu$)
%%     across projects.}
%% \end{figure}

%% Figure~\ref{fig:relativesd} shows the distribution of RSD across
%% medium and long-running projects.  Results show that the distribution
%% is skewed to the right indicating that test costs are relatively well
%% distributed in most costly projects we analyzed \Fix{$\leftarrow$
%%   confirm}.

%% analyzed the execution time
%% for the \numMedLong{} projects from the \longg{} and \medg{} groups
%% (see Section~\ref{sec:rqA}).
%% For each subject we calculated the
%% relative standard deviation of the test cases: we collected the
%% elapsed time of each individual test, calculated the standard
%% deviation, and divided by the mean. \Jbc{I need to clarify the
%%   relationship "well/bad-balanced" regression test and relative
%%   standard deviation}

%% Results indicated that \Fix{...elaborate...}. \Jbc{We may identify
%% different groups of subjects}\Fix{TODO: collect data + compute the
%% statistic, create a scatter plot to identify groups of subjects}

%% \Jbc{I shouldn't introduce parallelization arguments here but we have
%% to address this at some point:
%% Regression tests that are well distributed may benefit from
%% parallelism since more tests executes at the same time while the
%% opposite scenario may require a different approach. In the later
%% scenario, executing tests in parallel may have insignificant impact
%% since a small subset of test cases dominates the execution.}

\subsection{Answering research question \numRQC{}}
\label{sec:rqC}

\begin{itemize}
    \item \emph{\RQC}
\end{itemize}

To evaluate the presence of parallelization settings (see Section
\ref{sec:modes}), we considered all projects classified as \medg{} and
\longg{}-running, a total of \numMedLong{} subjects (see
Figure~\ref{fig:rq1-barplot}). To reduce noise we did not consider
projects with \shortg{}-running test suites as we observed that
\percentShortSequential{} of those projects do not use parallelism;
including them would favor sequential execution in our analysis.

To answer \numRQC{} we initially mined build files (\ie, \pomf{}) that
contain parallelization keywords, namely \CodeIn{forkMode},
\CodeIn{forkCount}, and \CodeIn{parallel}.  According to the Maven
surefire documentation\Fix{cite surefire doc}, any configuration to
run tests in parallel must contain one of these keywords. This mining
step, however, is not sound as a matching project may not indicate a
project with a valid configuration for running tests in parallel.
False positive can happen because of comments, for instance.  
To eliminate the cases of false positives and also to categorize true
positive cases, we complemented the initial mining step with a manual
inspection of files.


%% settings); the second step (inspection) consists in a manual
%% inspection to confirm the presence of parallelism settings in the
%% build file and classify them according to the parallelism level.
%% Figure \Fix{removed} describes the discovery step: we list the paths
%% of all build files and filter only the files that contain any of the

Figure~\ref{tab:inspection-table} summarizes our results. The first
column indicates the group of projects according to their time cost.
The second column indicates the number of build files per group.  The
third column is a subset from the previous column and represents the
number of build files found in the mining step (including
false-positives highlighted in parentheses). The last column indicates
the ratio of projects with parallelization settings.

From the \numMedLong{} subjects, we found \pomMedLong{} \pomf{} files
where \numPomMatched{} of those files match one of the keywords
mentioned above.  From these \numPomMatched{} build files, we
eliminated six false-positives: two were names of project modules
(\eg, \emph{camel-cookbook-parallel-processing}), one was part of a
comment, and three were commented configurations. The
\numPomMatchedValid{} build files with valid configurations are
distributed across \numProjectsPar{} projects from our sample.
\emph{From these results we found that $\sim$51\% of medium and
long-running projects do not use parallel features to run test
suites.}\Mar{please make it consistent with research
question}\Mar{explain this is over(under)-estimated...}

\begin{figure}[ht!]
    \centering
    \begin{tabular*}{0.48\textwidth}{@{\extracolsep{\fill}}cccc}
        \toprule
        \multirow{2}{*}{Group} %1st row, 1st cell
            & \multirow{2}{*}{\# \pomf{}}
            & \# \pomf{} matched
            & \# projects matched\\%
        % 2nd row - empty cell
            & % empty cell
            & (false-positives)
            & / total\\%
        \midrule%
        Long   & 1613 & 79 (1) & 19 / \numLong{}\\%
        Medium & 1238 & 30 (5) & 22 / \numMed{}\\%
        \midrule%
        Total % last row, first cell
            & \pomMedLong{}
            & \numPomMatched{} (6)
            & \numProjectsPar{} / \numMedLong{}\\%
        \bottomrule%
    \end{tabular*}
    \caption{Presence of parallelization settings in build files: the
    first column indicates the group of projects according to their
    time cost; the second column indicates the number of build files;
    the third column is the subset of files with parallelization
    keywords; the last column indicates the ratio of projects with
    parallelism support.}
    \label{tab:inspection-table} 
\end{figure}

\subsubsection{Distribution of parallel modes}

From the \numProjectsPar{} projects identified above, we investigated
further the \numPomMatchedValid{} build files with parallel settings.
We analyzed the support and distribution of parallel modes from this
subset of projects. To calculate the distribution of parallel modes,
we considered only the presence of the mode in at least one of the
project settings.  Recall that a build file may contain more than one
parallel setting and a project may contain several sub-modules with
build files.  In case the value of a parallel option is resolved
dynamically (\eg, via command-line argument or system variable) we
compute all modes related to the option. For instance, depending on
the value, the \CodeIn{parallel} option can be \Seq{} (\CodeIn{none}),
\ParClassSeqMeth{} (\CodeIn{classes}), \SeqClassParMeth{},
(\CodeIn{methods}), and \ParClassParMeth{} (\CodeIn{all}).
Figure~\ref{fig:freqmodes} summarizes our findings. \Jbc{Given that we
checked each configuration, we could elaborate how these settings are
used.} \Fix{Missing conclusion: Fork the most used configuration}

\begin{figure}[h!]
    \centering
    \includegraphics[width=0.32\textwidth]{plots/parallel-modes.pdf}
    \caption{\label{fig:freqmodes}Distribution of parallel modes
    identified in a subset of \numProjectsPar{} projects (unused modes
    omitted). A single project may have support to more than one
    parallel mode.}
\end{figure}

\subsection{Answering research question \numRQD{}}
\label{sec:rqD}

\begin{itemize}
    \item \emph{\RQD}
\end{itemize}

To answer \numRQD{}, we considered the \numProjectsPar{} subjects with
parallelization settings identified in \numRQC{} and compared their
time cost with sequential execution. \Jbc{should have a sentence to
motivate this RQ} 
\Jbc{explain methodology}.

\Fix{---------------------}

%%To evaluate the distribution of execution time per project, we sorted
%%the test cases by decreasing order of elapsed time and calculated the
%%number of tests executed in 90\% of the total time. Later, we reported
%%the \Fix{balance} of execution time by dividing the number of tests
%%that represents 90\% of the execution time by the number of tests
%%cases. For instance, a balance of 50\% indicates \Fix{...}.  \Fix{We
%%collected the elapsed time from test cases for each generated report.
%%Maven Surefire generates an XML report with execution information
%%(\eg, number of skipped tests and elapsed time) per test suite
%%\Jbc{Should I use the previous sentence as a footnote or should I
%%delete it?}. We noticed that some test cases reported an elapsed time
%%of zero: since the reported time is in milliseconds, some tests may
%%execute in a shorter time. \Fix{..to be continued...}}. Results
%%indicate that \Fix{...}.
%%
%%\begin{figure}[h!]
%%    \centering
%%    \includegraphics[width=0.4\textwidth]{results/plots/balance.pdf}
%%    \caption{\Fix{balance}}
%%\end{figure}

%% \subsection{Answering research question RQ3}
%% \label{sec:rqThree}
%% 
%% \begin{itemize}
%%     \item \RQB
%% \end{itemize}
%% 
%% To evaluate the distribution of CPU and IO intensive test suites from
%% the sample set, we used the command \emph{sar} to monitor the system
%% activity in background while tests ran. \emph{Sar} is a command that
%% collects and reports statistics (\eg, percentage of IO waiting and
%% usage of CPU in user mode) based on the kernel activity and it is
%% highly configurable to collect detailed information (\eg, usage of a
%% specific processor core or percentage of network interface
%% utilization). We configured \emph{sar} to report \Fix{...explain how
%% we executed and what fields we are interested}. \Fix{explain fields}.
%% Figure \Fix{A} shows the distribution of subjects grouped in intervals
%% of \Fix{W}\% of CPU utilization. Results indicates that \Fix{...}
%% 
%% \begin{figure}[h!]
%%     \centering
%%     \includegraphics[width=0.4\textwidth]{results/plots/cpuness.pdf}
%%     \caption{\Fix{cpu usage}}
%% \end{figure}
%% 
%% \Comment{we proposed the definition of \emph{cpuness}
%% computed as the follow: $((user\_t + system\_t) / elapsed\_t) * 100$,
%% where \emph{user\_t} is the elapsed time of execution in \emph{user
%% mode}, \emph{system\_t} is the elapsed in \emph{kernel mode}, and
%% \emph{elapsed\_t} is the elapsed time to finish the execution. We
%% measured the \emph{cpuness} of each regression test \Fix{...elaborate
%% the meaning of cpuness} \Fix{Describe how I measured user, system and
%% "wall" time}.  \Fix{Explain results}.  \Fix{show plots}}
%% 


\Comment{\Jbc{Ensure to describe that: subjects that could not run the tests
due to incompatible testing environment were removed (likely to happen
for subjects that rely on database systems and other external
factors), the exit status for all executions are consistent and I
removed flaky projs}}

We ran each subject's tests three times and grouped them according to
the time cost (in average): tests that run in less than a minute
(\shortg{} group), tests than run in one minute and less than five
minutes (\medg{} group), and tests that run in five (or more) minutes
(\longg{} group). We based our classification from a previous work
\cite{gligoric-etal-issta2015} and added the \medg{} group due to the
variability of the time cost from subjects out of the \shortg{} group.
\Jbc{these results will be updated $\rightarrow$}
Figure~\ref{fig:rq1-boxplot} shows the time cost distribution of each
group: the \shortg{} group is the most right skewed distribution with
outliers closed to the \medg{} group and 50\% of the subjects run in
less than 15 seconds; the median from the \medg{} group is nearly two
minutes and most of the subjects run in less than four minutes; most
of the \longg{} group runs in less than 25 minutes but has outliers
that require more than 50 minutes to execute.
Figure~\ref{fig:rq1-barplot} summarizes the proportion of subjects
from each group.

\begin{figure}[ht]
    \centering
    \begin{subfigure}{0.24\textwidth}
        \centering
        \includegraphics[width=\textwidth]{plots/barplot-timecost.pdf}
        \caption{\label{fig:rq1-barplot}}
    \end{subfigure}%
    ~
    \begin{subfigure}{0.24\textwidth}
        \centering
        \includegraphics[width=\textwidth]{plots/boxplot-timecost.pdf}
        \caption{\label{fig:rq1-boxplot}}
    \end{subfigure}%
    \caption{(a) Subjects grouped by time cost ($t$): short run ($t <
    1m$), medium run ($1m \le t < 5m$), and long run ($5m \le t$); (b)
    Distribution of time cost by group.}
\end{figure}

\Jbc{these results will be updated $\rightarrow$}
It is important to mention that Figure~\ref{fig:rq1-barplot}
represents a lower bound estimation for time cost because some
projects omit long-running tests on their default execution. For
instance, it is necessary to invoke Maven with additional parameters
to run integration tests from project \CodeIn{apache.maven-surefire}
(nearly 30 min in our environment).  Also, some tests may finish
earlier than expected due to existing bugs on the ongoing revision.
From the \numSubjs{} compiled projects, 286 successfully executed
all tests and 118 reported test failures.

\begin{center}
\fbox{
\begin{minipage}{8cm}
    \textit{Answering \numRQA{}:}~\emph{Given that
    \percentMedLongRunning{} of the projects are costly (\ie,
    \numMedLong{} projects from \medg{} and \longg{}), we concluded
    that time-consuming regression test suites are relatively frequent
    in open-source projects.}
\end{minipage}
}
\end{center}

\subsection{Answering research question \numRQB{}}
\label{sec:rqB}

\begin{itemize}
    \item \emph{\RQB}
\end{itemize}

\Jbc{pending updates! $\rightarrow$}
Section~\ref{sec:rqA} showed that medium and long-running projects are
not uncommon; they account to nearly \percentMedLongRunning{} of the
\numSubjs{} projects we analyzed.  However, parallelism by itself does
not assure speedups on those projects.  One sensible factor that is
important to the effectiveness of low-level parallelism is the
distribution of test costs in the test suite.  In the limit, if cost
is dominated by a single test, it is unlikely that parallelization
will be beneficial as a test method is the smallest working unit in
test frameworks (see Section~\ref{sec:frameworks}).

\begin{figure}[ht]
    \centering

    \begin{subfigure}{0.5\textwidth}
      \centering
      \includegraphics[width=\textwidth]{R/testcost-med.pdf}
      \caption{\label{fig:medtcost}Medium}
    \end{subfigure}\\
    
    \begin{subfigure}{0.5\textwidth}
      \centering
      \includegraphics[width=\textwidth]{R/testcost-long.pdf}
      \caption{\label{fig:longtcost}Long}
    \end{subfigure}%
    
    \begin{subfigure}{0.5\textwidth}
      \centering
      \begin{tabular}{rrrr}
        \toprule
        & $\sigma\leq1$ & $1<\sigma\leq5$ & $\sigma\ge5$ \\
        \midrule    
        Medium & 17 & 20 & 7 \\
        Long   &  9 & 14 & 5 \\
        \bottomrule
      \end{tabular}
      \caption{\label{fig:sd}...\Fix{Looks like there are missing
        projects...}}    
    \end{subfigure}%

    \caption{\label{fig:time-distributions}Distribution of time cost per project.}    
\end{figure}

\sloppy Figures~\ref{fig:medtcost} and~\ref{fig:longtcost} show the
time distribution of tests per project.  We observed that the average
median value of execution cost for a test was relatively small (dashed
horizontal red lines), namely 0.41s for \medg{} projects and 0.23s for
\longg{} projects.  Note that correlation between cost of test suites
and cost of their individual tests is not to be expected.  The
standard deviation ($\sigma$) associated with each ditribution was
relatively low.  Figure~\ref{fig:sd} shows the number of projects
within specific ranges of $\sigma$ values.  We also observed that
projects with very high standard deviations may or may not be
indicative of monopolization of execution by a small number of tests.
For example, the highest value of $\sigma$ ocurred in project
\CodeIn{wildfly-swarm}, whose test suite was classified as
long-running.  This project contains only 65 tests, 62 of which take
less than 0.5s to run, 1 of which takes nearly 3s to run, and 2 of
which take 40m to run.  This is an example where test suite
parallelization may not be very beneficial as $>$99\% execution cost
is dominated by only 3\% of the tests.  Without these two tests this
project would have been classified as short-running.  A closer
inspection in the data indicates that the project
\CodeIn{wildfly-swarm} was an extreme case; we did not find projects
with similar characteristics of time monopolization.  Project
\CodeIn{jenkinsci.gerrit-trigger-plugin} is also classified as
long-running and has the second highest value of $\sigma$.  For this
project, however, cost is distributed more smoothly across 529 tests,
of which 119 take more than 1s to run.  


\begin{center}
\fbox{
\begin{minipage}{8cm}
    \textit{Answering \numRQB{}:}~\emph{Overall, results indicate that
    projects with a very small number of tests monopolizing end-to-end
    execution time were rare.}
\end{minipage}
}
\end{center}

%% We are interested to know whether
%% most of the execution cost of a subject is dominated by a small subset
%% of test cases or if the cost is nearly equally distributed. 

%% We also evaluated the dispersion of time distributions (one
%% distribution per project) to answer research question \numRQB{}.  To
%% measure dispersion \emph{across} projects we used Relative Standard
%% Deviation (RSD)~\cite{everitt-book-stats-2010}.  Note that, if we were
%% to analyze each project in isolation, the standard deviation of a
%% distribution ($\sigma$) would suffice to quantify how dispersed the
%% (time) distribution is.  However, in our case, we would like to be
%% able to compare and summarize dispersion across projects.  The RSD,
%% which is obtained dividing the standard deviation by the mean ($\mu$)
%% of a distribution, provides such normalization effect.  This metric
%% provides a lower bound (zero) but not an upper bound (somewhere close
%% to 1).  The smaller (larger) the value of RSD the more (less) uniform
%% the distribution is.  Consequently, the lower the value of RSD the
%% more parallelizable a test suite should be.



%% \begin{figure}[h!]
%%   \centering
%%   \includegraphics[width=0.5\textwidth]{R/testcost.pdf}  
%%   \caption{\label{fig:relativesd}Distribution of RSD ($\sigma/\mu$)
%%     across projects.}
%% \end{figure}

%% Figure~\ref{fig:relativesd} shows the distribution of RSD across
%% medium and long-running projects.  Results show that the distribution
%% is skewed to the right indicating that test costs are relatively well
%% distributed in most costly projects we analyzed \Fix{$\leftarrow$
%%   confirm}.

%% analyzed the execution time
%% for the \numMedLong{} projects from the \longg{} and \medg{} groups
%% (see Section~\ref{sec:rqA}).
%% For each subject we calculated the
%% relative standard deviation of the test cases: we collected the
%% elapsed time of each individual test, calculated the standard
%% deviation, and divided by the mean. \Jbc{I need to clarify the
%%   relationship "well/bad-balanced" regression test and relative
%%   standard deviation}

%% Results indicated that \Fix{...elaborate...}. \Jbc{We may identify
%% different groups of subjects}\Fix{TODO: collect data + compute the
%% statistic, create a scatter plot to identify groups of subjects}

%% \Jbc{I shouldn't introduce parallelization arguments here but we have
%% to address this at some point:
%% Regression tests that are well distributed may benefit from
%% parallelism since more tests executes at the same time while the
%% opposite scenario may require a different approach. In the later
%% scenario, executing tests in parallel may have insignificant impact
%% since a small subset of test cases dominates the execution.}

\subsection{Answering research question \numRQC{}}
\label{sec:rqC}

\begin{itemize}
    \item \emph{\RQC}
\end{itemize}

To evaluate the presence of parallelization settings (see Section
\ref{sec:modes}), we considered all projects classified as \medg{} and
\longg{}-running, a total of \numMedLong{} subjects (see
Figure~\ref{fig:rq1-barplot}). To reduce noise we did not consider
projects with \shortg{}-running test suites as we observed that
\percentShortSequential{} of those projects do not use parallelism;
including them would favor sequential execution in our analysis.

To answer \numRQC{} we initially mined build files (\ie, \pomf{}) that
contain parallelization keywords, namely \CodeIn{forkMode},
\CodeIn{forkCount}, and \CodeIn{parallel}.  According to the Maven
surefire documentation\Fix{cite surefire doc}, any configuration to
run tests in parallel must contain one of these keywords. This mining
step, however, is not sound as a matching project may not indicate a
project with a valid configuration for running tests in parallel.
False positive can happen because of comments, for instance.  
To eliminate the cases of false positives and also to categorize true
positive cases, we complemented the initial mining step with a manual
inspection of files.


%% settings); the second step (inspection) consists in a manual
%% inspection to confirm the presence of parallelism settings in the
%% build file and classify them according to the parallelism level.
%% Figure \Fix{removed} describes the discovery step: we list the paths
%% of all build files and filter only the files that contain any of the

Figure~\ref{tab:inspection-table} summarizes our results. The first
column indicates the group of projects according to their time cost.
The second column indicates the number of build files per group.  The
third column is a subset from the previous column and represents the
number of build files found in the mining step (including
false-positives highlighted in parentheses). The last column indicates
the ratio of projects with parallelization settings.

From the \numMedLong{} subjects, we found \pomMedLong{} \pomf{} files
where \numPomMatched{} of those files match one of the keywords
mentioned above.  From these \numPomMatched{} build files, we
eliminated six false-positives: two were names of project modules
(\eg, \emph{camel-cookbook-parallel-processing}), one was part of a
comment, and three were commented configurations. The
\numPomMatchedValid{} build files with valid configurations are
distributed across \numProjectsPar{} projects from our sample.
\emph{From these results we found that $\sim$51\% of medium and
long-running projects do not use parallel features to run test
suites.}\Mar{please make it consistent with research
question}\Mar{explain this is over(under)-estimated...}

\begin{figure}[ht!]
    \centering
    \begin{tabular*}{0.48\textwidth}{@{\extracolsep{\fill}}cccc}
        \toprule
        \multirow{2}{*}{Group} %1st row, 1st cell
            & \multirow{2}{*}{\# \pomf{}}
            & \# \pomf{} matched
            & \# projects matched\\%
        % 2nd row - empty cell
            & % empty cell
            & (false-positives)
            & / total\\%
        \midrule%
        Long   & 1613 & 79 (1) & 19 / \numLong{}\\%
        Medium & 1238 & 30 (5) & 22 / \numMed{}\\%
        \midrule%
        Total % last row, first cell
            & \pomMedLong{}
            & \numPomMatched{} (6)
            & \numProjectsPar{} / \numMedLong{}\\%
        \bottomrule%
    \end{tabular*}
    \caption{Presence of parallelization settings in build files: the
    first column indicates the group of projects according to their
    time cost; the second column indicates the number of build files;
    the third column is the subset of files with parallelization
    keywords; the last column indicates the ratio of projects with
    parallelism support.}
    \label{tab:inspection-table} 
\end{figure}

\subsubsection{Distribution of parallel modes}

From the \numProjectsPar{} projects identified above, we investigated
further the \numPomMatchedValid{} build files with parallel settings.
We analyzed the support and distribution of parallel modes from this
subset of projects. To calculate the distribution of parallel modes,
we considered only the presence of the mode in at least one of the
project settings.  Recall that a build file may contain more than one
parallel setting and a project may contain several sub-modules with
build files.  In case the value of a parallel option is resolved
dynamically (\eg, via command-line argument or system variable) we
compute all modes related to the option. For instance, depending on
the value, the \CodeIn{parallel} option can be \Seq{} (\CodeIn{none}),
\ParClassSeqMeth{} (\CodeIn{classes}), \SeqClassParMeth{},
(\CodeIn{methods}), and \ParClassParMeth{} (\CodeIn{all}).
Figure~\ref{fig:freqmodes} summarizes our findings. \Jbc{Given that we
checked each configuration, we could elaborate how these settings are
used.} \Fix{Missing conclusion: Fork the most used configuration}

\begin{figure}[h!]
    \centering
    \includegraphics[width=0.32\textwidth]{plots/parallel-modes.pdf}
    \caption{\label{fig:freqmodes}Distribution of parallel modes
    identified in a subset of \numProjectsPar{} projects (unused modes
    omitted). A single project may have support to more than one
    parallel mode.}
\end{figure}

\subsection{Answering research question \numRQD{}}
\label{sec:rqD}

\begin{itemize}
    \item \emph{\RQD}
\end{itemize}

To answer \numRQD{}, we considered the \numProjectsPar{} subjects with
parallelization settings identified in \numRQC{} and compared their
time cost with sequential execution. \Jbc{should have a sentence to
motivate this RQ} 
\Jbc{explain methodology}.

\Fix{---------------------}

%%To evaluate the distribution of execution time per project, we sorted
%%the test cases by decreasing order of elapsed time and calculated the
%%number of tests executed in 90\% of the total time. Later, we reported
%%the \Fix{balance} of execution time by dividing the number of tests
%%that represents 90\% of the execution time by the number of tests
%%cases. For instance, a balance of 50\% indicates \Fix{...}.  \Fix{We
%%collected the elapsed time from test cases for each generated report.
%%Maven Surefire generates an XML report with execution information
%%(\eg, number of skipped tests and elapsed time) per test suite
%%\Jbc{Should I use the previous sentence as a footnote or should I
%%delete it?}. We noticed that some test cases reported an elapsed time
%%of zero: since the reported time is in milliseconds, some tests may
%%execute in a shorter time. \Fix{..to be continued...}}. Results
%%indicate that \Fix{...}.
%%
%%\begin{figure}[h!]
%%    \centering
%%    \includegraphics[width=0.4\textwidth]{results/plots/balance.pdf}
%%    \caption{\Fix{balance}}
%%\end{figure}

%% \subsection{Answering research question RQ3}
%% \label{sec:rqThree}
%% 
%% \begin{itemize}
%%     \item \RQB
%% \end{itemize}
%% 
%% To evaluate the distribution of CPU and IO intensive test suites from
%% the sample set, we used the command \emph{sar} to monitor the system
%% activity in background while tests ran. \emph{Sar} is a command that
%% collects and reports statistics (\eg, percentage of IO waiting and
%% usage of CPU in user mode) based on the kernel activity and it is
%% highly configurable to collect detailed information (\eg, usage of a
%% specific processor core or percentage of network interface
%% utilization). We configured \emph{sar} to report \Fix{...explain how
%% we executed and what fields we are interested}. \Fix{explain fields}.
%% Figure \Fix{A} shows the distribution of subjects grouped in intervals
%% of \Fix{W}\% of CPU utilization. Results indicates that \Fix{...}
%% 
%% \begin{figure}[h!]
%%     \centering
%%     \includegraphics[width=0.4\textwidth]{results/plots/cpuness.pdf}
%%     \caption{\Fix{cpu usage}}
%% \end{figure}
%% 
%% \Comment{we proposed the definition of \emph{cpuness}
%% computed as the follow: $((user\_t + system\_t) / elapsed\_t) * 100$,
%% where \emph{user\_t} is the elapsed time of execution in \emph{user
%% mode}, \emph{system\_t} is the elapsed in \emph{kernel mode}, and
%% \emph{elapsed\_t} is the elapsed time to finish the execution. We
%% measured the \emph{cpuness} of each regression test \Fix{...elaborate
%% the meaning of cpuness} \Fix{Describe how I measured user, system and
%% "wall" time}.  \Fix{Explain results}.  \Fix{show plots}}
%% 


\Comment{We followed a similar methodology to group projects by time as
Gligoric~\etal{}~\cite{gligoric-etal-issta2015} in their work on
regression test selection.}
\Comment{ and added the \medg{} group due to the variability of the
time cost from subjects out of the \shortg{} group}

We ran the test suite for each subject three times, reporting averaged
execution times in three ranges: tests that run within a minute
(\shortg{}), tests that run in one to five minutes (\medg{}), and
tests that run in five or more minutes (\longg{}). Figure~\ref{fig:rq1-barplot} shows the number of projects in
each group.  As expected, \longg{} and \medg{} projects do not occur
as frequently as \shortg{} projects.  However, they do occur in
relatively high numbers.
Figure~\ref{fig:rq1-boxplot} shows the distribution of execution time of
test suites in each of these groups.
Note that the y-ranges are different.
The distribution associated with the \shortg{} group is the most
unbalanced (right skewed)\Comment{ with outliers closed to the \medg{}
group}.
The test suites in this group ran in 15 or less seconds for
over 75\% of the cases.\Comment{  Such scenarios constitute the majority of the
cases we analyzed.} Considering the groups \medg{} and \longg{},
however, we found many costly executions.  Nearly 75\% of the projects
from the \medg{} group take 3.5 or more minutes to run and nearly 75\% of
the projects from the \longg{} group take $\sim$20 minutes to run.  We
found cases in the \longg{} group were execution takes more than 50 minutes
to complete, as can be observed from the outliers in the boxplot.

\begin{figure}[ht]
  \centering
  \begin{subfigure}{0.182\textwidth}
    \centering
    \includegraphics[width=\textwidth]{results/barplot-timecost.pdf}
    \caption{\label{fig:rq1-barplot}}
  \end{subfigure}%
  \hfill
  \begin{subfigure}{0.25\textwidth}
    \centering
    \includegraphics[width=\textwidth]{results/boxplot-timecost.pdf}
    \caption{\label{fig:rq1-boxplot}}
  \end{subfigure}%
  \caption{(a)\Comment{ Testing time grouped by time cost ($t$): short run
    ($t<1m$), medium run ($1m \le t < 5m$), and long run ($5m \le{}
    t$);}~Number of projects in each cost group and
    (b)~Distribution of running times per cost group.}
  \vspace{-1ex}
\end{figure}

It is important to note that we under-estimated running times as we missed test modules not enabled for
execution in the root \emph{pom.xml}.\Comment{Some projects may omit long-running tests on their default
execution.} For instance, the project \CodeIn{apache.maven-surefire}
runs all unit tests in a few seconds.  According to our criteria, this
project is classified as \shortg{} but a closer look reveals
that only smoke tests are executed in this project by default.
In this project, integration and system tests, which take longer to run, are only accessible via
custom parameters, which we do not handle in our experimental setup.
We enabled such parameters for this specific project and observed that
testing time goes to nearly 30 minutes.  For simplicity, we considered
only the tests executed by default.
\Comment{
Although failing tests may introduce noise in our measurements, we
carefully eliminated subjects with more than \SuiteFailingThreshold{} of
test failures. 
}
From the \numSubjs{} testable projects, \numSubjsPass{} successfully
executed all tests and \numSubjsFail{} reported some test failure.
From these \numSubjsFail{} subjects, only 11 subjects
have more than 5\% of failing tests (7.3\% on average).
\begin{center}
\vspace{1ex}
\fbox{
\begin{minipage}{8cm}
  \textit{Answering \numRQFeasibilityOne{}:}~\emph{We conclude that
    time-consuming test suites are relatively frequent in
    open-source projects.  We found that \percentMedLongRunning{} of
    the \numSubjs{} projects we analyzed (\ie{}, nearly 1 in every 4
    projects) take at least 1 minute to run and
    \percentLongRunning{} of them take at least 5 minutes to run.\Comment{
      (\ie, \numMedLong{} projects from \medg{} and \longg{}).}}
\end{minipage}
}
\vspace{1ex}
\end{center}

\begin{itemize}
  \item \numRQFeasibilityTwo. \textbf{\RQFeasibilityTwo}
\end{itemize}

\begin{figure}[t!]
  \centering
  \includegraphics[width=.48\textwidth]{results/testcost-distribution.pdf}
  \caption{\label{fig:time-distributions}Distribution of test case time per project.}%
  \vspace{-3ex}
\end{figure}

Section~\ref{sec:rqA} showed that medium and long-running projects are
not uncommon, accounting to nearly \percentMedLongRunning{} of the
\numSubjs{} projects we analyzed.  Research question \numRQFeasibilityTwo{}
measures the distribution of test costs in test suites.\Comment{ as to estimate
potential of obtaining speedups with parallelization.}  In
the limit, if cost is dominated by a single test from a large test
suite, it is unlikely that parallelization will be beneficial as a
test method is the smallest working unit in test frameworks.
%% However, avoiding
%% frequent context switches is another factor to consider.  For example,
%% assuming there are at least two CPUs available for execution, cost can
%% be cut in half if two tests in a large test suite dominate execution
%% time and these tests are assigned to different CPUs.
%% It is therefore important to
%% speedup regressing testing in open-source projects.\Comment{not only
%%   to huge projects as those from Google~\cite{google-tap,google-ci}
%%   and Microsoft~\cite{prasad-shulte-ieee-microsoft-ci}.}
%% For the case
%% where cost is distributed more evenly across test cases, one expects
%% that speedups will be a function of the number of cores.
%% These contradictory forces, pushing number of tests and cost
%% of each test up and down, make prediction of effectiveness challenging.
Figure~\ref{fig:time-distributions} shows the time distribution of
individual test cases per project.
We observed that the average median times (see dashed horizontal red
lines) were small, namely 0.08s
for \medg{} projects and 0.16s for \longg{} projects, and the standard deviations associated with each distribution were
relatively low.\Comment{ Figure~\ref{fig:sd} shows the number of
  projects within specific ranges of $\sigma$ values.} High values of
$\sigma$ are indicative of CPU monopolization. We found only a small number
of those. The highest value of $\sigma$ occurred in
\CodeIn{uber\_chaperone}, a project from the \longg{} group.
This project contains only 26 tests, 17 of which take less than 0.5s
to run, one of which takes nearly 3s to run, two of which take nearly
11s to run, four of which takes on average 3m to run, and two of which
take $\sim$8m to run.
For this project, 98.4\% of the execution cost is dominated by 20\% of
the tests; without these two costly tests this project would have been
classified as short-running.
We did not find other projects with such extreme time monopolization
profile.
Project \CodeIn{facebookarchive\_linkbench} is also classified as
long-running and has the second highest value of $\sigma$.
For this project, however, cost is distributed more smoothly across 98
tests, of which 8 (8.1\%) take more than 1s to run with the rest of
the tests running faster.

Figure~\ref{fig:size-testsuites} shows the difference in the
distribution of test suite sizes across groups.  This figure indicates
that long projects have a higher median and much higher average
number of test cases.
Furthermore, we noted a strong positive correlation between running
time and number of test on projects in the \longg{} group.
Considering the \medg{} group, the correlation between these two
variables was weak.
Figure~\ref{fig:scattercost} illustrates the regression lines between
these the variables test suite cost and number of test cases.
To sum, we observed that for projects with long-running test suites
running time is typically
justified by the number of test cases as opposed to the cost of individual test cases.

%% Figure~\ref{fig:time-distributions} shows that the average median
%% times are similar for \medg{} and \longg{}-running test suites.

\begin{center}
\fbox{
\begin{minipage}{8cm}
  \textit{Answering \numRQFeasibilityTwo{}:}~\emph{Overall, results indicate that
  projects with a very small number of tests monopolizing end-to-end
  execution time were rare. Time most often is distributed evenly
  across test cases.}
\end{minipage}
}
\begin{figure}[h!]
  \centering
  \begin{subfigure}{0.13\textwidth}
    \centering
    \includegraphics[width=.9\textwidth]{results/boxplots-testcases.pdf}
    \caption{\label{fig:size-testsuites}}
  \end{subfigure}
  \hspace{3ex}
  \begin{subfigure}{0.28\textwidth}
    \centering
    \includegraphics[width=.9\textwidth]{results/scatter-testcost.pdf}
    \caption{\label{fig:scattercost}}
  \end{subfigure}
  \caption{\label{fig:time-versus-size}(a) Size of test suites; (b)
    Size versus running time of test suites.}%
\end{figure}
\end{center}

%% We are interested to know whether
%% most of the execution cost of a subject is dominated by a small subset
%% of test cases or if the cost is nearly equally distributed. 

%% We also evaluated the dispersion of time distributions (one
%% distribution per project) to answer research question \numRQFeasibilityTwo{}.  To
%% measure dispersion \emph{across} projects we used Relative Standard
%% Deviation (RSD)~\cite{everitt-book-stats-2010}.  Note that, if we were
%% to analyze each project in isolation, the standard deviation of a
%% distribution ($\sigma$) would suffice to quantify how dispersed the
%% (time) distribution is.  However, in our case, we would like to be
%% able to compare and summarize dispersion across projects.  The RSD,
%% which is obtained dividing the standard deviation by the mean ($\mu$)
%% of a distribution, provides such normalization effect.  This metric
%% provides a lower bound (zero) but not an upper bound (somewhere close
%% to 1).  The smaller (larger) the value of RSD the more (less) uniform
%% the distribution is.  Consequently, the lower the value of RSD the
%% more parallelizable a test suite should be.

%% \begin{figure}[h!]
%%   \centering
%%   \includegraphics[width=0.5\textwidth]{R/testcost.pdf}  
%%   \caption{\label{fig:relativesd}Distribution of RSD ($\sigma/\mu$)
%%     across projects.}
%% \end{figure}

%% Figure~\ref{fig:relativesd} shows the distribution of RSD across
%% medium and long-running projects.  Results show that the distribution
%% is skewed to the right indicating that test costs are relatively well
%% distributed in most costly projects we analyzed \Fix{$\leftarrow$
%%   confirm}.

%% analyzed the execution time
%% for the \numMedLong{} projects from the \longg{} and \medg{} groups
%% (see Section~\ref{sec:rqA}).
%% For each subject we calculated the
%% relative standard deviation of the test cases: we collected the
%% elapsed time of each individual test, calculated the standard
%% deviation, and divided by the mean. \Jbc{I need to clarify the
%%   relationship "well/bad-balanced" regression test and relative
%%   standard deviation}

%% Results indicated that \Fix{...elaborate...}. \Jbc{We may identify
%% different groups of subjects}\Fix{TODO: collect data + compute the
%% statistic, create a scatter plot to identify groups of subjects}

%% Regression tests that are well distributed may benefit from
%% parallelism since more tests executes at the same time while the
%% opposite scenario may require a different approach. In the later
%% scenario, executing tests in parallel may have insignificant impact
%% since a small subset of test cases dominates the execution.}


\subsection{Adoption}
\label{sec:rqC}
\label{sec:rqE}

%% Commented as we don't follow this pattern on the other RQs
\Comment{
Adoption concerns with the popularity of  parallelization schemes and
(\numRQAdoptionTwo) how developers involved in costly projects, not
using parallelization, perceive this technology.
}
\begin{itemize}
    \item \numRQAdoptionOne. \textbf{\RQAdoptionOne{}}
\end{itemize}

To answer \numRQAdoptionOne{} we used projects from the \medg{} and
\longg{} groups where parallelization can be more helpful. We used a
dynamic and a static approach to find manifestations of
parallelism. We discuss results obtained with these complementary
approaches in the following.

\vspace{1ex}
\subsubsection{Dynamic checking}
\label{sec:rqC-1}

To find dynamic evidence of parallelism, we ran the test suites from
our set of \numMedLong{} projects to output all key-value pairs of
Maven parameters.  To that end, we used the option~\CodeIn{-X} to
produce debug output and the option~\CodeIn{-DskipTests} to skip
execution of tests.  We skipped execution of tests as we observed from
sampling that only bootstrapping the Maven process suffices to infer
which parallel configuration modes it uses to run the
tests.  It is also important to point that we used the default
configurations specified in the project.  We inferred parallel
configurations by searching for certain configuration parameters in
log files. According to Maven's
documentation~\cite{maven-surefire-plugin}, a parallel configuration
depends either on (1) the parameter \CodeIn{parallel} to define the
parallelism mode within a JVM followed by the parameter
\CodeIn{threadCount} or (2) the parameter
\CodeIn{forkCount}\footnote{This parameter is named \CodeIn{forkMode}
  in old versions of Maven Surefire.} to define the number of forked
JVMs.  As such, we captured, for each project, all related key-value
pairs of Maven parameters and mapped those pairs to one of the
possible parallelization modes.  For instance, if a given project
contains a module with the parameter
\CodeIn{<forkCount>1C</forkCount>}, the possible classifications are
\ForkSeq{} or \ForkParMeth{}, depending on the presence and the value
of the parameter \CodeIn{parallel}.  If the parameter
\CodeIn{parallel} is set to \CodeIn{methods} the detected mode will be
\ForkParMeth{}.  Large projects may contain several test suites
distributed on different Maven modules potentially using different
configurations.  For those cases, we collected the Maven output from
each module discarding duplicates as to avoid inflating counts for
configuration modes that appear in several modules of the same
project. For instance, if a project contains two modules using the
same configuration, we counted only one occurrence.
Considering
our set of \numMedLong{} projects, we found that only
\numProjectsPar{} of those projects had parallelism enabled
by default, with only configurations \ParClassSeqMeth{},
\ParClassParMeth{}, and \ForkSeq{} being used. Configurations
\ParClassParMeth{} and \ForkSeq{} were the most popular among these
cases. Note that these results under-approximate real usage of
parallelism as we used default parameters in our scripts to spawn the
Maven process.  That decision could prevent execution of particular
test modules. Table~\ref{tab:freqmodes-dynamic} shows the
\numProjectsPar{} projects we identified where parallelism is enabled by default in Maven.
\begin{wraptable}{r}{0pt}%0.525\linewidth
  \footnotesize
  \centering
  \setlength{\tabcolsep}{2.5pt}
    \begin{tabular}{llcr}
        \toprule
        \multirow{2}{*}{\emph{Group}} & \multirow{2}{*}{\emph{Subject}} & \emph{\# of} & \multirow{2}{*}{\emph{Mode}}\\%
                                      &                                 & \emph{modules} &\\%
        \midrule%
        Medium & Californium & 2/20 & C2\\%
        Medium & Chaos & 1/1 & C2\\%
        Long & \Comment{apache} Flink & 66/74 & FC0\\%
        Long & \Comment{apache logging-}Log4J2 & 25/28 & FC0\\%
        Long & \Comment{javaslang }Javaslang & 3/3 & C3\\%
        Medium & Jcabi \Comment{jcabi-github} & 1/1 & C3\\%
        Long & \Comment{hazelcast hazelcast-}Jet & 6/7 & FC0\\%
        Long & \Comment{apache} Mahout & 8/9 & FC0\\%
        Long & \Comment{jankotek} MapDB & 1/1 & C3\\%
        Medium & \Comment{apache} OpenNLP & 4/4 & FC0\\%
        Medium & \Comment{yegor256} Rultor & 1/1 & C3\\%
        Medium & \Comment{yegor256} Takes & 1/1 & C3\\%
        Long & \Comment{vavr-io} Vavr & 3/3 & C3\\%
        \bottomrule%
    \end{tabular}
    \caption{Subjects with parallel test execution enabled by
    default.}
    \label{tab:freqmodes-dynamic}
%    \vspace{-1ex}
\end{wraptable}
Column ``\emph{Subject}'' indicates the name of the project, column
``\emph{\# of modules}'' indicates the fraction of modules containing
tests that use the configuration of parallelism mentioned in column
``\emph{Mode}''.
We note that, considering these projects, the modules that do not use
the configuration cited use the sequential configuration \Seq{}.
For example, three modules (=28-25) from Log4J2 use sequential
configuration. It came as a surprise the observation that
no project used distinct configurations in their modules. 

\subsubsection{Static checking}
\label{sec:rqC-2}
Given that the dynamic approach cannot detect parallelism manifested
through the default
configuration of projects, we also searched for indications of parallelism in build
files\Comment{ in the same sample set of \numMedLong{} projects}.  We
parsed all \emph{pom.xml} files under the project's directory and used
the same approach as in our previous analysis to classify
configurations.  We noticed initially that our approach was unable to
infer the configuration mode for cases where the decision depends on
the input (\eg,
\CodeIn{<parallel>\$\{parallel.type\}</parallel>}). For these
projects, the tester needs to provide additional parameters in the
command line to enable parallelization (\eg, \CodeIn{mvn test
  -Dparallel.type=classesAndMethods}). To handle those cases, we
considered all possible values for the parameter (in this case,
\CodeIn{\$\{parallel.type\}}).  It is also important to note that this
approach is not immune to false negatives, which can occur when
\emph{pom.xml} files are encapsulated in jar files or files downloaded from
the network.  Consequently, this approach complements the
the dynamic approach. Overall, we found \numProjectsParStatic{}
projects manifesting parallelism with this approach.
Compared to the set of projects listed in
Table~\ref{tab:freqmodes-dynamic}, we found four new projects, namely:
\CodeIn{Google Cloud\Comment{ Platform} DataflowJavaSDK} (using
configuration \ParClassParMeth), \CodeIn{Mapstruct} (using configuration
\ForkSeq{}), \CodeIn{T-SNE-Java} (using configuration \ForkSeq), and
\CodeIn{Urbanairship Datacube} (using configuration \ParClassParMeth).
Curiously, we also found that project \CodeIn{Jcabi}, \CodeIn{Rultor},
and \CodeIn{Takes} were not detected using this methodology.
That happened because these projects loaded a \emph{pom.xml} file from
a jar file that we missed.
Considering the static and dynamic methods together, we found a total
of \numProjectsParTotal{} distinct projects using parallelism,
corresponding to the union of the two subject sets.

\vspace{1ex}
\begin{center}
\fbox{
\begin{minipage}{8cm}
  \textit{Answering \numRQAdoptionOne{}:}~\emph{Results indicate that test
  suite parallelization is underused.  Overall, only
  \percentParallel{} of costly projects (\numProjectsParTotal{} out of \numMedLong)
  use parallelism.}
\end{minipage}
}
\end{center}
\vspace{1ex}

\begin{itemize}
  \item \numRQAdoptionTwo{}. \textbf{\RQAdoptionTwo{}}
\end{itemize}

To answer this research question we surveyed developers involved in a
selection of projects from our benchmark with time-consuming test
suites.  The goal of the survey is to better comprehend developer's
attitude towards the use of parallelism as a mechanism to speedup
regression testing.  We surveyed developers from a total of
\emailsProjects{} projects.  From the initial list of \numMedLong{}
project, we discarded 11 projects that we knew a priori used
parallelization, and \discartedProjects{} projects that we could not find
developer's emails from commit logs.  From this list of projects, we
mined potential participants for our study.  More precisely, we
searched for developer's name and email from the last 20 commits to
the corresponding project repository.  Using this approach, we
identified a total of \emailsSent{} eligible participants.  Finally,
we sent plain-text e-mails, containing the survey, to those developers.  In
total, \emailsAnswered{} developers replied but we discarded
\emailsFalseAnswers{} replies with subjective answers.  Considering
projects covered by the answers, a total of \emailsProjectsAnswered{}
projects (\percEmailsProjectsAnswered{} of the total) were represented
in those replies. Note that multiple developers on each project
received emails. In one specific case, one developer worked in
  multiple projects, and we consider it as a different answer. We sent the following set of questions to
developers:

\begin{enumerate}
\item How long does it take for tests to run in your environment? Can
  you briefly define your setup?
\item Do you confirm that your regression test suite does *not* run in parallel?
\item\label{questionThree} Select a reason for not using parallelization:
  \begin{enumerate}[label=\alph*)]
  \item I did not know it was possible
  \item I was concerned with concurrency issues
  \item I use a continuous integration server
  \item Some other reason. Please elaborate.
  \end{enumerate}
\end{enumerate}
%% \begin{enumerate}
%% 	\item How long does it take for test to run in your
%%		environment?
%%	\item Can you briefly define your setup?
%%	\item Do you confirm that your project does not run in
%%		parallel?
%%	\item Select a reason for not using paralellization:
%%		\begin{enumarate}
%%			\item I did not know it was possible;
%%			\item I was concerned with concurrency issues;
%%			\item I use a continuous integration server;
%%			\item Some other reason.
%%		\end{enumerate}
%% \end{enumerate}
%% One of the goals of the first questions is to identify potential
%% discrepancies between our experimental environment and the environment
%% of developers.  Overall, we found that \Fix{...}

Considering question 1, we confirmed that execution time was
compatible with the results we reported in Section~\ref{sec:rqA}.
Furthermore, \emailsCI{} of the participants indicated the use of
Continuous Integration (CI) to run tests, with \emailsDistributed{} of
these participants reporting that test suites are modularized and
those modules are tested independently in CI servers through different
parameters.  Those participants explained that such practice helps to
reduce time to observe test failures, which is the goal of speeding up
regression testing.  A total of \emailsLocal{} participants answered
that they do run tests in their local machines.  Note, however, that
CI does not preclude low-level parallelization.  For example,
installations of open-source CI tools (\eg{}, Jenkins~\cite{jenkins})
in dedicated servers would benefit from running tests faster through
low-level test suite parallelization.
% \emailsNotDescribed{} developers did not described their environment.

Considering question 2, the answers we collected indicated, to our
surprise, that six of the \emailsProjectsAnswered{} projects execute
tests in parallel.  This mismatch is justified by cases where neither
of our checks (static or dynamic) could detect presence of
parallelism.  A closer look at these projects revealed that one of
them contained a \emph{pom.xml} file encapsulated in a jar file
(similar case as reported in Section~\ref{sec:rqC-2}), in one of the
projects the participant considered that distributed CI was a form of
parallelism, and in four projects the team preferred to implement
parallelization instead of using existing features from the testing
framework and the build system~---~in two projects the team
implemented concurrency control with custom JUnit test runners and in
two other projects the team implemented concurrency within test
methods.  Note that, considering these four extra cases (ignored two
distributed CI cases), the usage of parallelization increases from
\percentParallel{} to \percentParallelUpdated{}.  We do not consider
this change significant enough to modify our conclusion about
practical adoption of parallelization (\numRQAdoptionOne{}).
%% , one runs a manually created Thread to run some
%% tests, and the other runs in parallel by using Java 8 collection
%% streams, that allows the developers to iterate over a list in
%% parallel.
%% did not confirmed, however,
%% the developers confirmed the need of an extra parameter at the command
%% line to execute in parallel.

Considering question 3, the distribution of answers was as follows.  A
total of \emailsA{} of the \emailsProjectsAnswered{} developers who
answered the survey did not know that parallelism was available in
Maven (option ``a''), \emailsB{} of developers mentioned that they did
not use parallelism concerned with possible concurrency issues (option
``b''), \emailsD{} of developers mentioned that continuous integration
suffices to provide timely feedback while running only smoke
\begin{wrapfigure}{r}{0.15\textwidth}
    \centering
    \includegraphics[width=.15\textwidth]{results/survey.pdf}
    \caption{\label{fig:rq5-answers}Summary of developer's answers to
      survey question~\ref{questionThree}.}
\end{wrapfigure}
tests (\ie{}, short-running tests) locally (option ``c'')\Comment{here
  I want to say that they use it for something like "non-blocking
  testing" while developing in a local machine}, and \emailsD{} of
developers who provided an alternative answer (option ``d'') mentioned
that using parallelism was not worth the effort of preparing the test
suites to take advantage of available processing power.  A total of
\emailsNA{} of participants did not answer the last question of the
survey.  The pie chart in Figure~\ref{fig:rq5-answers} 
summarizes the distribution of answers.

\begin{center}
\vspace{1ex}
\fbox{
\begin{minipage}{8cm}
  \textit{Answering \numRQAdoptionTwo{}:}~\emph{Results suggest that dealing
    with concurrency issues (\ie{}, the extra work to organize test
    suite to safely explore concurrency) was the principal reason
    for developers not investing in parallelism.  Other reasons
    included availability of continuous integration services and
    unfamiliarity with the technology.}
\end{minipage}
}
\vspace{1ex}
\end{center}

\subsection{Speedups}
\label{sec:rqD}

\begin{itemize}
    \item \numRQSpeedupOne{}. \textbf{\RQSpeedupOne}
\end{itemize}

%% In those projects, parallelization is active without
%% passing any extra parameters.  Section~\ref{sec:rqC-1} describes in
%% detail the methodology we used to find these subjects.
%and
%Section~\ref{seq:rq6-tradeoffs}, we verified that both
%% executions produce the same outcome to eliminate noise from failing
%% tests.  To compute the speedup, we divide the time obtained in the
%% sequential execution by the time obtained from the default execution.
%% For instance, if a project runs the tests sequentially in $10m$ and
%% the same execution runs in $5m$ with parallelization enabled (default
%% execution), the speedup is two.

To answer \numRQSpeedupOne{}, we considered the \numProjectsPar{}
subjects from our benchmark that use parallelization \emph{by default}
(see Table~\ref{tab:freqmodes-dynamic}).  We compared running times
of test suites with enabled parallelization, as configured by project
developers, and without parallelization.
Table~\ref{tab:speedup} summarizes results.
Lines are sorted by project names.
Columns ``\emph{Group}'' and ``\emph{Subject}'' indicate, respectively,
the cost group and the name of the project.
Column ``$T_s$'' shows sequential execution time and column ``$T_p$''
shows parallel execution time.
Column ``$T_s/T_p$'' shows speedup or slowdown.
As usual, a ratio above 1x indicates speedup and a ratio below 1x
indicates slowdown.

\begin{table}[t!]
\centering
\resizebox{.41\textwidth}{!}{%
  \scriptsize
\begin{tabular}{llrrr}
\toprule
\emph{Group} & \emph{Subject} & \multicolumn{1}{c}{$T_s$} & \multicolumn{1}{c}{$T_p$} & $T_s/T_p$ \\%
\midrule%
Medium & Californium & 1.45m & 1.40m & \cellcolor{lightgray}1.04x\\%
Medium & \Comment{BounceStorage} Chaos\Comment{ HTTP Proxy} & 1.51m & 1.47m & \cellcolor{lightgray}1.03x\\%
Medium &\Comment{ Apache }Flink& 11.79m & 2.57m & 4.59x\\%
Long &\Comment{ Apache }Log4J2& 8.24m & 8.21m & \cellcolor{lightgray}1.00x\\%
Medium &Javaslang& 2.18m & 1.82m & 1.20x\\%
Medium &Jcabi\Comment{ GitHub} & 2.76m & 0.30m & 9.20x\\%
Long &\Comment{ Hazelcast }Jet& 8.26m & 3.67m & 2.25x\\%
Long & \Comment{apache} Mahout & 27.38m & 18.15m & 1.51x\\%
Long &\Comment{ Jankotek }MapDB& 10.06m & 8.58m & 1.17x\\%
Medium & \Comment{apache} OpenNLP & 1.30m & 0.55m & 2.36x\\%
Medium & \Comment{yegor256} Rultor & 2.30m & 0.27m & 8.52x\\%
Medium & \Comment{yegor256} Takes & 2.00m & 0.19m & \cellcolor{lightgray}10.53x\\%
Long & \Comment{vavr-io} Vavr & 3.26m & 2.25m & 1.45x\\%
\midrule
\textbf{average} &  &  &  & \avgSpeedup{}x\\
\bottomrule%
\vspace{-5ex}
\end{tabular}}
\caption{\label{tab:speedup}Speedup (or slowdown) of parallel
  execution ($T_p$) over sequential execution ($T_s$).  Default
  parallel configuration of Maven is used.  Highest slowdown/speedup
  appears in gray color.}
\end{table}

Results show that, on average, parallel execution was
\avgSpeedup{} times faster compared to sequential execution.
Three cases worth special attention: \CodeIn{Log4J2}, \CodeIn{Chaos},
and \CodeIn{Takes}.
We note that parallel execution in \CodeIn{Log4J2} was
ineffective.  We found that Maven invokes several test modules in this
project but the test modules that dominate execution time run
sequentially by default. This was also the case for the highlighted
project \CodeIn{Californium}.
No significant speedup was observed in \CodeIn{Chaos}, a project with
only three test classes, of which one monopolizes the bulk of test
execution time.
This project uses configuration \ParClassSeqMeth{}, which runs test
classes in parallel but runs test methods, declared in each class,
sequentially.
Consequently, speedup cannot be obtained as the cost of the single
expensive test class cannot be broken down with the selected
configuration.
Finally, the speedup observed in project \CodeIn{Takes} was
the highest amongst all projects. This subject uses configuration
\ParClassParMeth{} and contains 419 test methods distributed nearly
equally among 148 test classes with a small number of test methods.
Furthermore, several methods in those classes are time-consuming.
As result, the CPUs available for testing are kept occupied for the
most part during test execution.

%% The third, runs parallel configuration in
%% \Fix{80\%} of the project modules, however, the test time is dominated
%% by one of the sequentially running modules.  \Luis{$\leftarrow$ rework
%%   this} \Fix{falar sobre o resultado geral dos speedups - elaborar
%%   menor e maior speedup... acho que so vale a pena discutir quando
%%   tiver conviccao dos 2 casos}
\begin{center}
\vspace{1ex}
\fbox{
  \begin{minipage}{8cm}
    \textit{Answering \numRQSpeedupOne{}:}~\emph{Considering the
      machine setup we used, the average speedup observed with default
      configurations of parallelization was \avgSpeedup{}x.
      }
  \end{minipage}
}
\vspace{1ex}
\end{center}

\begin{itemize}
    \item \numRQSpeedupTwo{}. \textbf{\RQSpeedupTwo}
\end{itemize}

\newcommand{\subjectScalability}{MapDB}

This experiment evaluates the impact of making a growing number of
CPUs available to the build system for testing.  For this reason, we
used a different machine, with more cores, compared to the one described in
Section~\ref{sec:setup}.  We used a Xeon E5-2660v2 (2.20GHz) Intel
processor machine with 80 virtual CPUs (40 cores with two native
threads each) and 256GB of memory, running Ubuntu 14.04 LTS Trusty
Tahr (64-bit version). This experiment
spawns a growing number of JVMs in different CPUs, using parallel
configuration \emph{\ForkSeq{}}. We selected
subject \subjectScalability{} in this experiment as it represents the
case of a long-running test suite (see Table~\ref{tab:speedup}) with
test cases distributed across many test classes~--~194.  
Recall that a test class is the smallest unit that can be used to spawn a test job
on a JVM and that we have no control over which test classes will be
assigned to which JVM that the build system forks.
Figure~\ref{fig:scalability} shows the reduction in running times as
more CPUs contribute to the execution.
We ran this experiment for a growing number of cores 1, 3, ..., 39. 
The plot omits results beyond 17 cores as the tendency for higher
values is clear.
We noticed that improvements are marginal after three cores, which is
\begin{wrapfigure}{r}{0.23\textwidth}
  \vspace{-1ex}
  \includegraphics[width=0.23\textwidth]{R/scalability/scalability.pdf}
  \caption{\label{fig:scalability}Scalability.}
  \vspace{-2ex}
\end{wrapfigure}
the basic setup we used in other experiments.
This saturation is justified by the presence of a single test class,
\CodeIn{org.mapdb.WALTruncat}, containing 15 test cases that take over
two minutes to run.

\begin{center}
\fbox{
  \begin{minipage}{8cm}
    \textit{Answering \numRQSpeedupTwo{}:}~\emph{Results suggest that
      execution FC0 scales with additional cores but there is a bound
      on the speedup that one can get related to how well the test suite is
      balanced across test classes.}
  \end{minipage}
}
\vspace{1ex}
\end{center}

\subsection{Tradeoffs}
\label{sec:rq6-tradeoffs}

%%  We are
%% interested in understanding how efficiency (\ie, testing cost) and
%% flakiness (\ie, failing tests) are affected when we run a test suite
%% with different parallel configurations.  

This dimension assesses the impact of using distinct parallel
configurations on test flakiness and speedup.  Increased parallelism
can increase resource contention leading to concurrency issues such as
data races across dependent
tests~\cite{luo-etal-fse2014,bell-etal-esecfse2015}.  Flakiness and
speedup are contradictory forces that could influence the decision of
practitioners about which parallel configuration should be used for
testing.  Note that Section~\ref{sec:rqD} evaluated speedup in
isolation.

\begin{itemize}
  \item \numRQIssuesOne{}. \textbf{\RQIssuesOne{}}
\end{itemize}

%% The intuition is that efficiency and flakiness are inversely
%% proportional in some cases: if too many tests depends on the state of
%% a single external resource, several tests are likely to fail as the
%% degree of concurrency increases by exploiting maximum CPU usage.

%% Table header macros
% $\Uparrow_\text{speed}$
\newcommand{\subcolA}{$\text{speedup}$}
\newcommand{\subcolB}{$\%_\text{fail}$}
\newcommand{\colheader}[1]{\multicolumn{2}{c}{\emph{#1}}}
\newcommand{\blankentry}{\entry{-}{-}}
\newcommand{\subcol}{\subcolA{} & \subcolB{}}
\newcommand{\entry}[2]{#1 & #2}

\begin{table*}[t]
\centering
\small
\setlength{\tabcolsep}{2pt}
\begin{tabular}{l|rr|rr|rr|rr|rr|rr}
\toprule
\multirow{2}{*}{\emph{Subject}} & \multicolumn{2}{c|}{\emph{\Seq}} &
    \colheader{\SeqClassParMeth} & \colheader{\ParClassSeqMeth} &
    \colheader{\ParClassParMeth} & \colheader{\ForkSeq} &
    \colheader{\ForkParMeth} \\ %\cline{2-12}
    & $T$ & $\mathit{N}$ & \subcol{} & \subcol{} & \subcol{} & \subcol{}
    & \subcol{}\\%
\midrule%
Urban Airship Datacube & 8.3m & 36 & 1.9x & 25.0\% & 4.7x & 44.4\% & 1.9x & 25.0\% & 1.0x & 0\% & 1.9x & 25.0\%\\%
MapDB & \entry{8.2m}{5,324}  & \entry{1.5x}{0\%} & \entry{2.7x}{0\%} & \entry{4.8x}{0\%}   & \entry{1.7x}{1.0\%} & \entry{3.4x}{1.0\%}\\%
Facebook Linkbench & 6.1m & 98 & 1.0x & 0\% & 1.6x & 1.0\% & 1.0x & 0\% & 1.7x & 0\% & 1.6x & 0\%\\%
Activiti & 5.9m & 2,029 & 72.1x & 96.3\% & 1.5x & 6.9\% & 75.9x & 96.3\% & 2.9x & 6.6\% & 3.1x & 8.0\%\\%
Stripe Java & \entry{4.3m}{302}  & \entry{4.8x}{6.3\%} & \entry{3.3x}{7.3\%} & \entry{21.5x}{15.0\%} & \entry{2.7x}{0\%} & \entry{8.6x}{11.6\%}\\%
AWS SDK Java Core & \entry{3.7m}{847}  & \entry{2.0x}{2.2\%} & \entry{2.5x}{2.8\%} & \entry{3.7x}{4.0\%} & \entry{1.9x}{0.2\%} & \entry{3.5x}{3.1\%}\\%
Moquette & \entry{3.7m}{169} & \entry{4.6x}{65.6\%} & \entry{3.4x}{33.0\%} & \entry{12.3x}{78.0\%} & \entry{2.5x}{22.5\%} & \entry{9.3x}{69.4\%} \\
JCTools Core & \entry{3.6m}{690}  & \entry{4.5x}{0\%} & \entry{3.6x}{0\%} & \entry{18.0x}{0\%} & \entry{2.8x}{0\%} & \entry{9.0x}{0\%}\\%
Vavr & 3.3m & 22,182 & 1.1x & 32.1\% & 1.4x & 32.1\% & 1.5x & 32.1\% & 1.8x & 0\% & 1.7x & 32.1\%\\%
Javaslang & 3.3m & 17,814 & 1.0x & 18.5\% & 1.4x & 18.5\% & 1.4x & 18.5\% & 1.7x & 0\% & 1.7x & 18.5\%\\%
Bucket4J & 3.0m & 110 & 2.5x & 0\% & 1.3x & 0.9\% & 4.2x & 1.8\% & 1.3x & 0\% & 3.7x & 0\%\\%
Spring Cloud AWS & 2.8m & 538 & 10.5x & 82.3\% & 11.9x & 82.3\% & 10.6x & 82.3\% & 1.0x & 0\% & 10.6x & 82.3\%\\%
Spring Cloud Function & 2.8m & 168 & 6.4x & 77.4\% & 1.3x & 0.6\% & 6.6x & 79.2\% & 1.1x & 0\% & 2.9x & 32.7\%\\%
Jcabi Github & \entry{2.6m}{634} & \entry{2.1x}{0\%} & \entry{17.7x}{0\%} & \entry{28.8x}{0\%} & \entry{2.0x}{0\%} & \entry{2.9x}{0\%} \\%
tabulapdf tabula-java & 2.4m & 186 & 7.2x & 0.5\% & 1.1x & 0\% & 7.2x & 2.7\% & 1.0x & 0\% & 7.2x & 1.6\%\\%
INRIA spoon & 2.3m & 1,042 & 1.2x & 28.8\% & 2.7x & 77.2\% & 1.6x & 56.6\% & 1.8x & 0\% & 1.8x & 29.0\%\\%
Graphaware Neo4J Timetree & 2.1m & 163 & 1.9x & 62.0\% & 3.4x & 73.6\% & 1.9x & 62.6\% & 1.0x & 0\% & 2.1x & 61.3\%\\%
Stream Lib & 2.1m & 149 & 0.9x & 0\% & 2.2x & 0\% & 2.4x & 0.7\% & 2.7x & 0\% & 3.6x & 0\%\\%
GoogleCloud Dataflow Java SDK & \entry{1.6m}{3,345}  & \entry{1.2x}{1.7\%} & \entry{2.7x}{1.0\%} & \entry{0.8x}{5.4\%} & \entry{0.8x}{1.7\%} & \entry{0.8x}{1.7\%}\\%
Walkmod & 1.1m & 141 & 6.3x & 69.5\% & 1.6x & 39.7\% & 2.0x & 66.7\% & 1.1x & 0.0\% & 6.4x & 64.5\%\\%
\midrule
Average & 3.7m & 2,798.4 & 6.7x & 28.4\% & 3.6x & 21.1\% & 10.4x & 31.3\% & 1.7x & 1.6\% & 4.3x & 22.1\%\\%

%%  & \entry{x.xm}{0}  & \entry{}{\%} & \entry{}{\%} & \entry{}{\%} & \entry{}{} & \entry{}{}\\%
%%OpenMRS Core (\CodeIn{api})  & \entry{15.4m}{3436} & \blankentry{}        & \blankentry{}        & \blankentry{} & \entry{1.5x}{0\%} & \entry{1.7x}{0}\\%
%%Apache Flume (\CodeIn{core}) &  \entry{7.7m}{392}  & \blankentry{}        & \blankentry{}        & \blankentry{}       & \entry{0.9x}{0\%} & \blankentry{} \\%
%%Facebook Archive Linkbench   &  \entry{4.5m}{98}   & \entry{1.0x}{0.2\%}  & \entry{1.6x}{1.2\%}       & \entry{1.0x}{0\%}  & \entry{1.7x}{0.5\%}  & \entry{1.7x}{0.2\%}\\%
%%AWS SDK Java (\CodeIn{core}) &  \entry{3.8m}{847}  & \multicolumn{6}{c}{\Fix{requires investigation}} & \entry{2.2x}{0.1\%} & \entry{3.2x}{2.0\%}\\%

%% OpenMRS Core (\CodeIn{api})  & \entry{15.4m}{3436} & \blankentry{}        & \blankentry{}        & \blankentry{} & \entry{1.5x}{0\%} & \entry{1.7x}{0}\\%
%% Jankotek MapDB               &  \entry{9.9m}{5218} & \multicolumn{6}{c}{\cellcolor{lightgray}\emph{JVM Crash}} & \entry{1.5x}{0\%} & \entry{1.7x}{0\%}\\%
%% Apache Flume (\CodeIn{core}) &  \entry{7.7m}{392}  & \blankentry{}        & \blankentry{}        & \blankentry{}       & \entry{0.9x}{0\%} & \blankentry{} \\%
%% Apache Giraph (\CodeIn{core})&  \entry{7.2m}{236}  & \entry{2.1x}{5.1\%}  & \colheader{\cellcolor{lightgray}timeout}   & \entry{1.0x}{0\%} & \entry{1.0x}{0}\% & \colheader{\cellcolor{lightgray}JVM Crash}\\%
%% Facebook Archive Linkbench   &  \entry{4.5m}{98}   & \entry{1.0x}{0.2\%}  & \entry{1.6x}{1.2\%}       & \entry{1.0x}{0\%}  & \entry{1.7x}{0.5\%}  & \entry{1.7x}{0.2\%}\\%
%% Stripe Java                  &  \entry{4.2m}{302}  & \entry{4.2x}{5.2\%}  & \entry{3.5x}{5.7\%}       & \entry{4.2x}{6.3\%} & \entry{1.0x}{0.3\%} & \entry{4.2x}{5.7\%}\\%
%% AWS SDK Java (\CodeIn{core}) &  \entry{3.8m}{847}  & \multicolumn{6}{c}{\Fix{requires investigation}} & \entry{2.2x}{0.1\%} & \entry{3.2x}{2.0\%}\\%

%% Jenkins CI Github            & \entry{Xm}{Z}       & \entry{x}{\%}          & \entry{x}{\%} & \entry{x}{\%} & \entry{x}{\%} & \entry{x}{\%}\\%
%% Jenkins CI Docker Workflow   & \entry{Xm}{Z}       & \entry{x}{\%}          & \entry{x}{\%} & \entry{x}{\%} & \entry{x}{\%} & \entry{x}{\%}\\%
%% Hazelcast                    & \entry{Xm}{Z}       & \entry{x}{\%}          & \entry{x}{\%} & \entry{x}{\%} & \entry{x}{\%} & \entry{x}{\%}\\%

\bottomrule%
\end{tabular}
\caption{Speedup versus Flakiness (\subcolB). Configuration
  \emph{\Seq{}} denotes the comparison baseline, which runs tests
  sequentially.  Columns $T$ and $N$ indicate time and number of
  tests, respectively.  Other columns show speedup and percentage of
  failing tests in different configurations, compared to
  \emph{\Seq{}}.}
\label{tab:rq6-table}
\vspace{-1ex}
\end{table*}


To answer this research question, we selected 15 different subjects,
ran their test suites against all configurations described in
Section~\ref{sec:modes}, and compared their running times and rate of
test flakiness.  We used the sequential execution configuration,
\emph{\Seq{}}, as the comparison baseline in this experiment.  To
select subjects, we sorted projects whose test suites run in 1m or
more by decreasing order of execution time and selected the first
fifteen projects that use JUnit 4.7 or later.  The rationale for this
criteria is to ensure compatibility with parallel configuration since
older versions of JUnit does not support parallel testing.  We ran
each project on each configuration for \SubjectsReruns{} times.
Overall, we needed to reran test suites 270 times, 18 times (3x6
configurations) on each project.  Given the low standard deviations
observed in our measurements and the aggregated time cost of 270 test
executions, we consider \SubjectsReruns{} reruns reasonable for this
experiment.
%% \Jbc{Is this
%%   legit/acceptable/convincing?  $\rightarrow$}The rationale for this
%% criteria was to maximize the diversity of domains from the subjects,
%% assuming these subjects have test suites with different
%% characteristics (\eg, architectural design and rate of cpu/io-bound
%% operations).
%% Notice that with parallelism enabled, the
%% number of failing tests may vary on different runs due to the
%% non-deterministic scheduling of threads and processes.  For that
%% reason, we re-executed the experiment and computed the average speedup
%% and flakiness on each scenario.
%%    The rationale for this selection criteria was to maximize the chances
%%    of observing speedup and flakiness, assuming that long-running tests
%%    also have many tests. Indeed, we confirmed that test suites in these
%%    projects contain at least 236 tests ($1,662.7$ in average).
%% 
%% In a first attempt of this experiment, we considered using Maven to
%% run the subjects on the different parallel configurations. However,
It is worth mentioning that we used custom JUnit runners as opposed to
Maven to run the test suites with different parallel configurations
(see Section~\ref{sec:modes}).  After carefully checking library
versions for compatibility issues and comparing results with JUnit's
we observed that several of Maven's executions exposed problems.  For
example, Maven incorrectly counts the number of test cases executed
for some cases where test flakiness are observed. These issues are
categorized and documented on our website~\cite{ourwebpage} and can be
reproduced with our scripts.  To address those issues we implemented
custom test runners for configurations \emph{\SeqClassParMeth{}},
\emph{\ParClassSeqMeth{}}, and \emph{\ParClassParMeth{}} and
implemented a bash script that coordinates the creation of JVMs and
invokes corresponding custom runners for configurations
\emph{\ForkSeq{}} and \emph{\ForkParMeth{}}.  As to faithfully reflect
Maven's behavior in our scripts, we carefully analyzed the source
code~\cite{maven-surefire-source} of the Maven Surefire plugin. We
implemented test runners using the ParallelComputer class from
JUnit~\cite{junit-parallel}.  

%% For the sequential execution, we invoke the
%% default JUnit test runner.
%% For parallel execution with forked JVMs\Comment{ (\ie, \emph{\ForkSeq{}} and
%% \emph{\ForkParMeth{}})}, we wrote bash scripts to spawn a JVM with our
%% custom runner, mentioned above, to
%% execute a test class in an individual process. 

We used Maven log files to identify test classes to run and used the
Maven dependency plugin~\cite{maven-dep} to build the project's
classpath (with the command \CodeIn{mvn dependency:build-classpath}).
Once we find the tests suite to run and the corresponding classpath,
we invoke the test runners mentioned above on them.  We configured
this experiment to run at most three JVMs in parallel.  Recall that in
our setup (see Section~\ref{sec:setup}), we limited our kernel to use
only three cores and reserved one core for OS-related processes.  To
ensure that our experiments terminate (recall that deadlock or
livelock could occur) we used the \CodeIn{timeout}
command~\cite{timeout-cmd} configured to dispatch a \emph{kill} signal
if test execution exceeds a given time limit. Finally, we save each
execution log and stack traces generated from JUnit to collect the
execution time, the number of failing tests, and to diagnose outliers
in our results.



%% \Jbc{Issues do Maven: (1) Report de contagem de testes, (2) crash de
%% jvm - maven surefire nao tratava algumas exceptions que aconteciam na
%% execucao do teste, (3) timeout}
%% The script we implemented to measure time and flakiness takes a list
%% of tests classes to run as input and outputs the results (\ie, number
%% of test methods, failures, skips, and elapsed time) for each parallel
%% configuration.
%%\Comment{ To obtain parallel configurations, we implemented a script
%%  that takes a subject and a configuration label as inputs (\eg
%%  \CodeIn{mapdb "c0"}), and the script outputs the modified version of
%%  the subject with the desired configuration. The workflow consists in
%%  copying the project directory to a new directory, finding all
%%  existing build files (\ie, \pomf{} files), and modifying all
%%  existing Maven Surefire configurations with new values for the
%%  parameters \CodeIn{parallel}, \CodeIn{forkCount}, and
%%  \CodeIn{threadCount} using an XPath~\cite{xpath} library to
%%  manipulate XML documents. For configurations with forked JVMs
%%  enabled (\ie, \ForkSeq{} and \ForkParMeth{}), we changed the
%%  \CodeIn{forkCount} with the value \CodeIn{1C} (\ie, one JVM per
%%  core).  To adjust the pool of threads for parallelism within a JVM,
%%  we changed the parameter \CodeIn{threadCount} with \Jbc{should we
%%    consider 2 as it is the number of native threads per core OR 6 as
%%    it represents 3 Cores * 2 native threads?}.To run the subjects and
%%  their respective variations (\ie, the modified versions according to
%%  the parallelism configuration), we used a similar approach as
%%  described in Figure~\ref{fig:mvn-execution} except that...  }
%% For efficiency, we reported the speedup (\ie, $T_{\text{s}} /
%% T_{\text{p}}$), and for flakiness, we reported the rate of failing
%% tests (\ie, $\mathit{\%_\text{fail} = failures / tests}$).
\Jbc{Falta atualizar essa discussao....
Table~\ref{tab:rq6-table} summarizes results ordered by subject's
name. Recall that we only executed one module for each subject.
Values are averaged across multiple executions.  We did not report
standard deviations as they are very small in all cases.  Considering
flakiness, results show that 0\% of flakiness have been reported in 26
of the 60 combinations analyzed (43\% of the total) and that for seven
of the ten projects 0\% flakiness was reported in at least one of the
configurations.  The projects with flakiness in all configurations
were \CodeIn{AWS SDK}, \CodeIn{GoogleCloud}, and \CodeIn{Moquette}.
It is worth highlighting the unfortunate case of \CodeIn{Moquette},
which manifested more than 20\% flaky tests in every configuration.
Considering time, it is noticeable from the averages, perhaps as
expected, an increasing speedup from configuration
\emph{\SeqClassParMeth} to \emph{\ParClassParMeth} and from
configuration \emph{\ForkSeq} to \emph{\ForkParMeth}.  It is also
worth mentioning that some combinations manifested slowdown instead of
speedup.  Recall that parallel execution introduces the overhead of
spawning and managing JVMs and threads.
}

\begin{center}
\fbox{
\begin{minipage}{8cm}
\textit{Answering \numRQIssuesOne{}:~Overall results indicate that the
  test suites of 70\% of the projects we analyzed could be run in
  parallel without manifesting any flaky tests.  In some of these
  cases, speedups were significant, ranging from 1x to 28.8x.}
\end{minipage}
}
\end{center}

%% \Jbc{Lembrar que investigar o slowdown de Dataflow... os logs indicam que varios
%% dos testes que quebraram tentavam fazer uma autenticacao em algum servico.
%% Possivelmente o slowdown se deve ao tempo de resposta do servico quando
%% a autentiticao falha}

%%  LocalWords:  RQ occurence parallelization Tradeoffs API readme th
%%  LocalWords:  mvn clearcut escapeinside xleftmargin untestable LTS
%%  LocalWords:  framexleftmargin CPUs Tahr sysstat gh Vagrantfile FC
%%  LocalWords:  javadoc isolcpus JUnit's JUnitCore Gligoric boxplots
%%  LocalWords:  outliers apache uber chaperone facebookarchive AWS
%%  LocalWords:  linkbench priori SDK GoogleCloud Moquette JVMs dir
%%  LocalWords:  xml basicstyle boxpos numberstyle deletekeywords JVM
%%  LocalWords:  uncompilable leftmargin quartile DskipTests forkMode
%%  LocalWords:  threadCount forkCount lrr BounceStorage Flink Gerrit
%%  LocalWords:  JenkinsCI Spotify Javaslang Jcabi Github Hazelcast
%%  LocalWords:  Jankotek MapDB Dparallel classesAndMethods Mapstruct
%%  LocalWords:  DataflowJavaSDK modularized JUnit llrrr lightgray rr
%%  LocalWords:  GitHub amongst Xeon runtime Scalability Dataflow sdk
%%  LocalWords:  JCTools RipMe classpath ParallelComputer livelock io
%%  LocalWords:  mapdb workflow XPath boxplot llcr javaslang jcabi
%%  LocalWords:  github hazelcast jankotek OpenNLP yegor Rultor vavr
%%  LocalWords:  Vavr SNE Urbanairship Datacube
