\section{Introduction}

Optimizing regression testing is important.  As software evolves the
size of test sets increases, potentially leading to disruption in the
development process caused by late reports of test failures.  This
problem manifests more dramatically at large IT companies
(e.g.,\Comment{ Groupon~\Fix{cite},} Google~\Fix{cite} and
Microsoft~\Fix{cite}) where the size of test sets often exceeds the
size of the code base.  Several approaches have been proposed to
optimize regression testing.  Research has focused mostly on test
selection~\Fix{cite}, test reduction~\Fix{cite}, and test
prioritization~\Fix{cite}.  Although these approaches can also be
applied in industrial settings, \Fix{the focus in industry has been
mostly in parallelizing execution of tests using server farms, /Is
this an assumption? Is there any evidence?} where worker machines
from the farm are assigned to execute distinct subsets of the original
test suite.

Given the proliferation of multicore machines it is not surprising
that popular build systems and testing frameworks provide today
support for parallel test execution with the goal of running tests
more efficiently~\cite{junit-org,testng,nunit,maven-surefire-plugin}.
The lower-level parallelism enabled through build systems and testing
frameworks is an important complement to the higher-level parallelism
enabled through server farms.  These solutions enable the use of
commodity hardware to maximize CPU usage\footnote{In the case of the
  Java language, for example, it is possible to explore parallelism
  across and within JVMs.}.  For the scenario of large IT
organizations, lower-level parallelization schemes could leverage the
computing power of server nodes in addition to the aggregate
processing power of the farm.  Lower-level parallelism fits
particularly well smaller organizations (/projects) with relatively
high testing costs but lower budgets.  Unfortunately, these solutions
can't be used out-of-the-box: unrestricted parallel execution of tests
can produce non-deterministic results as developers typically do not
provision protection to concurrent accesses originated from arbitrary
program points (\ie{}, tests).  We refer to this problem as Parallel
Execution Flakiness (\pef{}).

To illustrate the importance of parallel test execution and the
problem of \pef{} let's consider the case of the ``core'' module from
the Apache Camel project~\cite{apache-camel-web}.  This module
contains 5,679 test cases, declared in 2,356 test classes.  We ran
those tests in a machine with 16GB of memory and 8 virtual CPUs (4
cores with 2 native threads each).  Sequential test execution takes
24m50s to run this test set.  Execution of the same test set takes
2m28s when we configured parallel execution to fork a JVM per CPU and
execute test classes, uniquely allocated to that JVM, sequentially but
running test methods from each class in separate threads.  Note that
this is an order of magnitude speedup (10.07x)\Fix{Need to understand
  why this is 10x as opposed to something closer to 7x - I didn't get
  your concern here}.
Unfortunately, due to \pef{}, $\sim$2\% (114 of 5,679) of the tests
fail when executed in parallel.  It is important to notice that the
ratio of failures varies with the project as it depends on factors
such as length of test cases and amount of shared state across tests.

\pef{} is an important obstacle to enable parallel test execution.
Conceptually, higher parallelization can result in higher chances of
concurrency-related problems.  It is important to execute tests
efficiently without sacrificing reliability
\Fix{gap between pars?? What gap?}

This paper reports an empirical study to evaluate the impact of
parallelization options on test execution speedup and \pef{}. Based on
these results, we proposed different low-cost synchronization
strategies to avoid \pef{}.

%%  LocalWords:  parallelization multicore JUnit TestNG NUnit XXm YYm
%%  LocalWords:  Groupon parallelizing multi
