\section{Introduction}

\sloppy Testing is important...Test execution is
expensive...\Fix{examples}...Several approaches have been proposed to
mitigate this problem... Research has focused mostly on test
selection~\Fix{cite}, test reduction~\Fix{cite}, and test
prioritization~\Fix{cite}.

This work focuses on an orthogonal dimension to optimize regression
testing~--~parallelization, which intuitively enables regression
testing to scale with computing capacity.  Given the proliferation of
multicore machines it is not surprising that popular testing
frameworks and build systems provide today support for parallel test
execution~\Fix{cite}.  Unfortunately, these solutions can't be used
out-of-the-box: unrestricted parallel execution can produce
non-deterministic results as developers typically do not provision
protection to concurrent accesses originated from arbitrary program
points (\ie{}, tests).  We refer to this problem as Parallel Execution
Flakiness (\pef{}).

To illustrate the importance of parallel test execution and the
problem of \pef{} let's consider the case of the ``core'' module from
the Apache Camel project~\cite{apache-camel-web}.  This module
contains 5,679 test cases, declared in \Fix{NN} test classes.  We ran
those tests in a machine with 16GB of memory and 8 virtual CPUs (4
cores with 2 native threads each).  Sequential test execution takes
24m50s to run this test set.  Execution of the same test set takes
2m28s when we consider a parallel execution configuration that spawns
a JVM on each CPU and forks a thread per test class allocated to a
given JVM.  Note that this is an order of magnitude (10.07x)
speedup\Fix{Should we expect something close to 7x as opposed to
  10x?}.  Unfortunately, due to \pef{}, $\sim$2\% (114 of the 5,679)
of the tests fail when executed in parallel.  It is important to
notice that the ratio of failures varies with the project as it
depends on factors such as length of test cases and amount of shared
state across tests, for example.

This paper reports an empirical study to evaluate the impact of
parallelization options on test execution speedup and \pef{}. Based on
these results, we proposed different low-cost strategies to avoid
\pef{}.

%% To illustrate the \pef{} problem and discuss potential solutions, we
%% conducted a feasibility study involving \Fix{M} subject programs
%% containing test suites with at least \Fix{X,000} tests and observed
%% the following:

%% \Fix{move this!}Furthermore, we observed that parallel test execution can
%% significantly speedup regression testing.  The speedup obtained with
%% parallel test execution of the test suites we analyzed ranges from
%% \Fix{A} to \Fix{B}\%.  

%% %% Furthermore,
%% %% provisioning for such flexibility would create a significant burden on
%% %% developers and certainly produce hard-to-maintain code. 

%% \begin{itemize}
%% \item \textbf{\pef{} is prevalent.} For a given subject program, we ran
%%   the test suite in parallel multiple times.  \pef{} was observed in N of
%%   the M cases we considered.
%% \item \textbf{\pef{} affects a small but significant amount of tests.}
%%   Of the N cases where we observed \pef{}, the outcome of X\% of the
%%   tests, on average, was non-deterministic.
%% \end{itemize}

\Fix{-- I am here...}

\Fix{Move the following pars. somewhere else...}

These observations indicate that \pef{} is an important obstacle to
enable parallel test execution: it is important to enable reliable
execution of all tests in addition to executing tests efficiently.  It
is also important to shed some light on the practical value of
speeding up test execution.  In large IT companies (e.g.,
Groupon~\Fix{cite}, Google~\Fix{cite}, and Microsoft~\Fix{cite})
testing costs can be much higher compared to the costs we observed in
open-source projects, where test execution time ranged from \Fix{XXm}
to \Fix{YYm}.  Assigning worker machines from a server farm to execute
distinct subsets of test suites is not
uncommon\Fix{data!!}\footnote{Ideally, each part of the test suite
  would have cost proportional to the capacity of the assigned
  machine.}.  We conjecture that lighter weight solutions that use
commodity hardware to optimize test execution are also important.  For
the scenario above, these solutions could leverage the computing power
on each individual node of a server farm in addition to the aggregate
processing power of the farm.  Lighter weight solutions fits
particularly well relatively smaller lower-budget projects with
relatively high testing costs such as those that we observed in this
preliminary study.

It is important to note that previous work investigated test flakiness
manifested in sequential executions [cite cite].  For example, a test
case can fail because of a broken test dependency caused when ... or
because of an implicit ordering assumption was broken in a
multi-threaded test (under a sequential execution of the test suite).
Treating this source of flakiness is important but complementary to
this work: parallel execution flakiness can arise even without
sequential execution flakiness.  Discovering and fixing this source of
flakiness is challenging and essential to enable test parallelization.

%%  LocalWords:  parallelization multicore JUnit TestNG NUnit XXm YYm
%%  LocalWords:  Groupon parallelizing multi
