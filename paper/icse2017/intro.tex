\section{Introduction}

\sloppy Testing is important...Test execution is
expensive...\Fix{examples}...Several approaches have been proposed to
mitigate this problem...In the literature, research has focused mostly
in test selection\Fix{cite}, test reduction \Fix{cite}, and test
prioritization \Fix{cite}.  This work focuses on an orthogonal
dimension to optimize regression testing~--~parallelization, which
intuitively enables regression testing to scale with computing
capacity.

Given the proliferation of multicore machines it is not surprising
that popular testing frameworks provide today support for parallel
test execution (e.g., JUnit since version 4.?  \Fix{cite}, TestNG
since version \Fix{cite}, and \Fix{??NUnit??} since version \Fix{??}
\Fix{cite}).  Unfortunately, unrestricted parallel execution of tests
can produce non-deterministic results as developers typically do not
provision protection from concurrent accesses originated from
arbitrary program points (\ie{}, tests)\footnote{Testing frameworks
  document this limitation on the parallel execution
  feature~\Fix{cite}}.  We refer to this problem as Parallel Execution
Flakiness (\pef{}).  To illustrate the \pef{} problem and discuss
potential solutions, we conducted a feasibility study involving
\Fix{M} subject programs containing test suites with at least
\Fix{X,000} tests and observed the following:

\Fix{move this!}Furthermore, we observed that parallel test execution can
significantly speedup regression testing.  The speedup obtained with
parallel test execution of the test suites we analyzed ranges from
\Fix{A} to \Fix{B}\%.  

%% Furthermore,
%% provisioning for such flexibility would create a significant burden on
%% developers and certainly produce hard-to-maintain code. 

\begin{itemize}
\item \textbf{\pef{} is prevalent.} For a given subject program, we ran
  the test suite in parallel multiple times.  \pef{} was observed in N of
  the M cases we considered.
\item \textbf{\pef{} affects a small but significant amount of tests.}
  Of the N cases where we observed \pef{}, the outcome of X\% of the
  tests, on average, was non-deterministic.
\end{itemize}

These observations indicate that \pef{} is an important obstacle to
enable parallel test execution: it is important to enable reliable
execution of all tests in addition to executing tests efficiently.  It
is also important to shed some light on the practical value of
speeding up test execution.  In large IT companies (e.g.,
Groupon~\Fix{cite}, Google~\Fix{cite}, and Microsoft~\Fix{cite})
testing costs can be much higher compared to the costs we observed in
open-source projects, where test execution time ranged from \Fix{XXm}
to \Fix{YYm}.  Assigning worker machines from a server farm to execute
distinct subsets of test suites is not
uncommon\Fix{data!!}\footnote{Ideally, each part of the test suite
  would have cost proportional to the capacity of the assigned
  machine.}.  We conjecture that lighter weight solutions that use
commodity hardware to optimize test execution are also important.  For
the scenario above, these solutions could leverage the computing power
on each individual node of a server farm in addition to the aggregate
processing power of the farm.  Lighter weight solutions fits
particularly well relatively smaller lower-budget projects with
relatively high testing costs such as those that we observed in this
preliminary study.

It is important to note that previous work investigated test flakiness
manifested in sequential executions [cite cite].  For example, a test
case can fail because of a broken test dependency caused when ... or
because of an implicit ordering assumption was broken in a
multi-threaded test (under a sequential execution of the test suite).
Treating this source of flakiness is important but complementary to
this work: parallel execution flakiness can arise even without
sequential execution flakiness.  Discovering and fixing this source of
flakiness is challenging and essential to enable test parallelization.

%%  LocalWords:  parallelization multicore JUnit TestNG NUnit XXm YYm
%%  LocalWords:  Groupon parallelizing multi
