\section{Introduction}

Optimizing regression testing is important.  As software evolves the
size of test sets increases, potentially leading to disruption of the
development process caused by late reports of test failures.  This
problem manifests more dramatically at large IT companies
(e.g.,\Comment{ Groupon~\Fix{cite},} Google~\Fix{cite} and
Microsoft~\Fix{cite}) where the size of test sets often exceeds the
size of the code base.  Several approaches have been proposed to
optimize regression testing.  Research has focused mostly on test
selection~\Fix{cite}, test reduction~\Fix{cite}, and test
prioritization~\Fix{cite}.  Although these approaches can also be
applied in industrial settings, the focus in industry has been mostly
in parallelizing execution of tests using server farms, where worker
machines from the farm are assigned to execute distinct subsets of the
original test suite.

Given the proliferation of multicore machines it is not surprising
that popular testing frameworks and build systems provide today
support for parallel test execution with the goal of running tests
more efficiently~\cite{junit-org,testng,nunit,maven-surefire-plugin}.
These solutions use commodity hardware to maximize CPU usage in
individual machines.  In the case of the Java language, it is possible
to explore parallelism across JVMs and within the JVM.  The
lower-level parallelism enabled through build systems is an important
complement to the higher-level parallelism enabled through server
farms.  For the scenario of large IT organizations, lower-level
parallelization schemes leverage the computing power on each
individual node of a server farm in addition to the aggregate
processing power of the farm.  Lower-level parallelism fits
particularly well smaller organizations (/projects) with relatively
high testing costs but lower budgets.

Unfortunately, these solutions can't be used out-of-the-box:
unrestricted parallel execution of tests can produce non-deterministic
results as developers typically do not provision protection to
concurrent accesses originated from arbitrary program points (\ie{},
tests).  We refer to this problem as Parallel Execution Flakiness
(\pef{}).  To illustrate the importance of parallel test execution and
the problem of \pef{} let's consider the case of the ``core'' module
from the Apache Camel project~\cite{apache-camel-web}.  This module
contains 5,679 test cases, declared in \Fix{NN} test classes.  We ran
those tests in a machine with 16GB of memory and 8 virtual CPUs (4
cores with 2 native threads each).  Sequential test execution takes
24m50s to run this test set.  Execution of the same test set takes
2m28s when we configured parallel execution to spawn a JVM on each CPU
and fork a thread per test class, uniquely allocated to one JVM.  Note
that this is an order of magnitude (10.07x) speedup\Fix{Need to
  understand why this is 10x as opposed to something closer to 7x}.
Unfortunately, due to \pef{}, $\sim$2\% (114 of 5,679) of the tests
fail when executed in parallel.  It is important to notice that the
ratio of failures varies with the project as it depends on factors
such as length of test cases and amount of shared state across tests,
for example.  \pef{} is an important obstacle to enable parallel test
execution: it is important to execute tests efficiently without
sacrificing reliability.  Conceptually, higher parallelization can
result in higher chances of concurrency-related problems.  

This paper reports an empirical study to evaluate the impact of
parallelization options on test execution speedup and \pef{}. Based on
these results, we proposed different low-cost synchronization
strategies to avoid \pef{}.

%%  LocalWords:  parallelization multicore JUnit TestNG NUnit XXm YYm
%%  LocalWords:  Groupon parallelizing multi
