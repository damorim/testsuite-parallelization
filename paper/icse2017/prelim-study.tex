\section{Preliminary Study}

\begin{table}
  \centering
  \begin{tabular}{|c|c|c|c|}
    \hline
    \multirow{2}{*}{\emph{Subject}} & \multirow{2}{*}{\emph{\# Tests}} &  \multicolumn{2}{c|}{\emph{Time}}\\
    \cline{3-4}
    & & Seq & Par \\
    \hline
    jgit & - & - & - \\
    \hline
  \end{tabular}
  \caption{\label{table:cost}...}
\end{table}

\begin{table}
  \centering
  \begin{tabular}{|c|c|c|}
    \hline
    \emph{Subject} & Fail-Par. & Fail-Ind. \\
    \hline
    jgit & - & - \\
    \hline
  \end{tabular}
  \caption{\label{table:failures}...}
\end{table}

Iterative Test Execution (ITE) provides a means to reveal flakiness in
test runs caused by non-determinism.  In ITE the developer determines
a bound N for the number of re-executions.  Only tests that fail in
one given iteration (consequently, failed in all previous iterations)
are scheduled for re-execution in the next iteration.  At the end of
the process, a test is considered passing if it passed in the first
iteration, a test is considered failing if it fails in all iterations,
and it is considered flaky if it fails in all iterations but the last
it executed.  The process terminates when all tests eventually pass or
when the the bound N is reached.\Fix{illustrate with small
  figure...}~To note that ITE is supported by the popular maven
surefire plugin [cite].

