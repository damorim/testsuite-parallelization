\documentclass[conference]{IEEEtran}

\pdfpagewidth=8.5in
\pdfpageheight=11in

%% our macros and included packages
\newcommand{\Fix}[1]{\textbf{[[}{\color{red} #1}\textbf{]]}}
\newcommand{\pef}{PEF}
%% please, do not italicize ``i.e.'' and ``e.g.'' - Marcelo
\newcommand{\ie}{i.e.}
\newcommand{\eg}{e.g.}
\newcommand{\etal}{et al.}
\newcommand{\Comment}[1]{}
\newcommand{\CodeIn}[1]{{\small\texttt{#1}}}
\newcommand{\Jbc}[1]{\textbf{[[Jean: }{\color{orange} #1}\textbf{]]}}
\newcommand{\Luis}[1]{\textbf{[[Luis: }{\color{darkblue} #1}\textbf{]]}}
\newcommand{\Mar}[1]{\textbf{[[Marcelo: }{\color{darkgreen} #1}\textbf{]]}}

%% use this macros
\newcommand{\github}{Github} %% change consistently
\newcommand{\pomf}{\emph{pom.xml}} %% change consistently

%% Evaluation macros

%% SUBJECTS
\newcommand{\SubjectsGithub}{685}
\newcommand{\SubjectsGithubNotMaven}{48}
\newcommand{\SubjectsGithubNotTestable}{\Fix{132}}
\newcommand{\SubjectsGithubFlaky}{\Fix{??}}
\newcommand{\numSubjs}{\Fix{390}}
\newcommand{\numSubjsShort}{\Fix{307}}
\newcommand{\numSubjsMed}{\Fix{49}}
\newcommand{\numSubjsLong}{\Fix{34}}

%% CONFIGURATION MODES
\newcommand{\Seq}{C0}
\newcommand{\SeqClassParMeth}{C1}
\newcommand{\ParClassSeqMeth}{C2}
\newcommand{\ParClassParMeth}{C3}
\newcommand{\Fork}{F}
\newcommand{\ForkSeq}{\Fork{}\Seq{}}
\newcommand{\ForkParMeth}{\Fork{}\SeqClassParMeth{}}

%% RQ1
\newcommand{\medg}{medium}
\newcommand{\longg}{long}
\newcommand{\shortg}{short}

\newcommand{\numSubjsPass}{\Fix{250}}
\newcommand{\numSubjsFail}{\Fix{140}}

\newcommand{\numMed}{\Fix{49}}
\newcommand{\numLong}{\Fix{34}}
\newcommand{\numMedLong}{83}
\newcommand{\percentMedLongRunning}{\Fix{21\%}}
\newcommand{\percentLongRunning}{\Fix{10\%}}

%% RQ3&RQ4
\newcommand{\avgSpeedup}{2.71}
\newcommand{\pomMedLong}{67}
\newcommand{\numPomMed}{38}
\newcommand{\numPomLong}{29}
\newcommand{\numPomMatched}{\pomMedLong{}}
\newcommand{\percentParallel}{13.3\%}
\newcommand{\numNonParallel}{74}
%% \newcommand{\numPomMatchedValid}{\Fix{103}}
\newcommand{\numProjectsPar}{nine}
\newcommand{\percentShortSequential}{\Fix{96.7\%}}

\newcommand{\averageMedLongRunning}{8m36s}
\newcommand{\percentParallelForLongRunning}{\Fix{80\%}}

%% RQ5
\newcommand{\emailsSent}{297}
\newcommand{\emailsFalseAnswers}{three}
\newcommand{\emailsTrueAnswers}{32}
\newcommand{\emailsProjectsAnswered}{36}
\newcommand{\percEmailsProjectsAnswered}{61.29\%}
\newcommand{\emailsAnswered}{38}
\newcommand{\percEmailsAnswered}{12.8\%}
\newcommand{\discartedProjects}{10}
\newcommand{\emailsProjects}{62}

%% RQ5 Survey answers
\newcommand{\emailsCI}{12} % 8 + distributed
\newcommand{\emailsLocal}{6}
\newcommand{\emailsDistributed}{4}
\newcommand{\emailsNotDescribed}{14}

\newcommand{\emailsSequential}{30}
\newcommand{\emailsParallel}{6}

\newcommand{\emailsA}{8.33\%} %3
\newcommand{\emailsB}{33.3\%} %12
\newcommand{\emailsC}{22.22\%} %8
\newcommand{\emailsD}{16.66\%} %6
\newcommand{\emailsNA}{19.44\%} %7


\usepackage[T1]{fontenc}
\usepackage{microtype}
\usepackage{graphicx}
\usepackage{listings}
\usepackage{wrapfig}
\usepackage{enumitem}
\usepackage{balance}
\usepackage{float}
\usepackage{url}
\usepackage{tabularx}
\usepackage[table]{xcolor}
\usepackage{booktabs}
\usepackage{amsmath} 
\usepackage{subfigure} 
\usepackage{multirow}
\usepackage{mdframed}

% *** CITATION PACKAGES ***
\ifCLASSOPTIONcompsoc
  % IEEE Computer Society needs nocompress option % requires cite.sty v4.0 or later (November 2003) \usepackage[nocompress]{cite}
\else
  \usepackage{cite}
\fi

% correct bad hyphenation here
\hyphenation{op-tical net-works semi-conduc-tor or-gan-ize}

\def\denseitems{
  \itemsep1pt plus1pt minus1pt
  \parsep0pt plus0pt
  \parskip0pt\topsep0pt}

\clubpenalty = 10000
\widowpenalty = 10000
\displaywidowpenalty = 10000

\pagenumbering{gobble}
\begin{document}

\title{Test Suite Parallelization in Open-Source Projects:\\ A Study on Its Usage and Impact}

\author{\IEEEauthorblockN{Jeanderson Candido\qquad{}Luis Melo\qquad{}Marcelo d'Amorim}
\IEEEauthorblockA{Federal University of Pernambuco\\
Pernambuco, Brazil\\
\{jbc5,lhsm,damorim\}@cin.ufpe.br}}

\maketitle
\thispagestyle{plain}
\pagestyle{plain}

\begin{abstract}
Dealing with high testing costs remains an important problem in
Software Engineering.  Test suite parallelization is an important
approach to address this problem.  This paper reports our findings on
the usage and impact of test suite parallelization in open-source
projects.\Comment{ This study brings to light the benefits and burdens
  of that approach.} It provides recommendations to practitioners and
tool developers to speed up test execution.

Considering a set of \numSubjs{} popular Java projects we analyzed, we
found that \percentMedLongRunning{} of the projects contain costly
test suites but parallelization features still seem underutilized in
practice~---~only \percentParallelUpdated{} of costly projects use
parallelization.  The main reported reason for adoption resistance was
the concern to deal with concurrency issues.  Results suggest that, on
average, developers prefer high predictability than high performance
in running tests.
\end{abstract}

\IEEEpeerreviewmaketitle

\section{Introduction}

As software evolves it is expected that the number of tests and the
length of test runs increase.  Both elements can add up on the
aggregate cost of running a test suite, leading to late reports of test
failures, which can disrupt the development
process~\cite{hilton-etal-ase2016}.  This is often referred to as the
``regression testing problem''.  At large IT companies (e.g.,
Groupon~\cite{kim-etal-fse2013}), dealing with high testing costs is a
common problem .  \Mar{Ask Milos if he can share stories on how
  prevalent regression testing is at Microsoft and Google, if he can
  explain how they run tests, and how prevalent is parallelization (in
  contrast to regression testing).}

Several approaches have been proposed in the literature to optimize
regression testing.  Research has focused mostly on test suite
minimization, prioritization, reduction, and
selection~\cite{yoo-harman-stvr2012}.  Most of these techniques are
unsound (\ie{}, they do not guarantee that fault-revealing tests will
be selected), however, more recently, sound techniques have been
proposed~\cite{gligoric-etal-issta2015,soetens-etal-2016}.  For
example, to soundly select tests for execution,
Ekstazi\cite{ekstazi-web,gligoric-etal-issta2015} analyzes which files
that a given test depends on have been impacted by changes. File
dependencies are computed and maintained at low cost.

Although regression testing is a recognized problem in academia and at
large IT organizations, to the best of our knowledge no prior work has
investigated how prevalent the problem is in open-source projects.  We
understand that the alternative of renting server farms (or even
building one) to mitigate the regression testing problem is a
potential escape for large IT organizations but not as attractive for
open-source projects and projects from smaller organizations with
lower budgets for building testing infrastructures.  This paper
studies (1) how relevant the problem of long-running test suites in
open-source software is, (2) what are the causes of high execution
cost (CPU vs. IO), and (3) what is the impact of existing open-source
sound approaches to reducing cost.

\Fix{--------------}

Given the proliferation of multicore machines it is not surprising
that popular build systems and testing frameworks provide today
support for parallel test execution with the goal of running tests
more efficiently~\cite{junit-org,testng,nunit,maven-surefire-plugin}.
The lower-level parallelism enabled through build systems and testing
frameworks is an important complement to the higher-level parallelism
enabled through server farms.  These solutions enable the use of
commodity hardware to maximize CPU usage\footnote{In the case of the
Java language, for example, it is possible to explore parallelism
across and within JVMs.}.  For the scenario of large IT organizations,
lower-level parallelization schemes could leverage the computing power
of server nodes in addition to the aggregate processing power of the
farm.  Lower-level parallelism fits particularly well smaller
organizations (/projects) with relatively high testing costs but lower
budgets.  \Comment{Unfortunately, these solutions can't be used out-of-the-box:
unrestricted parallel execution of tests can produce non-deterministic
results as developers typically do not provision protection to
concurrent accesses originated from arbitrary program points (\ie{},
tests).  We refer to this problem as Parallel Execution Flakiness
(\pef{}).}

To illustrate the importance of parallel test execution, \Jbc{TODO -
Write a paragraph motivating this paper.}

\Comment{
To illustrate the importance of parallel test execution and
the problem of \pef{} let's consider the case of the ``core'' module
from the Apache Camel project~\cite{apache-camel-web}.  This module
contains 5,679 test cases, declared in 2,356 test classes.  We ran
those tests in a machine with 16GB of memory and 8 virtual CPUs (4
cores with 2 native threads each).  Sequential test execution takes
24m50s to run this test set.  Execution of the same test set takes
2m28s when we configured parallel execution to fork a JVM per CPU and
execute test classes, uniquely allocated to that JVM, sequentially but
running test methods from each class in separate threads.  Note that
this is an order of magnitude speedup (10.07x)\Fix{Need to understand
why this is 10x as opposed to something closer to 7x - I didn't get
your concern here}.  Unfortunately, due to \pef{}, $\sim$2\% (114 of
5,679) of the tests fail when executed in parallel.  It is important
to notice that the ratio of failures varies with the project as it
depends on factors such as length of test cases and amount of shared
state across tests.

\pef{} is an important obstacle to enable parallel test execution.
Conceptually, higher parallelization can result in higher chances of
concurrency-related problems.  It is important to execute tests
efficiently without sacrificing reliability
\Fix{gap between pars?? What gap?}

}

This paper reports an empirical study to evaluate the impact of
parallelization options on test execution speedup \Fix{...}\Jbc{TODO -
Summary of the contributions of this paper} .

%%  LocalWords:  parallelization multicore JUnit TestNG NUnit XXm YYm
%%  LocalWords:  Groupon parallelizing multi JVMs CPUs JVM Milos TODO
%%  LocalWords:  Ekstazi

\newcommand{\Seq}{L0}
\newcommand{\SeqClassParMeth}{L1}
\newcommand{\ParClassSeqMeth}{L2}
\newcommand{\ParClassParMeth}{L3}
\newcommand{\Fork}{F}
\newcommand{\ForkSeq}{\Fork{}\Seq{}}
\newcommand{\ForkParMeth}{\Fork{}\SeqClassParMeth{}}

\section{Parallel Execution of Test Suites}
\label{sec:modes}

\begin{figure}[t!]
  \centering
  \includegraphics[width=0.35\textwidth]{figs/parallel-levels.pdf}
  \vspace{-1ex}
  \caption{\label{fig:levels}Levels of parallelism.}
\end{figure}

%\Mar{add a sentence to introduce this section at a higher abstraction
%  level.}

This section provides some background on parallel test execution.

Parallelism in test execution can be obtained at different levels.
Figure~\ref{fig:levels} illustrates relevant levels.  The highest
level indicates parallelism that can be obtained through different
machines on the network.  For instance, using virtual machines from a
cloud service to perform a distributed execution.  The machine space
sits under the ``network space'' and provides a \emph{lower-level}
parallelism.  In this case, computation can be offloaded at different
CPUs within a machine and at different threads within each CPU.  Note
that all these levels are complementary:  low-level parallelization
schemes could leverage the computing power of server nodes in addition
to the aggregate processing power of the farm.

%% Lower-level parallelism fits
%% particularly well smaller organizations (/projects) with relatively
%% high testing costs but lower budgets.

This paper focuses at low-level parallelism (\ie, machine and CPU
space), which can be enabled through build systems and testing
frameworks.  Given the proliferation of multi-core machines it is not
surprising that testing frameworks provide today support for parallel
test execution (e.g., JUnit~\cite{junit-org}, TestNG~\cite{testng},
and NUnit~\cite{nunit}).  In the following we elaborate relevant
features of and testing frameworks build systems.  We focused on Java,
Maven, and JUnit but the discussion can be generalized to other
language and tools.

\Comment{
    Unfortunately, these solutions can't be used out-of-the-box:
    unrestricted parallel execution of tests can produce
    non-deterministic results as developers typically do not provision
    protection to concurrent accesses originated from arbitrary program
    points (\ie{}, tests).  We refer to this problem as Parallel
    Execution Flakiness (\pef{}).

    %% These solutions enable the
    %% use of commodity hardware to maximize CPU usage\footnote{In the case
    %%   of the Java language, for example, it is possible to explore
    %%   parallelism across and within JVMs.}.  

    To illustrate the importance of parallel test execution and
    the problem of \pef{} let's consider the case of the ``core'' module
    from the Apache Camel project~\cite{apache-camel-web}.  This module
    contains 5,679 test cases, declared in 2,356 test classes.  We ran
    those tests in a machine with 16GB of memory and 8 virtual CPUs (4
    cores with 2 native threads each).  Sequential test execution takes
    24m50s to run this test set.  Execution of the same test set takes
    2m28s when we configured parallel execution to fork a JVM per CPU and
    execute test classes, uniquely allocated to that JVM, sequentially but
    running test methods from each class in separate threads.  Note that
    this is an order of magnitude speedup (10.07x)\Fix{Need to understand
    why this is 10x as opposed to something closer to 7x - I didn't get
    your concern here}.  Unfortunately, due to \pef{}, $\sim$2\% (114 of
    5,679) of the tests fail when executed in parallel.  It is important
    to notice that the ratio of failures varies with the project as it
    depends on factors such as length of test cases and amount of shared
    state across tests.

    \pef{} is an important obstacle to enable parallel test execution.
    Conceptually, higher parallelization can result in higher chances of
    concurrency-related problems.  It is important to execute tests
    efficiently without sacrificing reliability
    \Fix{gap between pars?? What gap?}
}

%%Build systems delegate functionalities to the testing framework they
%%use~\cite{maven-surefire-plugin}.  Figure~\ref{fig:surefire}
%%illustrates how to configure Maven for mode
%%\ParClassSeqMeth{}~--~classes sequentially (default) and test methods
%%in parallel. Although this configuration is defined in the build file
%%(\eg, \emph{pom.xml} on Maven), these settings are forwarded to the
%%underlying testing framework (in this case, JUnit).

%% \Jbc{I have to understand better how Surefire executes tests. I'm
%% afraid it has its own wrapper classes and reuse JUnit core
%% functionalities. I'm saying this because JUnit does not offer a finer
%% control over thread numbers while this configuration is possible with
%% Surefire} For example,

%\Fix{we
%  need to know how Maven allocates test classes to JVMs}

\subsection{Testing Frameworks}
\label{sec:frameworks}

The list below shortly describes the choices to control parallelism
within a Java virtual machine (JVM).

\begin{itemize}
\item
    \textbf{Sequential (\Seq).}~This configuration corresponds to the
        default behaviour in a test execution. No parallelism is
        involved.
\item
    \textbf{Sequential classes; parallel methods
        (\SeqClassParMeth).}~When running a test class, the testing
        framework executes methods on multiple threads. Each class
        runs sequentially.
\item
    \textbf{Parallel classes; sequential methods
        (\ParClassSeqMeth{}).}~The testing framework runs test classes
        on multiple threads and methods run sequentially within each
        thread.
\item
    \textbf{Parallel classes; Parallel methods
        (\ParClassParMeth).}~This configuration corresponds
        conceptually to the union of \ParClassSeqMeth{} and
        \SeqClassParMeth{}: the testing framework runs test methods
        from several classes on multiple threads.
\end{itemize}

Typically, testing frameworks executes tests in sequence by default
(\Seq{}) and may allow the developer to use an ordering criteria (\eg,
test name).  For the \SeqClassParMeth{} setting, \Fix{elaborate...}.
For the \ParClassSeqMeth{} setting, \Fix{elaborate...}. Finally, the
\ParClassParMeth{} setting \Fix{...elaborate...}.

\subsection{Build Systems}
\label{sec:builder}

The list below shortly describes the choices to control parallelism
through build systems:

\begin{itemize}
\item
    \textbf{Forked classes; sequential methods (\ForkSeq).}~This
        configuration corresponds to the parallel execution of test
        classes on different JVMs and test methods run sequentially
        within each JVM.
\item
    \textbf{Forked classes; parallel methods (\ForkParMeth).}~This
        configuration corresponds to the parallel execution of test
        classes on different JVMs and test methods run on multiple
        threads within each JVM.
\end{itemize}

Several build systems support the execution of test classes in forked
processes running a JVM. In the \ForkSeq{} configuration, each spawned
process runs a different test class at time and the builder merges the
results from each execution. In the \ForkParMeth{} configuration, the
build system forwards settings to the underlying testing framework to
enable parallel execution within each process in addition to running
test classes on different processes.  However, some configurations
(\eg, \ParClassSeqMeth) may not take effect since only one test class
can run in a forked process at time.  To the best of our knowledge, we
are unaware of any build system capable of running multiple classes at
time within a forked process.  \Comment{In addition, the builder can
reuse these processes to execute the next test classes in the queue
according to the developer's preference. For instance, one might want
the builder to spawn a new process for each test class to ensure
maximum isolation. Notice that to exploit parallelism at build system
level, the number of forked processes must be proportional to the
number of available CPUs, otherwise, the high frequency of
context-switching and processes competing for a CPU may degrade the
execution performance  \Jbc{$\leftarrow$ I'm not sure if we should
keep this. We can remove it if necessary}.}

%% In this paper, we use the prefix ``\Fork{}'' to indicate that a given
%% testing configuration has the feature of spawning JVMs to run test
%% classes enabled.

\subsubsection*{Illustration}~\Fix{elaborate the following example to
conclude this section}

\begin{figure}[h!]
\centering
\scriptsize
\lstset{
    escapeinside={@}{@},
    numbers=left,xleftmargin=1em,frame=single,framexleftmargin=0.5em,
    basicstyle=\ttfamily\scriptsize, boxpos=c, numberstyle=\tiny,
    morekeywords={parallel, threadCount, perCoreThreadCount,
    forkCount, reuseFork},
    deletekeywords={true}
}
\begin{lstlisting}
<plugin>
    <groupId>org.apache.maven.plugins</groupId>
    <artifactId>maven-surefire-plugin</artifactId>
    <configuration>
        <forkCount>1C</forkCount>
        <parallel>methods</parallel>
        <threadCount>5</threadCount>
    </configuration>
</plugin>
\end{lstlisting}
    \caption{\label{fig:surefire} Maven with a \ForkParMeth{}
    configuration. Maven forks one JVM per core (\CodeIn{forkCount}
    parameter) and \Jbc{check this $\rightarrow$} uses five threads
    (\CodeIn{threadCount} parameter) to run methods (\CodeIn{parallel}
    parameter) within each JVM.}
    
%%    test methods run in parallel (\CodeIn{parallel}
%%    parameter) using two threads (\CodeIn{threadCount} parameter) per
%%    available core (flag \CodeIn{perCoreThreadCount}).}
\end{figure}

%% \Fix{-----}
%% In JUnit, \emph{ParallelComputer} provides support to parallel
%% execution: it instantiates a test runner with an
%% \emph{ExecutorService} from the Java Concurrent API. Each test method
%% is executed in a separated thread as the thread pool creates new
%% threads on-demand and reuse them if any thread is available.
%% \Comment{http://junit.org/junit4/javadoc/latest/src-html/org/junit/experimental/ParallelComputer.html#line.14}

%% As in \ParClassSeqMeth, the test runner has a cached thread poll but
%% instead of executing test methods from a single test class, the test
%% suite runner executes classes in separated threads and methods
%% sequentially within each thread.

%% \Jbc{I need to confirm this but AFAIK this is how it happens} As in
%% \SeqClassParMeth, classes execute on separated threads but test
%% methods are also executed in parallel. More precisely, there is a
%% thread pool to execute classes in separated threads (\ie, like in
%% \ParClassSeqMeth) and, within each class runner, there is an
%% \emph{ExecutorService} to run test methods in parallel (\ie, like in
%% \SeqClassParMeth).

%% \Fix{-----}


\section{Objects of Analysis}
\label{sec:subjects}

We used \github{}'s search API~\cite{githubsearch} to identify
projects that satisfy the following criteria: (1) the primary language
is Java\footnote{In case of projects in multiple languages, the
  \github{} API considers the predominant language as the primary
  language.}, (2) the project has at least 100 stars, (3) the latest
update was on or after January 1st, 2016, and (4) the \emph{readme}
file contains the string \emph{mvn}.  We focused on Java for its
popularity.  Although there is no clearcut limit on the number of
\github{} stars~\cite{github-stars} to define relevant projects, we
observed that one hundred stars was enough to eliminate trivial subjects. The
third
criteria serves to skip projects without recent activity. The fourth
criteria is an approximation to find Maven projects.\Comment{ The
  rationale is that if the string \emph{mvn} exists in the
  \emph{readme} file, it may represent a Maven call (\eg, to compile
or to test the project).} We focused on Maven for its popularity on
Java projects.  Important to highlight that, as of now, the
\github{}'s search API can only reflect contents from repository
statistics (\eg, number of forks, main programming language); it does
not provide a feature to search for projects containing certain files
(\eg{}, \emph{pom.xml}) in the directory structure.
Figure~\ref{fig:subject-query} illustrates the query to the \github{}
API as an HTTP request.   The result set is sorted
in descending order of stars.

\begin{figure}[t!]
\centering
\tiny
\lstset{
    escapeinside={@}{@},
    numbers=left,xleftmargin=1em,frame=single,framexleftmargin=0.5em,
    basicstyle=\ttfamily\scriptsize, boxpos=c, numberstyle=\tiny,
    showstringspaces=false
}
\begin{lstlisting}
https://api.github.com/search/repositories?q=language:java
 +stars:>=100+pushed:>=2016+mvn%20in:readme&sort=stars
\end{lstlisting}
  \vspace{-3mm}
  \caption{\label{fig:subject-query} Query to the \github{} API for
  projects that (1) use Java, (2) contains at least 100
  stars, (3) has been updated on January 1st, 2016 (or later), (4) contains
  the string \emph{mvn} in the \emph{readme} file.}
  \vspace{-3mm}
\end{figure}

We used the following methodology to select projects for
analysis. After obtaining the list of potential projects from GitHub, we
filtered those
containing a \pomf{} file in the root directory.\Comment{  A Maven project may
contain several sub-modules with multiple \pomf{} files.}
Then, considering this set of Maven projects, we
executed the tests for \SubjectsReruns{} times to discard those projects with
issues
in the build file and non-deterministic results observed from sequential
executions.
As of August 25th 2017, our search criteria returned a total of
\SubjectsGithub{}
subjects.
From this set of projects,
\SubjectsGithubNotMaven{} projects were not Maven or did not have a
\pomf{} in the root directory, 
\SubjectsGithubNotTestable{} projects were not considered because of
environment incompatibility
(\eg, missing\Comment{ required web browser or database management
system}~DBMS),
\SubjectsGithubFlaky{} projects were discarded because of
``flaky tests''~\cite{luo-etal-fse2014}. A ``flaky'' test is a test that passes
or fails under
the same circumstances leading to non-deterministic results.
As some of our experiments consist of running tests on different
threads, we ignored these projects as it would be impractical
to identify whether a test failed due to a race condition or some
other source of flakiness.
From the remaining \SubjectsGithubConsistant{} projects with
deterministic results, we eliminated \SubjectsGithubTooManyFailures{}
projects with \SuiteFailingThreshold{} or more failing tests as to
reduce bias. For the
remaining projects with failing tests, we used the JUnit's
\CodeIn{@Ignore} annotation to ignore failing tests.
Our final set of subjects contains \numSubjs{} projects.
Figure~\ref{fig:subjects} summarizes our sample set.

\begin{figure}[ht]
  \vspace{-5mm}
  \centering
  \includegraphics[width=0.27\textwidth]{results/piechart-subjs.pdf}
  \caption{\label{fig:subjects}We fetched \SubjectsGithub{} popular projects
  hosted on \github{}. From this initial sample, we ignored
  \SubjectsGithubNotMaven{} projects without Maven support,
  \SubjectsGithubNotTestable{} with missing dependencies,
  \SubjectsGithubFlaky{} projects with flaky tests, and
  \SubjectsGithubTooManyFailures{} projects had at least
  \SuiteFailingThreshold{} of failing tests. We considered
  \numSubjs{} projects to conduct our study.}
\end{figure}

\label{sec:setup}
To run our experiments, we used a Core i7-4790 (3.60 GHz) Intel processor
machine with eight virtual CPUs (four cores with two native threads each) and
16GB of memory, running Ubuntu 14.04 LTS Trusty Tahr (64-bit version).  We
used\Comment{ Git,} Java 8 and Maven 3.3.9 to build projects and run test
suites. To process test results and generate plots we used Python\Comment{ 3.4},
Bash, R and Ruby\Comment{ 2.3}.  All source artifacts are publicly available for
replication on our website~\cite{ourwebpage}.  This includes supporting
scripts\Comment{ (\eg, the scripts to run the tests and generate raw analysis
data)} and the full list of projects.


\section{Evaluation}
\label{sec:eval}

We are interested in understanding the prevalence of time-consuming
test suites and main sources of cost. We want to understand how the
execution cost is distributed on test cases within a test suite and
how developers approach test execution. Based on that, we study
parallelization of testing frameworks.  More precisely, we investigate
how prevalent test parallelization is, the potential for improving
execution cost, issues of flakiness that hinders the use of
parallelization, and how to address those issues.  More specifically,
we pose the following research questions:

\newcommand{\numRQA}{RQ1}
\newcommand{\numRQB}{RQ2}
\newcommand{\numRQC}{RQ3}
\newcommand{\numRQD}{RQ4}
%%\newcommand{\numRQE}{RQ5}
\newcommand{\numRQF}{RQ5}

\newcommand{\RQA}{How prevalent is the occurence of time-consuming
  regression test suites in open-source projects?}
\newcommand{\RQB}{How the time cost is distributed across test cases?}
\newcommand{\RQC}{How often parallelization settings appear in build
  files?}
\newcommand{\RQD}{What is the impact of parallelization?}
%%\newcommand{\RQE}{What factors contribute to improve performance
%%  through parallelization?\Jbc{Previously as "When does
%%    parallelization works and when it does not so great?"}}
\newcommand{\RQF}{What are the limitations of low-level parallelism?}

\begin{itemize}
    \item \textbf{\numRQA.} \RQA
    \item \textbf{\numRQB.} \RQB
    \item \textbf{\numRQC.} \RQC
    \item \textbf{\numRQD.} \RQD
%%    \item \textbf{\numRQE.} \RQE
    \item \textbf{\numRQF.} \RQF
\end{itemize}

%%\newcommand{\RQB}{What is the distribution of CPU and IO bound
%%regression test suites from the sample set?}
%%
%%\newcommand{\RQC}{How uniformly distributed is the execution time
%%across test cases in costly projects?}
%%
%%\newcommand{\RQD}{How often developers use the parallelism features
%%from build systems to improve runtime performance?}

The first research question addresses the prevalence of long-running
test suites. We are interested to know if costly test suites are
common in open-source projects.  The second research question
addresses the relationship of test cases and the overall execution
cost: we are interested to investigate how the time cost is
distributed among test cases.  In the third research question, we are
in understanding if developers are aware of low-level parallelism
features available out-of-the-box through build systems (\eg,
configuring Maven to use JUnit with multiple threads to speedup test
execution). In addition, we want to identify which configurations are
often used and why they more popular (if any).  The fourth research
question addresses the impact of low-level parallelism on the
regression tests from our sample set. For subjects that use
parallelization settings, we are interested to compare the execution
performance when these settings are activated and deactivated (\ie,
tests run sequentially).
%%\Jbc{This RQ has to be reworked:} \Fix{The fifth research question
%%addresses the characteristics of regression tests from the previous
%%experiment.  More specifically, we want to investigate if there is a
%%relation between the balance of test execution (\ie, how uniformly
%%distributed is the execution time across tests cases) and the usage of
%%computational resources (\ie, if tests are mostly CPU or IO intense)
%%that impacts the effectiveness of parallelization}.
Finally, the fifth research question discusses the limitations and
insights to overcome the pitfalls of parallelization.

\Comment{
    \Fix{distribution of execution time per test case. For each subject
    identified in the first research question, we investigate how
    balanced is the cost of the test suite in contrast to the cost of
    test cases and if there are subjects where the time cost is mostly
    dominated by a small fraction of test cases.} \Fix{The third research
    question addresses the distribution of regression tests according
    to the use of computational resources.  We are interested in
    investigating if regression test suites are CPU intensive and if there
    are opportunities to improve performance. The RQ4 addresses}
    \Fix{...elaborate...}

    The rationale is that if the time cost of a regression test is equally
    distributed among test cases, the execution cost could be potentially
    improved by running tests in parallel (in contrast to the scenario
    where only one test case dominates most of the execution time).
}

\subsection{Subjects}
\label{sec:subjects}

We evaluated the characteristics of regression tests in open-source
development from a sample set of Java projects from \github{}.  We are
interested to evaluate non-trivial test suites on popular projects
that are in activity. We used the \github{}'s search
API~\cite{githubsearch} to fetch Java projects according to the
following criteria: (1) the primary language must be Java\footnote{In
case of projects in multiple languages, the \github{} API considers
the predominant language as the primary language.}; (2) the project
has at least 100 stars; (3) latest update on (or after) January 1st,
2016; (4) the \emph{readme} file contains the string \emph{mvn}.

On \github{}, when a user adds a star to a project, s/he demonstrated
appreciation and bookmarked it for later
reference~\cite{github-stars}.  Although there is not a specific range
for the number of stars, our criteria is an estimation to avoid
trivial subjects: we assumed that \github{} users are likely to
demonstrate interest on well-tested projects. The third criteria is a
constraint to skip projects without recent activity. The fourth
criteria is an approximation to find projects with Maven support. The
rationale is that if the string \emph{mvn} exists in the \emph{readme}
file, it may represent a Maven call (\eg, to compile or test the
project). We used Maven as a reference due to its popularity on Java
projects and to automate our evaluation scripts.
Figure~\ref{fig:subject-query} illustrates the query as an HTTP
request.

\begin{figure}[h!]
\centering
\scriptsize
\lstset{
    escapeinside={@}{@},
    numbers=left,xleftmargin=1em,frame=single,framexleftmargin=0.5em,
    basicstyle=\ttfamily\scriptsize, boxpos=c, numberstyle=\tiny,
    deletekeywords={true}
}
\begin{lstlisting}
https://api.github.com/search/repositories?q=language:java
        +stars:>=100+pushed:>=2016-01-01
        +mvn%20in:readme+sort:stars
\end{lstlisting}
    \caption{\label{fig:subject-query} Query to the \github{} API for
    projects with the following criteria: (1) Java, (2) at least 100
    stars, (3) updated on January 1st, 2016 (or later), (4) contains
    the string \emph{mvn} in the \emph{readme} file. Output is
    paginated in descending order of stars.}
\end{figure}

A Maven project may contain several sub-modules with multiple \pomf{}
files. We considered only projects with a \pomf{} file located in the
root directory.  As of March 25th 2017, our search criteria returned
\SubjectsGithub{} subjects.  From the \SubjectsGithub{} downloaded
projects, \SubjectsGithubNotMaven{} were not Maven projects (or did
not have a \pomf{} in the root directory) and
\SubjectsGithubNotTestable{} were in an untestable revision (\eg,
missing dependencies or incompatible testing environment). To ensure
that our sample set is stable, we retested the remaining subjects to
eliminate flaky tests (total of \SubjectsGithubFlaky{} subjects). Our
final set consists in \numSubjs{} testable subjects.

\subsection{Setup and Replication}
\label{sec:setup}

To run our experiments, we used a Core i7-4790 (3.60 GHz) Intel
processor machine with eight virtual CPUs (four cores with two native
threads each) and 16GB of memory, running Ubuntu 14.04 LTS Trusty Tahr
(64-bit version). Software settings include \Comment{the Linux
\emph{sysstat} package to measure performance, }git to fetch subjects,
Java 8 and Maven 3.3.9 to build and test subjects. Our evaluation
scripts depends on Python 3.4 and Bash to execute and R to analyze the
data. For replication, all source artifacts are publicly available at
\Fix{create gh-pages}, including supporting scripts (\eg, the script
that test subjects and generates the raw data), and the full list of
projects. \Comment{, and a \emph{Vagrantfile} to emulate our hardware
and all software dependencies.}

\subsection{Answering research question \numRQA{}}
\label{sec:rqA}

\begin{itemize}
    \item \emph{\RQA}
\end{itemize}

To evaluate the frequency of time-consuming regression tests, we
considered the \numSubjs{} subjects from Section~\ref{sec:subjects}
and compared the elapsed time to run their tests.
Figure~\ref{fig:mvn-execution} shows the commands used in our script
to test each subject. The main loop (lines 5-11) iterates over the
list of subjects and invokes Maven in separated steps (lines 7-9). To
avoid inflating the measured time we executed Maven in different
steps: we first compiled the source and test files (line 7), made all
dependencies available locally (line 8) and later, we ran the tests in
offline mode (line 9) to bypass package updates. After the execution,
we used a regular expression on the output log to extract the elapsed
time (line 10). Before collecting the time cost, we executed all
subjects and randomly selected 100 logs to inspect and identify
non-related tasks (\eg, \emph{javadoc} generation and static analysis)
to ignore during the experiment (lines 1-3).  Also, measured the
approximated overhead from the build system after skipping non-related
tasks. We identified the generated test reports and compared the
difference between the elapsed time reported from Maven and the sum of
all test cases executed.
To avoid noise from operating system events, we used a dedicated
server remotely via SSH with the operating system running only
essential services (\eg, the SSH server). In addition, we configured
the \CodeIn{isolcpus} option from the Linux Kernel \cite{linux-kernel}
to isolate six virtual CPUs to execute our experiment and the
remaining CPUs to run the operating system. By isolating a set of
virtual CPUs, we prevent context-switching from experiment processes
and OS-related processes.

\section{Evaluation}
\label{sec:eval}

We are interested in understanding the prevalence of time-consuming
test suites and main sources of cost. We want to understand how the
execution cost is distributed on test cases within a test suite and
how developers approach test execution. Based on that, we study
parallelization of testing frameworks.  More precisely, we investigate
how prevalent test parallelization is, the potential for improving
execution cost, issues of flakiness that hinders the use of
parallelization, and how to address those issues.  More specifically,
we pose the following research questions:

\newcommand{\numRQA}{RQ1}
\newcommand{\numRQB}{RQ2}
\newcommand{\numRQC}{RQ3}
\newcommand{\numRQD}{RQ4}
%%\newcommand{\numRQE}{RQ5}
\newcommand{\numRQF}{RQ5}

\newcommand{\RQA}{How prevalent is the occurence of time-consuming
  regression test suites in open-source projects?}
\newcommand{\RQB}{How the time cost is distributed across test cases?}
\newcommand{\RQC}{How often parallelization settings appear in build
  files?}
\newcommand{\RQD}{What is the impact of parallelization?}
%%\newcommand{\RQE}{What factors contribute to improve performance
%%  through parallelization?\Jbc{Previously as "When does
%%    parallelization works and when it does not so great?"}}
\newcommand{\RQF}{What are the limitations of low-level parallelism?}

\begin{itemize}
    \item \textbf{\numRQA.} \RQA
    \item \textbf{\numRQB.} \RQB
    \item \textbf{\numRQC.} \RQC
    \item \textbf{\numRQD.} \RQD
%%    \item \textbf{\numRQE.} \RQE
    \item \textbf{\numRQF.} \RQF
\end{itemize}

%%\newcommand{\RQB}{What is the distribution of CPU and IO bound
%%regression test suites from the sample set?}
%%
%%\newcommand{\RQC}{How uniformly distributed is the execution time
%%across test cases in costly projects?}
%%
%%\newcommand{\RQD}{How often developers use the parallelism features
%%from build systems to improve runtime performance?}

The first research question addresses the prevalence of long-running
test suites. We are interested to know if costly test suites are
common in open-source projects.  The second research question
addresses the relationship of test cases and the overall execution
cost: we are interested to investigate how the time cost is
distributed among test cases.  In the third research question, we are
in understanding if developers are aware of low-level parallelism
features available out-of-the-box through build systems (\eg,
configuring Maven to use JUnit with multiple threads to speedup test
execution). In addition, we want to identify which configurations are
often used and why they more popular (if any).  The fourth research
question addresses the impact of low-level parallelism on the
regression tests from our sample set. For subjects that use
parallelization settings, we are interested to compare the execution
performance when these settings are activated and deactivated (\ie,
tests run sequentially).
%%\Jbc{This RQ has to be reworked:} \Fix{The fifth research question
%%addresses the characteristics of regression tests from the previous
%%experiment.  More specifically, we want to investigate if there is a
%%relation between the balance of test execution (\ie, how uniformly
%%distributed is the execution time across tests cases) and the usage of
%%computational resources (\ie, if tests are mostly CPU or IO intense)
%%that impacts the effectiveness of parallelization}.
Finally, the fifth research question discusses the limitations and
insights to overcome the pitfalls of parallelization.

\Comment{
    \Fix{distribution of execution time per test case. For each subject
    identified in the first research question, we investigate how
    balanced is the cost of the test suite in contrast to the cost of
    test cases and if there are subjects where the time cost is mostly
    dominated by a small fraction of test cases.} \Fix{The third research
    question addresses the distribution of regression tests according
    to the use of computational resources.  We are interested in
    investigating if regression test suites are CPU intensive and if there
    are opportunities to improve performance. The RQ4 addresses}
    \Fix{...elaborate...}

    The rationale is that if the time cost of a regression test is equally
    distributed among test cases, the execution cost could be potentially
    improved by running tests in parallel (in contrast to the scenario
    where only one test case dominates most of the execution time).
}

\subsection{Subjects}
\label{sec:subjects}

We evaluated the characteristics of regression tests in open-source
development from a sample set of Java projects from \github{}.  We are
interested to evaluate non-trivial test suites on popular projects
that are in activity. We used the \github{}'s search
API~\cite{githubsearch} to fetch Java projects according to the
following criteria: (1) the primary language must be Java\footnote{In
case of projects in multiple languages, the \github{} API considers
the predominant language as the primary language.}; (2) the project
has at least 100 stars; (3) latest update on (or after) January 1st,
2016; (4) the \emph{readme} file contains the string \emph{mvn}.

On \github{}, when a user adds a star to a project, s/he demonstrated
appreciation and bookmarked it for later
reference~\cite{github-stars}.  Although there is not a specific range
for the number of stars, our criteria is an estimation to avoid
trivial subjects: we assumed that \github{} users are likely to
demonstrate interest on well-tested projects. The third criteria is a
constraint to skip projects without recent activity. The fourth
criteria is an approximation to find projects with Maven support. The
rationale is that if the string \emph{mvn} exists in the \emph{readme}
file, it may represent a Maven call (\eg, to compile or test the
project). We used Maven as a reference due to its popularity on Java
projects and to automate our evaluation scripts.
Figure~\ref{fig:subject-query} illustrates the query as an HTTP
request.

\begin{figure}[h!]
\centering
\scriptsize
\lstset{
    escapeinside={@}{@},
    numbers=left,xleftmargin=1em,frame=single,framexleftmargin=0.5em,
    basicstyle=\ttfamily\scriptsize, boxpos=c, numberstyle=\tiny,
    deletekeywords={true}
}
\begin{lstlisting}
https://api.github.com/search/repositories?q=language:java
        +stars:>=100+pushed:>=2016-01-01
        +mvn%20in:readme+sort:stars
\end{lstlisting}
    \caption{\label{fig:subject-query} Query to the \github{} API for
    projects with the following criteria: (1) Java, (2) at least 100
    stars, (3) updated on January 1st, 2016 (or later), (4) contains
    the string \emph{mvn} in the \emph{readme} file. Output is
    paginated in descending order of stars.}
\end{figure}

A Maven project may contain several sub-modules with multiple \pomf{}
files. We considered only projects with a \pomf{} file located in the
root directory.  As of March 25th 2017, our search criteria returned
\SubjectsGithub{} subjects.  From the \SubjectsGithub{} downloaded
projects, \SubjectsGithubNotMaven{} were not Maven projects (or did
not have a \pomf{} in the root directory) and
\SubjectsGithubNotTestable{} were in an untestable revision (\eg,
missing dependencies or incompatible testing environment). To ensure
that our sample set is stable, we retested the remaining subjects to
eliminate flaky tests (total of \SubjectsGithubFlaky{} subjects). Our
final set consists in \numSubjs{} testable subjects.

\subsection{Setup and Replication}
\label{sec:setup}

To run our experiments, we used a Core i7-4790 (3.60 GHz) Intel
processor machine with eight virtual CPUs (four cores with two native
threads each) and 16GB of memory, running Ubuntu 14.04 LTS Trusty Tahr
(64-bit version). Software settings include \Comment{the Linux
\emph{sysstat} package to measure performance, }git to fetch subjects,
Java 8 and Maven 3.3.9 to build and test subjects. Our evaluation
scripts depends on Python 3.4 and Bash to execute and R to analyze the
data. For replication, all source artifacts are publicly available at
\Fix{create gh-pages}, including supporting scripts (\eg, the script
that test subjects and generates the raw data), and the full list of
projects. \Comment{, and a \emph{Vagrantfile} to emulate our hardware
and all software dependencies.}

\subsection{Answering research question \numRQA{}}
\label{sec:rqA}

\begin{itemize}
    \item \emph{\RQA}
\end{itemize}

To evaluate the frequency of time-consuming regression tests, we
considered the \numSubjs{} subjects from Section~\ref{sec:subjects}
and compared the elapsed time to run their tests.
Figure~\ref{fig:mvn-execution} shows the commands used in our script
to test each subject. The main loop (lines 5-11) iterates over the
list of subjects and invokes Maven in separated steps (lines 7-9). To
avoid inflating the measured time we executed Maven in different
steps: we first compiled the source and test files (line 7), made all
dependencies available locally (line 8) and later, we ran the tests in
offline mode (line 9) to bypass package updates. After the execution,
we used a regular expression on the output log to extract the elapsed
time (line 10). Before collecting the time cost, we executed all
subjects and randomly selected 100 logs to inspect and identify
non-related tasks (\eg, \emph{javadoc} generation and static analysis)
to ignore during the experiment (lines 1-3).  Also, measured the
approximated overhead from the build system after skipping non-related
tasks. We identified the generated test reports and compared the
difference between the elapsed time reported from Maven and the sum of
all test cases executed.
To avoid noise from operating system events, we used a dedicated
server remotely via SSH with the operating system running only
essential services (\eg, the SSH server). In addition, we configured
the \CodeIn{isolcpus} option from the Linux Kernel \cite{linux-kernel}
to isolate six virtual CPUs to execute our experiment and the
remaining CPUs to run the operating system. By isolating a set of
virtual CPUs, we prevent context-switching from experiment processes
and OS-related processes.

\section{Evaluation}
\label{sec:eval}

We are interested in understanding the prevalence of time-consuming
test suites and main sources of cost. We want to understand how the
execution cost is distributed on test cases within a test suite and
how developers approach test execution. Based on that, we study
parallelization of testing frameworks.  More precisely, we investigate
how prevalent test parallelization is, the potential for improving
execution cost, issues of flakiness that hinders the use of
parallelization, and how to address those issues.  More specifically,
we pose the following research questions:

\newcommand{\numRQA}{RQ1}
\newcommand{\numRQB}{RQ2}
\newcommand{\numRQC}{RQ3}
\newcommand{\numRQD}{RQ4}
%%\newcommand{\numRQE}{RQ5}
\newcommand{\numRQF}{RQ5}

\newcommand{\RQA}{How prevalent is the occurence of time-consuming
  regression test suites in open-source projects?}
\newcommand{\RQB}{How the time cost is distributed across test cases?}
\newcommand{\RQC}{How often parallelization settings appear in build
  files?}
\newcommand{\RQD}{What is the impact of parallelization?}
%%\newcommand{\RQE}{What factors contribute to improve performance
%%  through parallelization?\Jbc{Previously as "When does
%%    parallelization works and when it does not so great?"}}
\newcommand{\RQF}{What are the limitations of low-level parallelism?}

\begin{itemize}
    \item \textbf{\numRQA.} \RQA
    \item \textbf{\numRQB.} \RQB
    \item \textbf{\numRQC.} \RQC
    \item \textbf{\numRQD.} \RQD
%%    \item \textbf{\numRQE.} \RQE
    \item \textbf{\numRQF.} \RQF
\end{itemize}

%%\newcommand{\RQB}{What is the distribution of CPU and IO bound
%%regression test suites from the sample set?}
%%
%%\newcommand{\RQC}{How uniformly distributed is the execution time
%%across test cases in costly projects?}
%%
%%\newcommand{\RQD}{How often developers use the parallelism features
%%from build systems to improve runtime performance?}

The first research question addresses the prevalence of long-running
test suites. We are interested to know if costly test suites are
common in open-source projects.  The second research question
addresses the relationship of test cases and the overall execution
cost: we are interested to investigate how the time cost is
distributed among test cases.  In the third research question, we are
in understanding if developers are aware of low-level parallelism
features available out-of-the-box through build systems (\eg,
configuring Maven to use JUnit with multiple threads to speedup test
execution). In addition, we want to identify which configurations are
often used and why they more popular (if any).  The fourth research
question addresses the impact of low-level parallelism on the
regression tests from our sample set. For subjects that use
parallelization settings, we are interested to compare the execution
performance when these settings are activated and deactivated (\ie,
tests run sequentially).
%%\Jbc{This RQ has to be reworked:} \Fix{The fifth research question
%%addresses the characteristics of regression tests from the previous
%%experiment.  More specifically, we want to investigate if there is a
%%relation between the balance of test execution (\ie, how uniformly
%%distributed is the execution time across tests cases) and the usage of
%%computational resources (\ie, if tests are mostly CPU or IO intense)
%%that impacts the effectiveness of parallelization}.
Finally, the fifth research question discusses the limitations and
insights to overcome the pitfalls of parallelization.

\Comment{
    \Fix{distribution of execution time per test case. For each subject
    identified in the first research question, we investigate how
    balanced is the cost of the test suite in contrast to the cost of
    test cases and if there are subjects where the time cost is mostly
    dominated by a small fraction of test cases.} \Fix{The third research
    question addresses the distribution of regression tests according
    to the use of computational resources.  We are interested in
    investigating if regression test suites are CPU intensive and if there
    are opportunities to improve performance. The RQ4 addresses}
    \Fix{...elaborate...}

    The rationale is that if the time cost of a regression test is equally
    distributed among test cases, the execution cost could be potentially
    improved by running tests in parallel (in contrast to the scenario
    where only one test case dominates most of the execution time).
}

\subsection{Subjects}
\label{sec:subjects}

We evaluated the characteristics of regression tests in open-source
development from a sample set of Java projects from \github{}.  We are
interested to evaluate non-trivial test suites on popular projects
that are in activity. We used the \github{}'s search
API~\cite{githubsearch} to fetch Java projects according to the
following criteria: (1) the primary language must be Java\footnote{In
case of projects in multiple languages, the \github{} API considers
the predominant language as the primary language.}; (2) the project
has at least 100 stars; (3) latest update on (or after) January 1st,
2016; (4) the \emph{readme} file contains the string \emph{mvn}.

On \github{}, when a user adds a star to a project, s/he demonstrated
appreciation and bookmarked it for later
reference~\cite{github-stars}.  Although there is not a specific range
for the number of stars, our criteria is an estimation to avoid
trivial subjects: we assumed that \github{} users are likely to
demonstrate interest on well-tested projects. The third criteria is a
constraint to skip projects without recent activity. The fourth
criteria is an approximation to find projects with Maven support. The
rationale is that if the string \emph{mvn} exists in the \emph{readme}
file, it may represent a Maven call (\eg, to compile or test the
project). We used Maven as a reference due to its popularity on Java
projects and to automate our evaluation scripts.
Figure~\ref{fig:subject-query} illustrates the query as an HTTP
request.

\begin{figure}[h!]
\centering
\scriptsize
\lstset{
    escapeinside={@}{@},
    numbers=left,xleftmargin=1em,frame=single,framexleftmargin=0.5em,
    basicstyle=\ttfamily\scriptsize, boxpos=c, numberstyle=\tiny,
    deletekeywords={true}
}
\begin{lstlisting}
https://api.github.com/search/repositories?q=language:java
        +stars:>=100+pushed:>=2016-01-01
        +mvn%20in:readme+sort:stars
\end{lstlisting}
    \caption{\label{fig:subject-query} Query to the \github{} API for
    projects with the following criteria: (1) Java, (2) at least 100
    stars, (3) updated on January 1st, 2016 (or later), (4) contains
    the string \emph{mvn} in the \emph{readme} file. Output is
    paginated in descending order of stars.}
\end{figure}

A Maven project may contain several sub-modules with multiple \pomf{}
files. We considered only projects with a \pomf{} file located in the
root directory.  As of March 25th 2017, our search criteria returned
\SubjectsGithub{} subjects.  From the \SubjectsGithub{} downloaded
projects, \SubjectsGithubNotMaven{} were not Maven projects (or did
not have a \pomf{} in the root directory) and
\SubjectsGithubNotTestable{} were in an untestable revision (\eg,
missing dependencies or incompatible testing environment). To ensure
that our sample set is stable, we retested the remaining subjects to
eliminate flaky tests (total of \SubjectsGithubFlaky{} subjects). Our
final set consists in \numSubjs{} testable subjects.

\subsection{Setup and Replication}
\label{sec:setup}

To run our experiments, we used a Core i7-4790 (3.60 GHz) Intel
processor machine with eight virtual CPUs (four cores with two native
threads each) and 16GB of memory, running Ubuntu 14.04 LTS Trusty Tahr
(64-bit version). Software settings include \Comment{the Linux
\emph{sysstat} package to measure performance, }git to fetch subjects,
Java 8 and Maven 3.3.9 to build and test subjects. Our evaluation
scripts depends on Python 3.4 and Bash to execute and R to analyze the
data. For replication, all source artifacts are publicly available at
\Fix{create gh-pages}, including supporting scripts (\eg, the script
that test subjects and generates the raw data), and the full list of
projects. \Comment{, and a \emph{Vagrantfile} to emulate our hardware
and all software dependencies.}

\subsection{Answering research question \numRQA{}}
\label{sec:rqA}

\begin{itemize}
    \item \emph{\RQA}
\end{itemize}

To evaluate the frequency of time-consuming regression tests, we
considered the \numSubjs{} subjects from Section~\ref{sec:subjects}
and compared the elapsed time to run their tests.
Figure~\ref{fig:mvn-execution} shows the commands used in our script
to test each subject. The main loop (lines 5-11) iterates over the
list of subjects and invokes Maven in separated steps (lines 7-9). To
avoid inflating the measured time we executed Maven in different
steps: we first compiled the source and test files (line 7), made all
dependencies available locally (line 8) and later, we ran the tests in
offline mode (line 9) to bypass package updates. After the execution,
we used a regular expression on the output log to extract the elapsed
time (line 10). Before collecting the time cost, we executed all
subjects and randomly selected 100 logs to inspect and identify
non-related tasks (\eg, \emph{javadoc} generation and static analysis)
to ignore during the experiment (lines 1-3).  Also, measured the
approximated overhead from the build system after skipping non-related
tasks. We identified the generated test reports and compared the
difference between the elapsed time reported from Maven and the sum of
all test cases executed.
To avoid noise from operating system events, we used a dedicated
server remotely via SSH with the operating system running only
essential services (\eg, the SSH server). In addition, we configured
the \CodeIn{isolcpus} option from the Linux Kernel \cite{linux-kernel}
to isolate six virtual CPUs to execute our experiment and the
remaining CPUs to run the operating system. By isolating a set of
virtual CPUs, we prevent context-switching from experiment processes
and OS-related processes.

\input{codes/evaluation}

\Comment{\Jbc{Ensure to describe that: subjects that could not run the tests
due to incompatible testing environment were removed (likely to happen
for subjects that rely on database systems and other external
factors), the exit status for all executions are consistent and I
removed flaky projs}}

We ran each subject's tests three times and grouped them according to
the time cost (in average): tests that run in less than a minute
(\shortg{} group), tests than run in one minute and less than five
minutes (\medg{} group), and tests that run in five (or more) minutes
(\longg{} group). We based our classification from a previous work
\cite{gligoric-etal-issta2015} and added the \medg{} group due to the
variability of the time cost from subjects out of the \shortg{} group.
\Jbc{these results will be updated $\rightarrow$}
Figure~\ref{fig:rq1-boxplot} shows the time cost distribution of each
group: the \shortg{} group is the most right skewed distribution with
outliers closed to the \medg{} group and 50\% of the subjects run in
less than 15 seconds; the median from the \medg{} group is nearly two
minutes and most of the subjects run in less than four minutes; most
of the \longg{} group runs in less than 25 minutes but has outliers
that require more than 50 minutes to execute.
Figure~\ref{fig:rq1-barplot} summarizes the proportion of subjects
from each group.

\begin{figure}[ht]
    \centering
    \begin{subfigure}{0.24\textwidth}
        \centering
        \includegraphics[width=\textwidth]{plots/barplot-timecost.pdf}
        \caption{\label{fig:rq1-barplot}}
    \end{subfigure}%
    ~
    \begin{subfigure}{0.24\textwidth}
        \centering
        \includegraphics[width=\textwidth]{plots/boxplot-timecost.pdf}
        \caption{\label{fig:rq1-boxplot}}
    \end{subfigure}%
    \caption{(a) Subjects grouped by time cost ($t$): short run ($t <
    1m$), medium run ($1m \le t < 5m$), and long run ($5m \le t$); (b)
    Distribution of time cost by group.}
\end{figure}

\Jbc{these results will be updated $\rightarrow$}
It is important to mention that Figure~\ref{fig:rq1-barplot}
represents a lower bound estimation for time cost because some
projects omit long-running tests on their default execution. For
instance, it is necessary to invoke Maven with additional parameters
to run integration tests from project \CodeIn{apache.maven-surefire}
(nearly 30 min in our environment).  Also, some tests may finish
earlier than expected due to existing bugs on the ongoing revision.
From the \numSubjs{} compiled projects, 286 successfully executed
all tests and 118 reported test failures.

\begin{center}
\fbox{
\begin{minipage}{8cm}
    \textit{Answering \numRQA{}:}~\emph{Given that
    \percentMedLongRunning{} of the projects are costly (\ie,
    \numMedLong{} projects from \medg{} and \longg{}), we concluded
    that time-consuming regression test suites are relatively frequent
    in open-source projects.}
\end{minipage}
}
\end{center}

\subsection{Answering research question \numRQB{}}
\label{sec:rqB}

\begin{itemize}
    \item \emph{\RQB}
\end{itemize}

\Jbc{pending updates! $\rightarrow$}
Section~\ref{sec:rqA} showed that medium and long-running projects are
not uncommon; they account to nearly \percentMedLongRunning{} of the
\numSubjs{} projects we analyzed.  However, parallelism by itself does
not assure speedups on those projects.  One sensible factor that is
important to the effectiveness of low-level parallelism is the
distribution of test costs in the test suite.  In the limit, if cost
is dominated by a single test, it is unlikely that parallelization
will be beneficial as a test method is the smallest working unit in
test frameworks (see Section~\ref{sec:frameworks}).

\begin{figure}[ht]
    \centering

    \begin{subfigure}{0.5\textwidth}
      \centering
      \includegraphics[width=\textwidth]{R/testcost-med.pdf}
      \caption{\label{fig:medtcost}Medium}
    \end{subfigure}\\
    
    \begin{subfigure}{0.5\textwidth}
      \centering
      \includegraphics[width=\textwidth]{R/testcost-long.pdf}
      \caption{\label{fig:longtcost}Long}
    \end{subfigure}%
    
    \begin{subfigure}{0.5\textwidth}
      \centering
      \begin{tabular}{rrrr}
        \toprule
        & $\sigma\leq1$ & $1<\sigma\leq5$ & $\sigma\ge5$ \\
        \midrule    
        Medium & 17 & 20 & 7 \\
        Long   &  9 & 14 & 5 \\
        \bottomrule
      \end{tabular}
      \caption{\label{fig:sd}...\Fix{Looks like there are missing
        projects...}}    
    \end{subfigure}%

    \caption{\label{fig:time-distributions}Distribution of time cost per project.}    
\end{figure}

\sloppy Figures~\ref{fig:medtcost} and~\ref{fig:longtcost} show the
time distribution of tests per project.  We observed that the average
median value of execution cost for a test was relatively small (dashed
horizontal red lines), namely 0.41s for \medg{} projects and 0.23s for
\longg{} projects.  Note that correlation between cost of test suites
and cost of their individual tests is not to be expected.  The
standard deviation ($\sigma$) associated with each ditribution was
relatively low.  Figure~\ref{fig:sd} shows the number of projects
within specific ranges of $\sigma$ values.  We also observed that
projects with very high standard deviations may or may not be
indicative of monopolization of execution by a small number of tests.
For example, the highest value of $\sigma$ ocurred in project
\CodeIn{wildfly-swarm}, whose test suite was classified as
long-running.  This project contains only 65 tests, 62 of which take
less than 0.5s to run, 1 of which takes nearly 3s to run, and 2 of
which take 40m to run.  This is an example where test suite
parallelization may not be very beneficial as $>$99\% execution cost
is dominated by only 3\% of the tests.  Without these two tests this
project would have been classified as short-running.  A closer
inspection in the data indicates that the project
\CodeIn{wildfly-swarm} was an extreme case; we did not find projects
with similar characteristics of time monopolization.  Project
\CodeIn{jenkinsci.gerrit-trigger-plugin} is also classified as
long-running and has the second highest value of $\sigma$.  For this
project, however, cost is distributed more smoothly across 529 tests,
of which 119 take more than 1s to run.  


\begin{center}
\fbox{
\begin{minipage}{8cm}
    \textit{Answering \numRQB{}:}~\emph{Overall, results indicate that
    projects with a very small number of tests monopolizing end-to-end
    execution time were rare.}
\end{minipage}
}
\end{center}

%% We are interested to know whether
%% most of the execution cost of a subject is dominated by a small subset
%% of test cases or if the cost is nearly equally distributed. 

%% We also evaluated the dispersion of time distributions (one
%% distribution per project) to answer research question \numRQB{}.  To
%% measure dispersion \emph{across} projects we used Relative Standard
%% Deviation (RSD)~\cite{everitt-book-stats-2010}.  Note that, if we were
%% to analyze each project in isolation, the standard deviation of a
%% distribution ($\sigma$) would suffice to quantify how dispersed the
%% (time) distribution is.  However, in our case, we would like to be
%% able to compare and summarize dispersion across projects.  The RSD,
%% which is obtained dividing the standard deviation by the mean ($\mu$)
%% of a distribution, provides such normalization effect.  This metric
%% provides a lower bound (zero) but not an upper bound (somewhere close
%% to 1).  The smaller (larger) the value of RSD the more (less) uniform
%% the distribution is.  Consequently, the lower the value of RSD the
%% more parallelizable a test suite should be.



%% \begin{figure}[h!]
%%   \centering
%%   \includegraphics[width=0.5\textwidth]{R/testcost.pdf}  
%%   \caption{\label{fig:relativesd}Distribution of RSD ($\sigma/\mu$)
%%     across projects.}
%% \end{figure}

%% Figure~\ref{fig:relativesd} shows the distribution of RSD across
%% medium and long-running projects.  Results show that the distribution
%% is skewed to the right indicating that test costs are relatively well
%% distributed in most costly projects we analyzed \Fix{$\leftarrow$
%%   confirm}.

%% analyzed the execution time
%% for the \numMedLong{} projects from the \longg{} and \medg{} groups
%% (see Section~\ref{sec:rqA}).
%% For each subject we calculated the
%% relative standard deviation of the test cases: we collected the
%% elapsed time of each individual test, calculated the standard
%% deviation, and divided by the mean. \Jbc{I need to clarify the
%%   relationship "well/bad-balanced" regression test and relative
%%   standard deviation}

%% Results indicated that \Fix{...elaborate...}. \Jbc{We may identify
%% different groups of subjects}\Fix{TODO: collect data + compute the
%% statistic, create a scatter plot to identify groups of subjects}

%% \Jbc{I shouldn't introduce parallelization arguments here but we have
%% to address this at some point:
%% Regression tests that are well distributed may benefit from
%% parallelism since more tests executes at the same time while the
%% opposite scenario may require a different approach. In the later
%% scenario, executing tests in parallel may have insignificant impact
%% since a small subset of test cases dominates the execution.}

\subsection{Answering research question \numRQC{}}
\label{sec:rqC}

\begin{itemize}
    \item \emph{\RQC}
\end{itemize}

To evaluate the presence of parallelization settings (see Section
\ref{sec:modes}), we considered all projects classified as \medg{} and
\longg{}-running, a total of \numMedLong{} subjects (see
Figure~\ref{fig:rq1-barplot}). To reduce noise we did not consider
projects with \shortg{}-running test suites as we observed that
\percentShortSequential{} of those projects do not use parallelism;
including them would favor sequential execution in our analysis.

To answer \numRQC{} we initially mined build files (\ie, \pomf{}) that
contain parallelization keywords, namely \CodeIn{forkMode},
\CodeIn{forkCount}, and \CodeIn{parallel}.  According to the Maven
surefire documentation\Fix{cite surefire doc}, any configuration to
run tests in parallel must contain one of these keywords. This mining
step, however, is not sound as a matching project may not indicate a
project with a valid configuration for running tests in parallel.
False positive can happen because of comments, for instance.  
To eliminate the cases of false positives and also to categorize true
positive cases, we complemented the initial mining step with a manual
inspection of files.


%% settings); the second step (inspection) consists in a manual
%% inspection to confirm the presence of parallelism settings in the
%% build file and classify them according to the parallelism level.
%% Figure \Fix{removed} describes the discovery step: we list the paths
%% of all build files and filter only the files that contain any of the

Figure~\ref{tab:inspection-table} summarizes our results. The first
column indicates the group of projects according to their time cost.
The second column indicates the number of build files per group.  The
third column is a subset from the previous column and represents the
number of build files found in the mining step (including
false-positives highlighted in parentheses). The last column indicates
the ratio of projects with parallelization settings.

From the \numMedLong{} subjects, we found \pomMedLong{} \pomf{} files
where \numPomMatched{} of those files match one of the keywords
mentioned above.  From these \numPomMatched{} build files, we
eliminated six false-positives: two were names of project modules
(\eg, \emph{camel-cookbook-parallel-processing}), one was part of a
comment, and three were commented configurations. The
\numPomMatchedValid{} build files with valid configurations are
distributed across \numProjectsPar{} projects from our sample.
\emph{From these results we found that $\sim$51\% of medium and
long-running projects do not use parallel features to run test
suites.}\Mar{please make it consistent with research
question}\Mar{explain this is over(under)-estimated...}

\begin{figure}[ht!]
    \centering
    \begin{tabular*}{0.48\textwidth}{@{\extracolsep{\fill}}cccc}
        \toprule
        \multirow{2}{*}{Group} %1st row, 1st cell
            & \multirow{2}{*}{\# \pomf{}}
            & \# \pomf{} matched
            & \# projects matched\\%
        % 2nd row - empty cell
            & % empty cell
            & (false-positives)
            & / total\\%
        \midrule%
        Long   & 1613 & 79 (1) & 19 / \numLong{}\\%
        Medium & 1238 & 30 (5) & 22 / \numMed{}\\%
        \midrule%
        Total % last row, first cell
            & \pomMedLong{}
            & \numPomMatched{} (6)
            & \numProjectsPar{} / \numMedLong{}\\%
        \bottomrule%
    \end{tabular*}
    \caption{Presence of parallelization settings in build files: the
    first column indicates the group of projects according to their
    time cost; the second column indicates the number of build files;
    the third column is the subset of files with parallelization
    keywords; the last column indicates the ratio of projects with
    parallelism support.}
    \label{tab:inspection-table} 
\end{figure}

\subsubsection{Distribution of parallel modes}

From the \numProjectsPar{} projects identified above, we investigated
further the \numPomMatchedValid{} build files with parallel settings.
We analyzed the support and distribution of parallel modes from this
subset of projects. To calculate the distribution of parallel modes,
we considered only the presence of the mode in at least one of the
project settings.  Recall that a build file may contain more than one
parallel setting and a project may contain several sub-modules with
build files.  In case the value of a parallel option is resolved
dynamically (\eg, via command-line argument or system variable) we
compute all modes related to the option. For instance, depending on
the value, the \CodeIn{parallel} option can be \Seq{} (\CodeIn{none}),
\ParClassSeqMeth{} (\CodeIn{classes}), \SeqClassParMeth{},
(\CodeIn{methods}), and \ParClassParMeth{} (\CodeIn{all}).
Figure~\ref{fig:freqmodes} summarizes our findings. \Jbc{Given that we
checked each configuration, we could elaborate how these settings are
used.} \Fix{Missing conclusion: Fork the most used configuration}

\begin{figure}[h!]
    \centering
    \includegraphics[width=0.32\textwidth]{plots/parallel-modes.pdf}
    \caption{\label{fig:freqmodes}Distribution of parallel modes
    identified in a subset of \numProjectsPar{} projects (unused modes
    omitted). A single project may have support to more than one
    parallel mode.}
\end{figure}

\subsection{Answering research question \numRQD{}}
\label{sec:rqD}

\begin{itemize}
    \item \emph{\RQD}
\end{itemize}

To answer \numRQD{}, we considered the \numProjectsPar{} subjects with
parallelization settings identified in \numRQC{} and compared their
time cost with sequential execution. \Jbc{should have a sentence to
motivate this RQ} 
\Jbc{explain methodology}.

\Fix{---------------------}

%%To evaluate the distribution of execution time per project, we sorted
%%the test cases by decreasing order of elapsed time and calculated the
%%number of tests executed in 90\% of the total time. Later, we reported
%%the \Fix{balance} of execution time by dividing the number of tests
%%that represents 90\% of the execution time by the number of tests
%%cases. For instance, a balance of 50\% indicates \Fix{...}.  \Fix{We
%%collected the elapsed time from test cases for each generated report.
%%Maven Surefire generates an XML report with execution information
%%(\eg, number of skipped tests and elapsed time) per test suite
%%\Jbc{Should I use the previous sentence as a footnote or should I
%%delete it?}. We noticed that some test cases reported an elapsed time
%%of zero: since the reported time is in milliseconds, some tests may
%%execute in a shorter time. \Fix{..to be continued...}}. Results
%%indicate that \Fix{...}.
%%
%%\begin{figure}[h!]
%%    \centering
%%    \includegraphics[width=0.4\textwidth]{results/plots/balance.pdf}
%%    \caption{\Fix{balance}}
%%\end{figure}

%% \subsection{Answering research question RQ3}
%% \label{sec:rqThree}
%% 
%% \begin{itemize}
%%     \item \RQB
%% \end{itemize}
%% 
%% To evaluate the distribution of CPU and IO intensive test suites from
%% the sample set, we used the command \emph{sar} to monitor the system
%% activity in background while tests ran. \emph{Sar} is a command that
%% collects and reports statistics (\eg, percentage of IO waiting and
%% usage of CPU in user mode) based on the kernel activity and it is
%% highly configurable to collect detailed information (\eg, usage of a
%% specific processor core or percentage of network interface
%% utilization). We configured \emph{sar} to report \Fix{...explain how
%% we executed and what fields we are interested}. \Fix{explain fields}.
%% Figure \Fix{A} shows the distribution of subjects grouped in intervals
%% of \Fix{W}\% of CPU utilization. Results indicates that \Fix{...}
%% 
%% \begin{figure}[h!]
%%     \centering
%%     \includegraphics[width=0.4\textwidth]{results/plots/cpuness.pdf}
%%     \caption{\Fix{cpu usage}}
%% \end{figure}
%% 
%% \Comment{we proposed the definition of \emph{cpuness}
%% computed as the follow: $((user\_t + system\_t) / elapsed\_t) * 100$,
%% where \emph{user\_t} is the elapsed time of execution in \emph{user
%% mode}, \emph{system\_t} is the elapsed in \emph{kernel mode}, and
%% \emph{elapsed\_t} is the elapsed time to finish the execution. We
%% measured the \emph{cpuness} of each regression test \Fix{...elaborate
%% the meaning of cpuness} \Fix{Describe how I measured user, system and
%% "wall" time}.  \Fix{Explain results}.  \Fix{show plots}}
%% 


\Comment{\Jbc{Ensure to describe that: subjects that could not run the tests
due to incompatible testing environment were removed (likely to happen
for subjects that rely on database systems and other external
factors), the exit status for all executions are consistent and I
removed flaky projs}}

We ran each subject's tests three times and grouped them according to
the time cost (in average): tests that run in less than a minute
(\shortg{} group), tests than run in one minute and less than five
minutes (\medg{} group), and tests that run in five (or more) minutes
(\longg{} group). We based our classification from a previous work
\cite{gligoric-etal-issta2015} and added the \medg{} group due to the
variability of the time cost from subjects out of the \shortg{} group.
\Jbc{these results will be updated $\rightarrow$}
Figure~\ref{fig:rq1-boxplot} shows the time cost distribution of each
group: the \shortg{} group is the most right skewed distribution with
outliers closed to the \medg{} group and 50\% of the subjects run in
less than 15 seconds; the median from the \medg{} group is nearly two
minutes and most of the subjects run in less than four minutes; most
of the \longg{} group runs in less than 25 minutes but has outliers
that require more than 50 minutes to execute.
Figure~\ref{fig:rq1-barplot} summarizes the proportion of subjects
from each group.

\begin{figure}[ht]
    \centering
    \begin{subfigure}{0.24\textwidth}
        \centering
        \includegraphics[width=\textwidth]{plots/barplot-timecost.pdf}
        \caption{\label{fig:rq1-barplot}}
    \end{subfigure}%
    ~
    \begin{subfigure}{0.24\textwidth}
        \centering
        \includegraphics[width=\textwidth]{plots/boxplot-timecost.pdf}
        \caption{\label{fig:rq1-boxplot}}
    \end{subfigure}%
    \caption{(a) Subjects grouped by time cost ($t$): short run ($t <
    1m$), medium run ($1m \le t < 5m$), and long run ($5m \le t$); (b)
    Distribution of time cost by group.}
\end{figure}

\Jbc{these results will be updated $\rightarrow$}
It is important to mention that Figure~\ref{fig:rq1-barplot}
represents a lower bound estimation for time cost because some
projects omit long-running tests on their default execution. For
instance, it is necessary to invoke Maven with additional parameters
to run integration tests from project \CodeIn{apache.maven-surefire}
(nearly 30 min in our environment).  Also, some tests may finish
earlier than expected due to existing bugs on the ongoing revision.
From the \numSubjs{} compiled projects, 286 successfully executed
all tests and 118 reported test failures.

\begin{center}
\fbox{
\begin{minipage}{8cm}
    \textit{Answering \numRQA{}:}~\emph{Given that
    \percentMedLongRunning{} of the projects are costly (\ie,
    \numMedLong{} projects from \medg{} and \longg{}), we concluded
    that time-consuming regression test suites are relatively frequent
    in open-source projects.}
\end{minipage}
}
\end{center}

\subsection{Answering research question \numRQB{}}
\label{sec:rqB}

\begin{itemize}
    \item \emph{\RQB}
\end{itemize}

\Jbc{pending updates! $\rightarrow$}
Section~\ref{sec:rqA} showed that medium and long-running projects are
not uncommon; they account to nearly \percentMedLongRunning{} of the
\numSubjs{} projects we analyzed.  However, parallelism by itself does
not assure speedups on those projects.  One sensible factor that is
important to the effectiveness of low-level parallelism is the
distribution of test costs in the test suite.  In the limit, if cost
is dominated by a single test, it is unlikely that parallelization
will be beneficial as a test method is the smallest working unit in
test frameworks (see Section~\ref{sec:frameworks}).

\begin{figure}[ht]
    \centering

    \begin{subfigure}{0.5\textwidth}
      \centering
      \includegraphics[width=\textwidth]{R/testcost-med.pdf}
      \caption{\label{fig:medtcost}Medium}
    \end{subfigure}\\
    
    \begin{subfigure}{0.5\textwidth}
      \centering
      \includegraphics[width=\textwidth]{R/testcost-long.pdf}
      \caption{\label{fig:longtcost}Long}
    \end{subfigure}%
    
    \begin{subfigure}{0.5\textwidth}
      \centering
      \begin{tabular}{rrrr}
        \toprule
        & $\sigma\leq1$ & $1<\sigma\leq5$ & $\sigma\ge5$ \\
        \midrule    
        Medium & 17 & 20 & 7 \\
        Long   &  9 & 14 & 5 \\
        \bottomrule
      \end{tabular}
      \caption{\label{fig:sd}...\Fix{Looks like there are missing
        projects...}}    
    \end{subfigure}%

    \caption{\label{fig:time-distributions}Distribution of time cost per project.}    
\end{figure}

\sloppy Figures~\ref{fig:medtcost} and~\ref{fig:longtcost} show the
time distribution of tests per project.  We observed that the average
median value of execution cost for a test was relatively small (dashed
horizontal red lines), namely 0.41s for \medg{} projects and 0.23s for
\longg{} projects.  Note that correlation between cost of test suites
and cost of their individual tests is not to be expected.  The
standard deviation ($\sigma$) associated with each ditribution was
relatively low.  Figure~\ref{fig:sd} shows the number of projects
within specific ranges of $\sigma$ values.  We also observed that
projects with very high standard deviations may or may not be
indicative of monopolization of execution by a small number of tests.
For example, the highest value of $\sigma$ ocurred in project
\CodeIn{wildfly-swarm}, whose test suite was classified as
long-running.  This project contains only 65 tests, 62 of which take
less than 0.5s to run, 1 of which takes nearly 3s to run, and 2 of
which take 40m to run.  This is an example where test suite
parallelization may not be very beneficial as $>$99\% execution cost
is dominated by only 3\% of the tests.  Without these two tests this
project would have been classified as short-running.  A closer
inspection in the data indicates that the project
\CodeIn{wildfly-swarm} was an extreme case; we did not find projects
with similar characteristics of time monopolization.  Project
\CodeIn{jenkinsci.gerrit-trigger-plugin} is also classified as
long-running and has the second highest value of $\sigma$.  For this
project, however, cost is distributed more smoothly across 529 tests,
of which 119 take more than 1s to run.  


\begin{center}
\fbox{
\begin{minipage}{8cm}
    \textit{Answering \numRQB{}:}~\emph{Overall, results indicate that
    projects with a very small number of tests monopolizing end-to-end
    execution time were rare.}
\end{minipage}
}
\end{center}

%% We are interested to know whether
%% most of the execution cost of a subject is dominated by a small subset
%% of test cases or if the cost is nearly equally distributed. 

%% We also evaluated the dispersion of time distributions (one
%% distribution per project) to answer research question \numRQB{}.  To
%% measure dispersion \emph{across} projects we used Relative Standard
%% Deviation (RSD)~\cite{everitt-book-stats-2010}.  Note that, if we were
%% to analyze each project in isolation, the standard deviation of a
%% distribution ($\sigma$) would suffice to quantify how dispersed the
%% (time) distribution is.  However, in our case, we would like to be
%% able to compare and summarize dispersion across projects.  The RSD,
%% which is obtained dividing the standard deviation by the mean ($\mu$)
%% of a distribution, provides such normalization effect.  This metric
%% provides a lower bound (zero) but not an upper bound (somewhere close
%% to 1).  The smaller (larger) the value of RSD the more (less) uniform
%% the distribution is.  Consequently, the lower the value of RSD the
%% more parallelizable a test suite should be.



%% \begin{figure}[h!]
%%   \centering
%%   \includegraphics[width=0.5\textwidth]{R/testcost.pdf}  
%%   \caption{\label{fig:relativesd}Distribution of RSD ($\sigma/\mu$)
%%     across projects.}
%% \end{figure}

%% Figure~\ref{fig:relativesd} shows the distribution of RSD across
%% medium and long-running projects.  Results show that the distribution
%% is skewed to the right indicating that test costs are relatively well
%% distributed in most costly projects we analyzed \Fix{$\leftarrow$
%%   confirm}.

%% analyzed the execution time
%% for the \numMedLong{} projects from the \longg{} and \medg{} groups
%% (see Section~\ref{sec:rqA}).
%% For each subject we calculated the
%% relative standard deviation of the test cases: we collected the
%% elapsed time of each individual test, calculated the standard
%% deviation, and divided by the mean. \Jbc{I need to clarify the
%%   relationship "well/bad-balanced" regression test and relative
%%   standard deviation}

%% Results indicated that \Fix{...elaborate...}. \Jbc{We may identify
%% different groups of subjects}\Fix{TODO: collect data + compute the
%% statistic, create a scatter plot to identify groups of subjects}

%% \Jbc{I shouldn't introduce parallelization arguments here but we have
%% to address this at some point:
%% Regression tests that are well distributed may benefit from
%% parallelism since more tests executes at the same time while the
%% opposite scenario may require a different approach. In the later
%% scenario, executing tests in parallel may have insignificant impact
%% since a small subset of test cases dominates the execution.}

\subsection{Answering research question \numRQC{}}
\label{sec:rqC}

\begin{itemize}
    \item \emph{\RQC}
\end{itemize}

To evaluate the presence of parallelization settings (see Section
\ref{sec:modes}), we considered all projects classified as \medg{} and
\longg{}-running, a total of \numMedLong{} subjects (see
Figure~\ref{fig:rq1-barplot}). To reduce noise we did not consider
projects with \shortg{}-running test suites as we observed that
\percentShortSequential{} of those projects do not use parallelism;
including them would favor sequential execution in our analysis.

To answer \numRQC{} we initially mined build files (\ie, \pomf{}) that
contain parallelization keywords, namely \CodeIn{forkMode},
\CodeIn{forkCount}, and \CodeIn{parallel}.  According to the Maven
surefire documentation\Fix{cite surefire doc}, any configuration to
run tests in parallel must contain one of these keywords. This mining
step, however, is not sound as a matching project may not indicate a
project with a valid configuration for running tests in parallel.
False positive can happen because of comments, for instance.  
To eliminate the cases of false positives and also to categorize true
positive cases, we complemented the initial mining step with a manual
inspection of files.


%% settings); the second step (inspection) consists in a manual
%% inspection to confirm the presence of parallelism settings in the
%% build file and classify them according to the parallelism level.
%% Figure \Fix{removed} describes the discovery step: we list the paths
%% of all build files and filter only the files that contain any of the

Figure~\ref{tab:inspection-table} summarizes our results. The first
column indicates the group of projects according to their time cost.
The second column indicates the number of build files per group.  The
third column is a subset from the previous column and represents the
number of build files found in the mining step (including
false-positives highlighted in parentheses). The last column indicates
the ratio of projects with parallelization settings.

From the \numMedLong{} subjects, we found \pomMedLong{} \pomf{} files
where \numPomMatched{} of those files match one of the keywords
mentioned above.  From these \numPomMatched{} build files, we
eliminated six false-positives: two were names of project modules
(\eg, \emph{camel-cookbook-parallel-processing}), one was part of a
comment, and three were commented configurations. The
\numPomMatchedValid{} build files with valid configurations are
distributed across \numProjectsPar{} projects from our sample.
\emph{From these results we found that $\sim$51\% of medium and
long-running projects do not use parallel features to run test
suites.}\Mar{please make it consistent with research
question}\Mar{explain this is over(under)-estimated...}

\begin{figure}[ht!]
    \centering
    \begin{tabular*}{0.48\textwidth}{@{\extracolsep{\fill}}cccc}
        \toprule
        \multirow{2}{*}{Group} %1st row, 1st cell
            & \multirow{2}{*}{\# \pomf{}}
            & \# \pomf{} matched
            & \# projects matched\\%
        % 2nd row - empty cell
            & % empty cell
            & (false-positives)
            & / total\\%
        \midrule%
        Long   & 1613 & 79 (1) & 19 / \numLong{}\\%
        Medium & 1238 & 30 (5) & 22 / \numMed{}\\%
        \midrule%
        Total % last row, first cell
            & \pomMedLong{}
            & \numPomMatched{} (6)
            & \numProjectsPar{} / \numMedLong{}\\%
        \bottomrule%
    \end{tabular*}
    \caption{Presence of parallelization settings in build files: the
    first column indicates the group of projects according to their
    time cost; the second column indicates the number of build files;
    the third column is the subset of files with parallelization
    keywords; the last column indicates the ratio of projects with
    parallelism support.}
    \label{tab:inspection-table} 
\end{figure}

\subsubsection{Distribution of parallel modes}

From the \numProjectsPar{} projects identified above, we investigated
further the \numPomMatchedValid{} build files with parallel settings.
We analyzed the support and distribution of parallel modes from this
subset of projects. To calculate the distribution of parallel modes,
we considered only the presence of the mode in at least one of the
project settings.  Recall that a build file may contain more than one
parallel setting and a project may contain several sub-modules with
build files.  In case the value of a parallel option is resolved
dynamically (\eg, via command-line argument or system variable) we
compute all modes related to the option. For instance, depending on
the value, the \CodeIn{parallel} option can be \Seq{} (\CodeIn{none}),
\ParClassSeqMeth{} (\CodeIn{classes}), \SeqClassParMeth{},
(\CodeIn{methods}), and \ParClassParMeth{} (\CodeIn{all}).
Figure~\ref{fig:freqmodes} summarizes our findings. \Jbc{Given that we
checked each configuration, we could elaborate how these settings are
used.} \Fix{Missing conclusion: Fork the most used configuration}

\begin{figure}[h!]
    \centering
    \includegraphics[width=0.32\textwidth]{plots/parallel-modes.pdf}
    \caption{\label{fig:freqmodes}Distribution of parallel modes
    identified in a subset of \numProjectsPar{} projects (unused modes
    omitted). A single project may have support to more than one
    parallel mode.}
\end{figure}

\subsection{Answering research question \numRQD{}}
\label{sec:rqD}

\begin{itemize}
    \item \emph{\RQD}
\end{itemize}

To answer \numRQD{}, we considered the \numProjectsPar{} subjects with
parallelization settings identified in \numRQC{} and compared their
time cost with sequential execution. \Jbc{should have a sentence to
motivate this RQ} 
\Jbc{explain methodology}.

\Fix{---------------------}

%%To evaluate the distribution of execution time per project, we sorted
%%the test cases by decreasing order of elapsed time and calculated the
%%number of tests executed in 90\% of the total time. Later, we reported
%%the \Fix{balance} of execution time by dividing the number of tests
%%that represents 90\% of the execution time by the number of tests
%%cases. For instance, a balance of 50\% indicates \Fix{...}.  \Fix{We
%%collected the elapsed time from test cases for each generated report.
%%Maven Surefire generates an XML report with execution information
%%(\eg, number of skipped tests and elapsed time) per test suite
%%\Jbc{Should I use the previous sentence as a footnote or should I
%%delete it?}. We noticed that some test cases reported an elapsed time
%%of zero: since the reported time is in milliseconds, some tests may
%%execute in a shorter time. \Fix{..to be continued...}}. Results
%%indicate that \Fix{...}.
%%
%%\begin{figure}[h!]
%%    \centering
%%    \includegraphics[width=0.4\textwidth]{results/plots/balance.pdf}
%%    \caption{\Fix{balance}}
%%\end{figure}

%% \subsection{Answering research question RQ3}
%% \label{sec:rqThree}
%% 
%% \begin{itemize}
%%     \item \RQB
%% \end{itemize}
%% 
%% To evaluate the distribution of CPU and IO intensive test suites from
%% the sample set, we used the command \emph{sar} to monitor the system
%% activity in background while tests ran. \emph{Sar} is a command that
%% collects and reports statistics (\eg, percentage of IO waiting and
%% usage of CPU in user mode) based on the kernel activity and it is
%% highly configurable to collect detailed information (\eg, usage of a
%% specific processor core or percentage of network interface
%% utilization). We configured \emph{sar} to report \Fix{...explain how
%% we executed and what fields we are interested}. \Fix{explain fields}.
%% Figure \Fix{A} shows the distribution of subjects grouped in intervals
%% of \Fix{W}\% of CPU utilization. Results indicates that \Fix{...}
%% 
%% \begin{figure}[h!]
%%     \centering
%%     \includegraphics[width=0.4\textwidth]{results/plots/cpuness.pdf}
%%     \caption{\Fix{cpu usage}}
%% \end{figure}
%% 
%% \Comment{we proposed the definition of \emph{cpuness}
%% computed as the follow: $((user\_t + system\_t) / elapsed\_t) * 100$,
%% where \emph{user\_t} is the elapsed time of execution in \emph{user
%% mode}, \emph{system\_t} is the elapsed in \emph{kernel mode}, and
%% \emph{elapsed\_t} is the elapsed time to finish the execution. We
%% measured the \emph{cpuness} of each regression test \Fix{...elaborate
%% the meaning of cpuness} \Fix{Describe how I measured user, system and
%% "wall" time}.  \Fix{Explain results}.  \Fix{show plots}}
%% 


\Comment{\Jbc{Ensure to describe that: subjects that could not run the tests
due to incompatible testing environment were removed (likely to happen
for subjects that rely on database systems and other external
factors), the exit status for all executions are consistent and I
removed flaky projs}}

We ran each subject's tests three times and grouped them according to
the time cost (in average): tests that run in less than a minute
(\shortg{} group), tests than run in one minute and less than five
minutes (\medg{} group), and tests that run in five (or more) minutes
(\longg{} group). We based our classification from a previous work
\cite{gligoric-etal-issta2015} and added the \medg{} group due to the
variability of the time cost from subjects out of the \shortg{} group.
\Jbc{these results will be updated $\rightarrow$}
Figure~\ref{fig:rq1-boxplot} shows the time cost distribution of each
group: the \shortg{} group is the most right skewed distribution with
outliers closed to the \medg{} group and 50\% of the subjects run in
less than 15 seconds; the median from the \medg{} group is nearly two
minutes and most of the subjects run in less than four minutes; most
of the \longg{} group runs in less than 25 minutes but has outliers
that require more than 50 minutes to execute.
Figure~\ref{fig:rq1-barplot} summarizes the proportion of subjects
from each group.

\begin{figure}[ht]
    \centering
    \begin{subfigure}{0.24\textwidth}
        \centering
        \includegraphics[width=\textwidth]{plots/barplot-timecost.pdf}
        \caption{\label{fig:rq1-barplot}}
    \end{subfigure}%
    ~
    \begin{subfigure}{0.24\textwidth}
        \centering
        \includegraphics[width=\textwidth]{plots/boxplot-timecost.pdf}
        \caption{\label{fig:rq1-boxplot}}
    \end{subfigure}%
    \caption{(a) Subjects grouped by time cost ($t$): short run ($t <
    1m$), medium run ($1m \le t < 5m$), and long run ($5m \le t$); (b)
    Distribution of time cost by group.}
\end{figure}

\Jbc{these results will be updated $\rightarrow$}
It is important to mention that Figure~\ref{fig:rq1-barplot}
represents a lower bound estimation for time cost because some
projects omit long-running tests on their default execution. For
instance, it is necessary to invoke Maven with additional parameters
to run integration tests from project \CodeIn{apache.maven-surefire}
(nearly 30 min in our environment).  Also, some tests may finish
earlier than expected due to existing bugs on the ongoing revision.
From the \numSubjs{} compiled projects, 286 successfully executed
all tests and 118 reported test failures.

\begin{center}
\fbox{
\begin{minipage}{8cm}
    \textit{Answering \numRQA{}:}~\emph{Given that
    \percentMedLongRunning{} of the projects are costly (\ie,
    \numMedLong{} projects from \medg{} and \longg{}), we concluded
    that time-consuming regression test suites are relatively frequent
    in open-source projects.}
\end{minipage}
}
\end{center}

\subsection{Answering research question \numRQB{}}
\label{sec:rqB}

\begin{itemize}
    \item \emph{\RQB}
\end{itemize}

\Jbc{pending updates! $\rightarrow$}
Section~\ref{sec:rqA} showed that medium and long-running projects are
not uncommon; they account to nearly \percentMedLongRunning{} of the
\numSubjs{} projects we analyzed.  However, parallelism by itself does
not assure speedups on those projects.  One sensible factor that is
important to the effectiveness of low-level parallelism is the
distribution of test costs in the test suite.  In the limit, if cost
is dominated by a single test, it is unlikely that parallelization
will be beneficial as a test method is the smallest working unit in
test frameworks (see Section~\ref{sec:frameworks}).

\begin{figure}[ht]
    \centering

    \begin{subfigure}{0.5\textwidth}
      \centering
      \includegraphics[width=\textwidth]{R/testcost-med.pdf}
      \caption{\label{fig:medtcost}Medium}
    \end{subfigure}\\
    
    \begin{subfigure}{0.5\textwidth}
      \centering
      \includegraphics[width=\textwidth]{R/testcost-long.pdf}
      \caption{\label{fig:longtcost}Long}
    \end{subfigure}%
    
    \begin{subfigure}{0.5\textwidth}
      \centering
      \begin{tabular}{rrrr}
        \toprule
        & $\sigma\leq1$ & $1<\sigma\leq5$ & $\sigma\ge5$ \\
        \midrule    
        Medium & 17 & 20 & 7 \\
        Long   &  9 & 14 & 5 \\
        \bottomrule
      \end{tabular}
      \caption{\label{fig:sd}...\Fix{Looks like there are missing
        projects...}}    
    \end{subfigure}%

    \caption{\label{fig:time-distributions}Distribution of time cost per project.}    
\end{figure}

\sloppy Figures~\ref{fig:medtcost} and~\ref{fig:longtcost} show the
time distribution of tests per project.  We observed that the average
median value of execution cost for a test was relatively small (dashed
horizontal red lines), namely 0.41s for \medg{} projects and 0.23s for
\longg{} projects.  Note that correlation between cost of test suites
and cost of their individual tests is not to be expected.  The
standard deviation ($\sigma$) associated with each ditribution was
relatively low.  Figure~\ref{fig:sd} shows the number of projects
within specific ranges of $\sigma$ values.  We also observed that
projects with very high standard deviations may or may not be
indicative of monopolization of execution by a small number of tests.
For example, the highest value of $\sigma$ ocurred in project
\CodeIn{wildfly-swarm}, whose test suite was classified as
long-running.  This project contains only 65 tests, 62 of which take
less than 0.5s to run, 1 of which takes nearly 3s to run, and 2 of
which take 40m to run.  This is an example where test suite
parallelization may not be very beneficial as $>$99\% execution cost
is dominated by only 3\% of the tests.  Without these two tests this
project would have been classified as short-running.  A closer
inspection in the data indicates that the project
\CodeIn{wildfly-swarm} was an extreme case; we did not find projects
with similar characteristics of time monopolization.  Project
\CodeIn{jenkinsci.gerrit-trigger-plugin} is also classified as
long-running and has the second highest value of $\sigma$.  For this
project, however, cost is distributed more smoothly across 529 tests,
of which 119 take more than 1s to run.  


\begin{center}
\fbox{
\begin{minipage}{8cm}
    \textit{Answering \numRQB{}:}~\emph{Overall, results indicate that
    projects with a very small number of tests monopolizing end-to-end
    execution time were rare.}
\end{minipage}
}
\end{center}

%% We are interested to know whether
%% most of the execution cost of a subject is dominated by a small subset
%% of test cases or if the cost is nearly equally distributed. 

%% We also evaluated the dispersion of time distributions (one
%% distribution per project) to answer research question \numRQB{}.  To
%% measure dispersion \emph{across} projects we used Relative Standard
%% Deviation (RSD)~\cite{everitt-book-stats-2010}.  Note that, if we were
%% to analyze each project in isolation, the standard deviation of a
%% distribution ($\sigma$) would suffice to quantify how dispersed the
%% (time) distribution is.  However, in our case, we would like to be
%% able to compare and summarize dispersion across projects.  The RSD,
%% which is obtained dividing the standard deviation by the mean ($\mu$)
%% of a distribution, provides such normalization effect.  This metric
%% provides a lower bound (zero) but not an upper bound (somewhere close
%% to 1).  The smaller (larger) the value of RSD the more (less) uniform
%% the distribution is.  Consequently, the lower the value of RSD the
%% more parallelizable a test suite should be.



%% \begin{figure}[h!]
%%   \centering
%%   \includegraphics[width=0.5\textwidth]{R/testcost.pdf}  
%%   \caption{\label{fig:relativesd}Distribution of RSD ($\sigma/\mu$)
%%     across projects.}
%% \end{figure}

%% Figure~\ref{fig:relativesd} shows the distribution of RSD across
%% medium and long-running projects.  Results show that the distribution
%% is skewed to the right indicating that test costs are relatively well
%% distributed in most costly projects we analyzed \Fix{$\leftarrow$
%%   confirm}.

%% analyzed the execution time
%% for the \numMedLong{} projects from the \longg{} and \medg{} groups
%% (see Section~\ref{sec:rqA}).
%% For each subject we calculated the
%% relative standard deviation of the test cases: we collected the
%% elapsed time of each individual test, calculated the standard
%% deviation, and divided by the mean. \Jbc{I need to clarify the
%%   relationship "well/bad-balanced" regression test and relative
%%   standard deviation}

%% Results indicated that \Fix{...elaborate...}. \Jbc{We may identify
%% different groups of subjects}\Fix{TODO: collect data + compute the
%% statistic, create a scatter plot to identify groups of subjects}

%% \Jbc{I shouldn't introduce parallelization arguments here but we have
%% to address this at some point:
%% Regression tests that are well distributed may benefit from
%% parallelism since more tests executes at the same time while the
%% opposite scenario may require a different approach. In the later
%% scenario, executing tests in parallel may have insignificant impact
%% since a small subset of test cases dominates the execution.}

\subsection{Answering research question \numRQC{}}
\label{sec:rqC}

\begin{itemize}
    \item \emph{\RQC}
\end{itemize}

To evaluate the presence of parallelization settings (see Section
\ref{sec:modes}), we considered all projects classified as \medg{} and
\longg{}-running, a total of \numMedLong{} subjects (see
Figure~\ref{fig:rq1-barplot}). To reduce noise we did not consider
projects with \shortg{}-running test suites as we observed that
\percentShortSequential{} of those projects do not use parallelism;
including them would favor sequential execution in our analysis.

To answer \numRQC{} we initially mined build files (\ie, \pomf{}) that
contain parallelization keywords, namely \CodeIn{forkMode},
\CodeIn{forkCount}, and \CodeIn{parallel}.  According to the Maven
surefire documentation\Fix{cite surefire doc}, any configuration to
run tests in parallel must contain one of these keywords. This mining
step, however, is not sound as a matching project may not indicate a
project with a valid configuration for running tests in parallel.
False positive can happen because of comments, for instance.  
To eliminate the cases of false positives and also to categorize true
positive cases, we complemented the initial mining step with a manual
inspection of files.


%% settings); the second step (inspection) consists in a manual
%% inspection to confirm the presence of parallelism settings in the
%% build file and classify them according to the parallelism level.
%% Figure \Fix{removed} describes the discovery step: we list the paths
%% of all build files and filter only the files that contain any of the

Figure~\ref{tab:inspection-table} summarizes our results. The first
column indicates the group of projects according to their time cost.
The second column indicates the number of build files per group.  The
third column is a subset from the previous column and represents the
number of build files found in the mining step (including
false-positives highlighted in parentheses). The last column indicates
the ratio of projects with parallelization settings.

From the \numMedLong{} subjects, we found \pomMedLong{} \pomf{} files
where \numPomMatched{} of those files match one of the keywords
mentioned above.  From these \numPomMatched{} build files, we
eliminated six false-positives: two were names of project modules
(\eg, \emph{camel-cookbook-parallel-processing}), one was part of a
comment, and three were commented configurations. The
\numPomMatchedValid{} build files with valid configurations are
distributed across \numProjectsPar{} projects from our sample.
\emph{From these results we found that $\sim$51\% of medium and
long-running projects do not use parallel features to run test
suites.}\Mar{please make it consistent with research
question}\Mar{explain this is over(under)-estimated...}

\begin{figure}[ht!]
    \centering
    \begin{tabular*}{0.48\textwidth}{@{\extracolsep{\fill}}cccc}
        \toprule
        \multirow{2}{*}{Group} %1st row, 1st cell
            & \multirow{2}{*}{\# \pomf{}}
            & \# \pomf{} matched
            & \# projects matched\\%
        % 2nd row - empty cell
            & % empty cell
            & (false-positives)
            & / total\\%
        \midrule%
        Long   & 1613 & 79 (1) & 19 / \numLong{}\\%
        Medium & 1238 & 30 (5) & 22 / \numMed{}\\%
        \midrule%
        Total % last row, first cell
            & \pomMedLong{}
            & \numPomMatched{} (6)
            & \numProjectsPar{} / \numMedLong{}\\%
        \bottomrule%
    \end{tabular*}
    \caption{Presence of parallelization settings in build files: the
    first column indicates the group of projects according to their
    time cost; the second column indicates the number of build files;
    the third column is the subset of files with parallelization
    keywords; the last column indicates the ratio of projects with
    parallelism support.}
    \label{tab:inspection-table} 
\end{figure}

\subsubsection{Distribution of parallel modes}

From the \numProjectsPar{} projects identified above, we investigated
further the \numPomMatchedValid{} build files with parallel settings.
We analyzed the support and distribution of parallel modes from this
subset of projects. To calculate the distribution of parallel modes,
we considered only the presence of the mode in at least one of the
project settings.  Recall that a build file may contain more than one
parallel setting and a project may contain several sub-modules with
build files.  In case the value of a parallel option is resolved
dynamically (\eg, via command-line argument or system variable) we
compute all modes related to the option. For instance, depending on
the value, the \CodeIn{parallel} option can be \Seq{} (\CodeIn{none}),
\ParClassSeqMeth{} (\CodeIn{classes}), \SeqClassParMeth{},
(\CodeIn{methods}), and \ParClassParMeth{} (\CodeIn{all}).
Figure~\ref{fig:freqmodes} summarizes our findings. \Jbc{Given that we
checked each configuration, we could elaborate how these settings are
used.} \Fix{Missing conclusion: Fork the most used configuration}

\begin{figure}[h!]
    \centering
    \includegraphics[width=0.32\textwidth]{plots/parallel-modes.pdf}
    \caption{\label{fig:freqmodes}Distribution of parallel modes
    identified in a subset of \numProjectsPar{} projects (unused modes
    omitted). A single project may have support to more than one
    parallel mode.}
\end{figure}

\subsection{Answering research question \numRQD{}}
\label{sec:rqD}

\begin{itemize}
    \item \emph{\RQD}
\end{itemize}

To answer \numRQD{}, we considered the \numProjectsPar{} subjects with
parallelization settings identified in \numRQC{} and compared their
time cost with sequential execution. \Jbc{should have a sentence to
motivate this RQ} 
\Jbc{explain methodology}.

\Fix{---------------------}

%%To evaluate the distribution of execution time per project, we sorted
%%the test cases by decreasing order of elapsed time and calculated the
%%number of tests executed in 90\% of the total time. Later, we reported
%%the \Fix{balance} of execution time by dividing the number of tests
%%that represents 90\% of the execution time by the number of tests
%%cases. For instance, a balance of 50\% indicates \Fix{...}.  \Fix{We
%%collected the elapsed time from test cases for each generated report.
%%Maven Surefire generates an XML report with execution information
%%(\eg, number of skipped tests and elapsed time) per test suite
%%\Jbc{Should I use the previous sentence as a footnote or should I
%%delete it?}. We noticed that some test cases reported an elapsed time
%%of zero: since the reported time is in milliseconds, some tests may
%%execute in a shorter time. \Fix{..to be continued...}}. Results
%%indicate that \Fix{...}.
%%
%%\begin{figure}[h!]
%%    \centering
%%    \includegraphics[width=0.4\textwidth]{results/plots/balance.pdf}
%%    \caption{\Fix{balance}}
%%\end{figure}

%% \subsection{Answering research question RQ3}
%% \label{sec:rqThree}
%% 
%% \begin{itemize}
%%     \item \RQB
%% \end{itemize}
%% 
%% To evaluate the distribution of CPU and IO intensive test suites from
%% the sample set, we used the command \emph{sar} to monitor the system
%% activity in background while tests ran. \emph{Sar} is a command that
%% collects and reports statistics (\eg, percentage of IO waiting and
%% usage of CPU in user mode) based on the kernel activity and it is
%% highly configurable to collect detailed information (\eg, usage of a
%% specific processor core or percentage of network interface
%% utilization). We configured \emph{sar} to report \Fix{...explain how
%% we executed and what fields we are interested}. \Fix{explain fields}.
%% Figure \Fix{A} shows the distribution of subjects grouped in intervals
%% of \Fix{W}\% of CPU utilization. Results indicates that \Fix{...}
%% 
%% \begin{figure}[h!]
%%     \centering
%%     \includegraphics[width=0.4\textwidth]{results/plots/cpuness.pdf}
%%     \caption{\Fix{cpu usage}}
%% \end{figure}
%% 
%% \Comment{we proposed the definition of \emph{cpuness}
%% computed as the follow: $((user\_t + system\_t) / elapsed\_t) * 100$,
%% where \emph{user\_t} is the elapsed time of execution in \emph{user
%% mode}, \emph{system\_t} is the elapsed in \emph{kernel mode}, and
%% \emph{elapsed\_t} is the elapsed time to finish the execution. We
%% measured the \emph{cpuness} of each regression test \Fix{...elaborate
%% the meaning of cpuness} \Fix{Describe how I measured user, system and
%% "wall" time}.  \Fix{Explain results}.  \Fix{show plots}}
%% 

\section{Discussion}

\vspace{1ex}\noindent{}\textbf{The Good.}\Fix{...}

\vspace{1ex}\noindent{}\textbf{The Bad.}\Fix{...}

\vspace{1ex}\noindent{}\textbf{The Ugly.}\Fix{...}

\section{Threats to Validity}

\Fix{To appear... Elaborate CONSTRUCT, INTERNAL and EXTERNAL threats.
I'm going to be writing this ASAP as I finish to present the results
from RQ6}

\chapter{Related Work}
\label{sec:related}

Regression testing research has focused mostly on test suite
minimization, prioritization, reduction, and
selection~\cite{yoo-harman-stvr2012,soetens-etal-2016}.  Most of these
techniques are unsound (\ie{}, they do not guarantee that
fault-revealing tests will be considered for testing).  The test
selection technique
Ekstazi~\cite{gligoric-etal-issta2015,celik-etal-fse2017} is an
example of a sound regression testing technique. It conservatively
computes which tests have been impacted by file changes.  A test is
discarded for execution if it does not depend on any changed file
dynamically reachable from execution.\Comment{ Curiously Ekstazi's
  evaluation discovered subjects with parallelism enabled by
  default.}\Comment{ \c{C}elik~\etal{}~\cite{} recently extended
  Ekstazi to track files accessed outside JVM boundaries.} Important
to note that regression testing techniques, including test selection,
is complementary to test suite parallelization.

ElectricTest~\cite{bell-etal-esecfse2015} is a tool for efficiently
detecting data dependencies across test cases.  Dependency tracking is
important as to avoid test flakiness when parallelizing test
suites. ElectricTest observes reads and writes on global resources
made by tests to identify these dependencies at low cost. We remain to
investigate the impact of ElectricTest to reduce flakiness in
unrestricted test suite parallelization.

The use of the Single Instruction Multiple Data (SIMD) design has been
previously explored in research to accelerate test
execution~\cite{damorim-etal-issta2007,damorim-etal-tse2008,kim-etal-issre2012,nguyen-etal-icse2014,rajan-etal-ase2014,sen-etal-fse2015,yaneva-etal-issta2017}. The
SIMD architecture, as implemented in modern GPUs, for instance, allows
the execution of a given instruction simultaneously against multiple
data.  For that reason, in principle, one test could be ran
simultaneously against multiple inputs provided that multiple test
inputs exist associated to that one test.  Recent
work~\cite{rajan-etal-ase2014,yaneva-etal-issta2017} explored that
idea to speedup test execution of embedded software using graphic
cards. Although benchmarks indicate superior performance compared to
traditional multicore CPUs, the use of the technology in broader
settings is limited. For example, execution of more general programs
can violate the SIMD's lock-step assumption on the control-flow of
threads.  This violation would affect negatively performance.
Furthermore, handling complex data is challenging in
SIMD~\cite{damorim-etal-issta2007,damorim-etal-tse2008}.  The approach
is promising when multiple input vectors exist for each test and the
testing code heavily manipulates scalar data types.  The datasets used
in those papers satisfied those constraints.

Google~\cite{google-tap,google-ci} and
Microsoft~\cite{prasad-shulte-ieee-microsoft-ci} have been creating
distributed infrastructures to efficiently build massive amounts of
code and run massive amounts of tests.  Those scenarios bring
different and challenging problems such as deciding when to trigger
the build under multiple file
updates~\cite{memon-etal-icse17}. Although such distributed systems
are targeted to extremely large scale code and test bases, the same
ideas can be applied to handle the build process of large, albeit not
as large, projects.  For example,
Gambi~\etal{}~\cite{gambi-etal-issta2017} recently proposed CUT, a
tool to automatically parallelize JUnit tests on the cloud. The tool allows
the developer to control resource allocation and deal with the project
specific test dependencies.  Note that test suite parallelization is
complementary to these high-level parallelism schemes.

Continuous Integration (CI) services, such as Travis CI~\cite{travis},
are becoming widely used in the open-source
community~\cite{hilton-etal-ase2016,vasilescu-etal-fse2015}. Accelerating
time to run tests in CI is important as to reduce the period between
test report updates.  Module-level regression
testing~\cite{vasic-etal-fse2017}, for example, can be helpful in that
setting. It is important to note that test failures are more common in
CI compared to an overnight run or a local run, for instance.  This
can happen because of semantic merge conflicts~\cite{brun-etal-fse11},
for instance.  As such effect can impact developer's perception and
tolerance towards failures, we are curious to know if developers would
be willing to receive more frequent test reports at the expense of
potentially increasing failure rates due to flakiness caused by
parallelism.

%%  LocalWords:  Ekstazi Ekstazi's elik JVM parallelization SIMD GPUs
%%  LocalWords:  ElectricTest parallelizing multicore CPUs SIMD's GCC
%%  LocalWords:  datasets Gambi parallelize JUnit

\section{Conclusions}

Testing is expensive.  Despite all advances in regression testing
research, dealing with high testing costs remains an important problem
in Software Engineering.  This paper reports our findings on the usage
and impact of test execution parallelization in open-source projects.
Multicore CPUs are widely available today.  Testing frameworks and
build systems that capitalize on these machines also became popular.
Despite some resistance observed from practitioners, our results
suggest that parallelization can be used in many cases without
sacrificing reliability. More research needs to be done to improve
automation (\eg{}, breaking test dependencies, refactoring test
suites, enforcing safe test schedules) as to safely optimize parallel
execution.  The artifacts we produced as result of this study are
available from the following web page:

\begin{center}
  \webpage{}
\end{center}  


%% Overall, this study brings to light the benefits and burdens of test
%% suite parallelization to improve test efficiency. It provides
%% recommendations to practicioners and developers of new techniques and
%% tools aiming to speed up test execution with parallelization.

%% Considering a set of \numSubjs{} popular Java projects we analyzed, we
%% found that \percentMedLongRunning{} of the projects contain costly
%% test suites but parallelization features still seem underutilized in
%% practice~---~only \percentParallelUpdated{} of costly projects use
%% parallelization.  The main reported reason for adoption resistance was
%% the concern to deal with concurrency issues.  Results suggest that, on
%% average, developers prefer high predictability than high performance
%% in running tests.





%%  LocalWords:  parallelization Multicore CPUs refactoring


%% uncomment for camera-ready
%% % use section* for acknowledgment
\ifCLASSOPTIONcompsoc
% The Computer Society usually uses the plural form
\section*{Acknowledgments}
\else
% regular IEEE prefers the singular form
\section*{Acknowledgment}
\fi

Jean(derson) and Luis are supported by FACEPE scholarphips
IBPG-0632-1.03/15 and IBPG-1175-1.03/16, respectively.  This work was
partially supported by CNPq grants 457756/2014-4 and 203981/2014-6.
%\newpage
% trigger a \newpage just before the given reference
% number - used to balance the columns on the last page
% adjust value as needed - may need to be readjusted if
% the document is modified later
%\IEEEtriggeratref{8}
% The "triggered" command can be changed if desired:
%\IEEEtriggercmd{\enlargethispage{-5in}}

% references section

% can use a bibliography generated by BibTeX as a .bbl file
% BibTeX documentation can be easily obtained at:
% http://mirror.ctan.org/biblio/bibtex/contrib/doc/
% The IEEEtran BibTeX style support page is at:
% http://www.michaelshell.org/tex/ieeetran/bibtex/
%\bibliographystyle{IEEEtran}
% argument is your BibTeX string definitions and bibliography database(s)
%\bibliography{IEEEabrv,../bib/paper}
%
% <OR> manually copy in the resultant .bbl file
% set second argument of \begin to the number of references
% (used to reserve space for the reference number labels box)

\balance
\bibliographystyle{plain}
\bibliography{tmp}

% that's all folks
\end{document}



%%  LocalWords:  Parallelization multi parallelization tmp tradeoffs
