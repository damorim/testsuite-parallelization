\section{Related Work}
\label{sec:related}
Researchers and practitioners have been shedding light to the to the
demand of techniques for optimizing test execution. For instance, a
recent paper from the automotive industry reported that test suites
can take days to run~\cite{artl-etal-icst2015}.  There are different
aspects to consider when dealing with the challenge of reducing
test cost. 

\textbf{Test selection vs. parallelism:}
Research has focused mostly on test suite minimization,
prioritization, reduction, and selection~\cite{yoo-harman-stvr2012}.
Most of these techniques are unsound (\ie{}, they do not guarantee
that fault-revealing tests will be selected), however, more recently,
sound techniques have been
proposed~\cite{gligoric-etal-issta2015,soetens-etal-2016}.  For
example, to soundly select tests for execution,
Ekstazi~\cite{ekstazi-web,gligoric-etal-issta2015} conservatively
computes which tests have been impacted by changes.  A test is
discarded for execution if it does not depend on any changed file
dynamically reachable from execution. Although Ekstazi is focused on
regression test selection, it highlights the importance of parallelism
as a complement to test selection.  In fact, as in their evaluation, we
also discovered subjects with parallelism enabled by default.  Test
suite parallelization is complementary to regression testing
techniques.

\textbf{Usage of massively parallel hardware:}
Previous research investigated the feasibility of testing embedded
software on \emph{Graphics Processing Units} (GPUs) as an alternative
to multicore CPUs~\cite{rajan-etal-ase2014}.
The \emph{Simple Instruction Multiple Data} (SIMD) architecture of
modern GPUs allows to run the same test case with different test data
over several threads.
Although benchmarks indicate far superior performance over traditional
multicore CPUs, existing GPU programming-models (\eg,
CUDA~\cite{cuda}, and OpenCL~\cite{opencl}) are compatible only with a
subset of the C/C++ language.
Recent work proposed a compiler-assisted approach to make programs
compatible to the relying programming
model~\cite{yaneva-etal-issta2017}; however, \Jbc{Justify they are out
of scope for our subjects...}.

\Fix{Please:
  (i) consider categorizing related work in sections  
  (ii) list papers that need to be discussed
  (iii) discuss papers that use multi-execution (SIMD cpus like GPUs)
  to speedup testing
  }

\Comment{
Running
tests incompatible complex test suites on GPUs remains a challenge.
\Comment{
Similar approach to regression test selection has been recently
applied in the context of the GCC~\cite{gcc} configurable
system~\cite{souto-damorim-jss}.}
The work of Luo et al~\cite{luo-etal-fse2014} reports and empirical
analysis on test flakiness. \Jbc{se relaciona com o nosso no
aspecto de falar dentre as demais categorias, sobre flakiness que se
manifestam em situacoes de concorrencias. Este trabalho eh
complementar ao nosso, uma vez que ele elabora as causas e solucoes
para algumas falhas que se manifestaram em concorrencia.
concorrencias.
O Trabalho de Hilton et al~\cite{hilton-etal-ase2016}, embora tenha
como objetivo entender o uso de CI, mostra que existe uma demanda em
acelerar o build e execucao de testes em servidores de integracao
    continua
}
}
