\section{Related Work}
\label{sec:related}
We discuss most related work in the following.
%% Researchers and practitioners have been shedding light to the demand
%% of techniques for optimizing test execution.

%% For instance, a recent paper from the automotive industry reported
%% that test suites can take days to run~\cite{artl-etal-icst2015}.
%% There are different aspects to consider when dealing with the
%% challenge of reducing test cost.

%ekstazi-web,
Regression testing research has focused mostly on test suite
minimization, prioritization, reduction, and
selection~\cite{yoo-harman-stvr2012,soetens-etal-2016}.  Most of these
techniques are unsoud (\ie{}, they do not guarantee that
fault-revealing tests will be considered for testing).  The test
selection technique
Ekstazi~\cite{gligoric-etal-issta2015,celik-etal-fse2017} is an
example of a sound regression testing technique. It conservatively
computes which tests have been impacted by file changes.  A test is
discarded for execution if it does not depend on any changed file
dynamically reachable from execution.\Comment{ Curiously Ekstazi's
  evaluation discovered subjects with parallelism enabled by
  default.}\Comment{ \c{C}elik~\etal{}~\cite{} recently extended
  Ekstazi to track files accessed outside JVM boundaries.} Important
to note that regression testing techniques, including test selection,
is complementary to test suite parallelization.

%\Luis{maybe we should create another division for test acceleration}

ElectricTest~\cite{bell-etal-esecfse2015} is a tool for efficiently
detecting data dependencies across test cases.  Dependency tracking is
important as to avoid test flakiness when parallelizing test
suites. ElectricTest observes reads and writes on global resources
made by tests to identify these dependencies at low cost. We remain to
investigate the impact of ElectricTest to reduce flakiness in
unrestricted test suite parallelization.

%% \textit{Usage of massively parallel hardware:}
%\textit{Hardware level massive parallelization:}

The use of the Single Instruction Multiple Data (SIMD) design has been
previoulsy explored in research to accelerate test
execution~\cite{damorim-etal-issta2007,damorim-etal-tse2008,kim-etal-issre2012,nguyen-etal-icse2014,rajan-etal-ase2014,sen-etal-fse2015,yaneva-etal-issta2017}. The
SIMD architecture, as implemented in modern GPUs, for instance, allows
the execution of a given instruction simultaneously against multiple
data.  For that reason, in principle, one test could be ran
simulteneously against multiple inputs provided that multiple test
inputs exist associated to that one test.  Recent
work~\cite{rajan-etal-ase2014,yaneva-etal-issta2017} explored that
idea to speedup test execution of embedded software using graphic
cards. Although benchmarks indicate superior performance compared to
traditional multicore CPUs, the use of the technology in broader
settings is limited. For example, execution of more general programs
can violate the SIMD's lock-step assumption on the control-flow of
threads.  This violation would affect negatively performance.
Furthermore, handling complex data is challenging in
SIMD~\cite{damorim-etal-issta2007,damorim-etal-tse2008}.  The approach
is promising when multiple input vectors exist for each test and the
testing code heavily manipulates scalar data types.  The datasets used
in those papers satisfied those constraints.


%% existing GPU programming-models (\eg{}, CUDA~\cite{cuda}, and
%% OpenCL~\cite{opencl}) are compatible only with a subset of the C/C++
%% language.
%% Recent work proposed a compiler-assisted approach to make programs
%% compatible with the relying programming-model used by
%% GPUs to parallelize tests~\cite{}.
%% \Jbc{...Justify they are out of scope for our subjects...}.

\Mar{can you complete this AND add text on high-level paralellism?}

Continuous Integration Services are widely used by the open-source
community~\cite{hilton-etal-ase2016,vasilescu-etal-fse2015}. Travis
CI~\cite{travis} is an example of one of these services. Recent work
by Hilton~\etal{}~\cite{hilton-etal-ase2016} showed that there is a
high acceptance rate for CI services. The average build time for those
projects using CI is under $\sim$8.3 min~\cite{hilton-etal-ase2016}. 

%% \Jbc{Missing work about testing in the cloud and Jon Bell}

\Comment{
\Fix{Please:
  (i) consider categorizing related work in sections  
  (ii) list papers that need to be discussed
  (iii) discuss papers that use multi-execution (SIMD cpus like GPUs)
  to speedup testing
  }
}
\Comment{
Similar approach to regression test selection has been recently
applied in the context of the GCC~\cite{gcc} configurable
system~\cite{souto-damorim-jss}.
O Trabalho de Hilton et al~\cite{hilton-etal-ase2016}, embora tenha
como objetivo entender o uso de CI, mostra que existe uma demanda em
acelerar o build e execucao de testes em servidores de integracao
    continua
}
