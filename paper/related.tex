\section{Related Work}
\label{sec:related}
Researchers and practitioners have been shedding light to the demand
of techniques for optimizing test execution.

%% For instance, a recent paper from the automotive industry reported
%% that test suites can take days to run~\cite{artl-etal-icst2015}.
%% There are different aspects to consider when dealing with the
%% challenge of reducing test cost.

\textit{Test selection:} Research has focused mostly on test suite
minimization, prioritization, reduction, and
selection~\cite{yoo-harman-stvr2012}.  \Fix{...} Most of these
techniques are unsoud (\ie{}, they do not guarantee that
fault-revealing tests will be selected), however, more recently, sound
techniques have been
proposed~\cite{gligoric-etal-issta2015,soetens-etal-2016}. Recent work
for test selection introduced
Ekstazi~\cite{ekstazi-web,gligoric-etal-issta2015} that conservatively
computes which tests have been impacted by file changes.  A test is
discarded for execution if it does not depend on any changed file
dynamically reachable from execution. Although Ekstazi is focused on
regression test selection, it highlights the importance of parallelism
as a complement to test selection.  In fact, as in their evaluation,
we also discovered subjects with parallelism enabled by default.
Test suite parallelization is complementary to regression testing
techniques.
\Luis{maybe we should create another division for test acceleration}
Bell et al.\Fix{ref} introduced ElectricTest, a tool for detecting
data dependencies between real world tests. A technique was introduced
to observe global resources read and write between tests, instead of
executing every possible permutation of all tests to detect data
dependencies. Dependency detection can be combined with test suite
parallelization to efficiently accelerate test execution.

%% \textit{Usage of massively parallel hardware:}
\textit{Hardware level massive parallelization:} Previous research
investigated the feasibility of testing embedded software on Graphics
Processing Units (GPUs) as an alternative to multicore
CPUs~\cite{rajan-etal-ase2014}.  The Simple Instruction Multiple Data
(SIMD) architecture of modern GPUs allows running the same test case
with different test data over several threads.  Although benchmarks
indicate far superior performance over traditional multicore CPUs,
there are severe limitations. Control-flow branching in source code
can impact performance negatively while violating the lock-step
execution during the divergence of instructions~\cite{rajan-etal-ase2014}
%% existing GPU programming-models (\eg{}, CUDA~\cite{cuda}, and
%% OpenCL~\cite{opencl}) are compatible only with a subset of the C/C++
%% language.
Recent work proposed a compiler-assisted approach to make programs
compatible with the relying programming-model used by
GPUs to parallelize tests~\cite{yaneva-etal-issta2017}.
\Jbc{...Justify they are out of scope for our subjects...}.

\textit{Continuous integration services:}
Recent work showed us that Continuous Integration Services are widely
used and improve the productivity~\Fix{ref ref ref}. 

%% \Jbc{Missing work about testing in the cloud and Jon Bell}

\Comment{
\Fix{Please:
  (i) consider categorizing related work in sections  
  (ii) list papers that need to be discussed
  (iii) discuss papers that use multi-execution (SIMD cpus like GPUs)
  to speedup testing
  }
}
\Comment{
Similar approach to regression test selection has been recently
applied in the context of the GCC~\cite{gcc} configurable
system~\cite{souto-damorim-jss}.
O Trabalho de Hilton et al~\cite{hilton-etal-ase2016}, embora tenha
como objetivo entender o uso de CI, mostra que existe uma demanda em
acelerar o build e execucao de testes em servidores de integracao
    continua
}
