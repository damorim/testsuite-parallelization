\section{Related Work}
\label{sec:related}

Regression testing research has focused mostly on test suite
minimization, prioritization, reduction, and
selection~\cite{yoo-harman-stvr2012,soetens-etal-2016}.  Most of these
techniques are unsound (\ie{}, they do not guarantee that
fault-revealing tests will be considered for testing).  The test
selection technique
Ekstazi~\cite{gligoric-etal-issta2015,celik-etal-fse2017} is an
example of a sound regression testing technique. It conservatively
computes which tests have been impacted by file changes.  A test is
discarded for execution if it does not depend on any changed file
dynamically reachable from execution.\Comment{ Curiously Ekstazi's
  evaluation discovered subjects with parallelism enabled by
  default.}\Comment{ \c{C}elik~\etal{}~\cite{} recently extended
  Ekstazi to track files accessed outside JVM boundaries.} Important
to note that regression testing techniques, including test selection,
is complementary to test suite parallelization.

ElectricTest~\cite{bell-etal-esecfse2015} is a tool for efficiently
detecting data dependencies across test cases.  Dependency tracking is
important as to avoid test flakiness when parallelizing test
suites. ElectricTest observes reads and writes on global resources
made by tests to identify these dependencies at low cost. We remain to
investigate the impact of ElectricTest to reduce flakiness in
unrestricted test suite parallelization.

The use of the Single Instruction Multiple Data (SIMD) design has been
previously explored in research to accelerate test
execution~\cite{damorim-etal-issta2007,damorim-etal-tse2008,kim-etal-issre2012,nguyen-etal-icse2014,rajan-etal-ase2014,sen-etal-fse2015,yaneva-etal-issta2017}. The
SIMD architecture, as implemented in modern GPUs, for instance, allows
the execution of a given instruction simultaneously against multiple
data.  For that reason, in principle, one test could be ran
simultaneously against multiple inputs provided that multiple test
inputs exist associated to that one test.  Recent
work~\cite{rajan-etal-ase2014,yaneva-etal-issta2017} explored that
idea to speedup test execution of embedded software using graphic
cards. Although benchmarks indicate superior performance compared to
traditional multicore CPUs, the use of the technology in broader
settings is limited. For example, execution of more general programs
can violate the SIMD's lock-step assumption on the control-flow of
threads.  This violation would affect negatively performance.
Furthermore, handling complex data is challenging in
SIMD~\cite{damorim-etal-issta2007,damorim-etal-tse2008}.  The approach
is promising when multiple input vectors exist for each test and the
testing code heavily manipulates scalar data types.  The datasets used
in those papers satisfied those constraints.

Google~\cite{google-tap,google-ci} and
Microsoft~\cite{prasad-shulte-ieee-microsoft-ci} have been creating
distributed infrastructures to efficiently build massive amounts of
code and run massive amounts of tests.  Those scenarios bring
different and challenging problems such as deciding when to trigger
the build under multiple file
updates~\cite{memon-etal-icse17}. Although such distributed systems
are targeted to extremely large scale code and test bases, the same
ideas can be applied to handle the build process of large, albeit not
as large, projects.  For example,
Gambi~\etal{}~\cite{gambi-etal-issta2017} recently proposed CUT, a
tool to automatically parallelize JUnit tests on the cloud. The tool allows
the developer to control resource allocation and deal with the project
specific test dependencies.  Note that test suite parallelization is
complementary to these high-level parallelism schemes.

Continuous Integration (CI) services, such as Travis CI~\cite{travis},
are becoming widely used in the open-source
community~\cite{hilton-etal-ase2016,vasilescu-etal-fse2015}. Accelerating
time to run tests in CI is important as to reduce the period between
test report updates.  Module-level regression
testing~\cite{vasic-etal-fse2017}, for example, can be helpful in that
setting. It is important to note that test failures are more common in
CI compared to an overnight run or a local run, for instance.  This
can happen because of semantic merge conflicts~\cite{brun-etal-fse11},
for instance.  As such effect can impact developer's perception and
tolerance towards failures, we are curious to know if developers would
be willing to receive more frequent test reports at the expense of
potentially increasing failure rates due to flakiness caused by
parallelism.

%%  LocalWords:  Ekstazi Ekstazi's elik JVM parallelization SIMD GPUs
%%  LocalWords:  ElectricTest parallelizing multicore CPUs SIMD's GCC
%%  LocalWords:  datasets Gambi parallelize JUnit
