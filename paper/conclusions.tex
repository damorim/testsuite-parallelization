\section{Conclusions}

Testing is expensive.  Despite all advances in regression testing
research, dealing with high testing costs remains an important problem
in Software Engineering.  This paper reports our findings on the usage
and impact of test execution parallelization in open-source projects.
Multicore CPUs, as well as testing frameworks and build systems to
capitalize on them, are widely available today.  Despite some
resistance observed from practicioners, our results suggest that
parallelization can be done in many cases without sacrificing
reliability and more research needs to be done to improve automation
(\eg{}, breaking test dependencies and refactoring test suites) as to
safely optimize parallel execution.  The artifacts we produced as
result of this study are available from the following web
page \webpage{}.


%% Overall, this study brings to light the benefits and burdens of test
%% suite parallelization to improve test efficiency. It provides
%% recommendations to practicioners and developers of new techniques and
%% tools aiming to speed up test execution with parallelization.

%% Considering a set of \numSubjs{} popular Java projects we analyzed, we
%% found that \percentMedLongRunning{} of the projects contain costly
%% test suites but parallelization features still seem underutilized in
%% practice~---~only \percentParallelUpdated{} of costly projects use
%% parallelization.  The main reported reason for adoption resistance was
%% the concern to deal with concurrency issues.  Results suggest that, on
%% average, developers prefer high predictability than high performance
%% in running tests.




