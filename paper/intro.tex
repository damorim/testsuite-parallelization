\section{Introduction}

%% . which can disrupt the development
%% process~\cite{hilton-etal-ase2016}.\Comment{  This is often referred to as the
%% ``regression testing problem''.}

As software evolves it is expected that the number of tests and the
length of test runs increase.  Both elements can add up to the
aggregate cost of running a test suite, which can lead to late reports
of test failures.  At large IT organizations, dealing with high
testing costs is a common problem.  The Google TAP
system~\cite{google-tap,google-ci} and the Microsoft CloudBuild
system~\cite{prasad-shulte-ieee-microsoft-ci} provide evidence about
the importance of speeding up test execution in industry.  This
problem is not specific for giants like Microsoft and Google; it also
occur in other domains.  For example, a recent paper from the
automotive industry reported that test suites can take days to
run~\cite{artl-etal-icst2015}.

%% papers also indicate the important of speeding up the testing process
%% in other industrial domain, e.g., in 
%% \Mar{Ask Milos if he can share stories on how prevalent regression
%%   testing is at Microsoft and Google, if he can explain how they run
%%   tests, and how prevalent is parallelization (in contrast to
%%   regression testing).}

Several approaches have been proposed in the literature to optimize
regression testing.  Research has focused mostly on test suite
minimization, prioritization, reduction, and
selection~\cite{yoo-harman-stvr2012}.  Although regression testing is
a recognized problem in academia and at large organizations, to the
best of our knowledge no prior work has investigated how prevalent the
problem is in open-source projects.  In large organizations, the
alternative of renting a cloud
service\footnote{\url{https://clutch.co/cloud}}, or even building a
server farm (for increased security), is a potential escape to the
regression testing problem.  Considering open-source projects and
lower-budget projects, the alternative of building or using high-end
infrastructures is not as attractive but it is unclear how relevant
regression testing is in this scenario.

%% eventually move this to macros.tex
\newcommand{\numSubjs}{143}
\newcommand{\numLongRunning}{\Fix{17\%}}
\newcommand{\percentParallelForLongRunning}{\Fix{80\%}}

Motivated by this potential gap in regression testing, we studied (1)
how prevalent long-running test suites are in open-source software
projects.  To conduct this experiment we selected \numSubjs{} Java
projects from github that are highly-popular and use the Maven build
system.  (Section~\ref{sec:eval} details our selection methodology.)
Our experimental results indicate that long-running test suites
(taking longer than 5m to complete\Fix{check if Ekstazi paper also
  used this value to indicate long runs and cite}) are not
rare~--~\numLongRunning{} of the projects we analyzed contain
long-running test suites.  \Mar{missing connection. why parallelism?}
From this result we investigated (2) how frequently the projects we
considered use parallelization to speedup test runs.  Our results
indicate that \percentParallelForLongRunning{} of long-running test
suites do use parallelism to speedup execution.  Given the popularity
of this solution, we further studied: (3) what are the causes of high
execution cost in individual tests (CPU vs. IO) and (4) how uniformly
cost is distributed across tests.  These questions help to assess the
impact of parallelization in test suite execution.  Intuitively,
IO-intensive test suites and test suites with uneven distribution of
cost per test are factors that limit the benefits of parallelization.
Results indicate that \Fix{...}  Finally, we investigated (5) what are
the limitations related to parallelization that could justify adoption
resistance.

\Fix{--------------}

To illustrate the importance of parallel test execution, \Jbc{TODO -
Write a paragraph motivating this paper.}

This paper reports an empirical study to evaluate the impact of
parallelization options on test execution speedup \Fix{...}\Jbc{TODO -
Summary of the contributions of this paper} .

%%  LocalWords:  parallelization multicore JUnit TestNG NUnit XXm YYm
%%  LocalWords:  Groupon parallelizing multi JVMs CPUs JVM Milos TODO
%%  LocalWords:  Ekstazi github CloudBuild
