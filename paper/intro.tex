\section{Introduction}

%% . which can disrupt the development
%% process~\cite{hilton-etal-ase2016}.\Comment{  This is often referred to as the
%% ``regression testing problem''.}

As software evolves it is expected that the number of tests and the
length of test runs increase.  Both elements can add up to the
aggregate cost of running a test suite, which can lead to late reports of test
failures.  At large IT companies (e.g.,
Groupon~\cite{kim-etal-fse2013}), dealing with high testing costs is a
common problem .  \Mar{Ask Milos if he can share stories on how
  prevalent regression testing is at Microsoft and Google, if he can
  explain how they run tests, and how prevalent is parallelization (in
  contrast to regression testing).}

Several approaches have been proposed in the literature to optimize
regression testing.  Research has focused mostly on test suite
minimization, prioritization, reduction, and
selection~\cite{yoo-harman-stvr2012}.  Most of these techniques are
unsound (\ie{}, they do not guarantee that fault-revealing tests will
be selected), however, more recently, sound techniques have been
proposed~\cite{gligoric-etal-issta2015,soetens-etal-2016}.  For
example, to soundly select tests for execution,
Ekstazi\cite{ekstazi-web,gligoric-etal-issta2015} conservatively
computes which tests have been impacted by changes.  It analyzes, for
each test, which file dependencies have been change impacted.

%; these
%dependencies are computed and maintained at low cost.

Although regression testing is a recognized problem in academia and at
large IT organizations, to the best of our knowledge no prior work has
investigated how prevalent the problem is in open-source projects.
For large IT organizations, the alternative of renting server farms,
or even building one, is a potential escape to the regression testing
problem.  However, for open-source projects and for projects of
smaller organizations with lower budgets for building testing
infrastructures, this alternative may not be as attractive.

\newcommand{\numSubjs}{143}

Motivated by this potential gap in regression testing practice, this
paper studies (1) how prevalent long-running test suites are in
open-source software projects.  To conduct this experiment we selected
\numSubjs{} Java projects from github that are highly-popular and use
the Maven build system.  (Section~\ref{sec:eval} details our selection
criteria.)  Our experimental results indicate that long-running test
suites (taking longer than 5m\Fix{check if Ekstazi paper also used
  this value and cite} to complete) are not rare (\Fix{17\%} of the
projects).  From this result we investigated (2) how frequently
long-running test suites use parallelization.  Our results indicate
that \Fix{80\%} of the long-running projects do use parallelization to
speedup execution.  Given the popularity of this solution, we further
studied: (3) what are the causes of high execution cost in individual
tests (CPU vs. IO) and (4) how uniformly cost is distributed across
tests.  These questions help to assess the impact of parallelization
in each of these projects.  Intuitively, IO-intensive test suites and
test suites with uneven distribution of cost per test are factors that
limit the effects of parallelization.  Results indicate that \Fix{...}
Finally, we investigated (5) what are the limitations related with
parallelization that could justify adoption resistance.

\Fix{--------------}

To illustrate the importance of parallel test execution, \Jbc{TODO -
Write a paragraph motivating this paper.}

This paper reports an empirical study to evaluate the impact of
parallelization options on test execution speedup \Fix{...}\Jbc{TODO -
Summary of the contributions of this paper} .

%%  LocalWords:  parallelization multicore JUnit TestNG NUnit XXm YYm
%%  LocalWords:  Groupon parallelizing multi JVMs CPUs JVM Milos TODO
%%  LocalWords:  Ekstazi github
