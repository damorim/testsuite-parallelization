\section{Introduction}

%% . which can disrupt the development
%% process~\cite{hilton-etal-ase2016}.\Comment{  This is often referred to as the
%% ``regression testing problem''.}

As software evolves it is expected that the number of tests and the
length of test runs increase.  Both components can add up to the
aggregate cost of running a test suite.  Dealing with high testing
costs is an important problem in software engineering research and
industrial practice.

Several approaches have been proposed in the research literature to
address the regression testing problem, with the focus mainly on test
suite minimization, prioritization, reduction, and
selection~\cite{yoo-harman-stvr2012}.  In industry, the focus has been
mainly on distributing the testing workload.  Evidence of this are the
Google TAP system~\cite{google-tap,google-ci} and the Microsoft
CloudBuild system~\cite{prasad-shulte-ieee-microsoft-ci}, which
provide distributed infrastructures to efficiently build massive
amounts of code and run tests.  Building server clusters is also a
popular mechanism to distribute testing workloads.  For example, as of
August 2013, the Groupon PWA system, which powers the
\url{groupon.com} website, included over 19K tests.  To run all those
tests under 10m, Groupon used a cluster of 4 computers with 24 cores
each~\cite{kim-etal-fse2013}.

At large organizations, the alternative of renting cloud
services\footnote{List of popular cloud services:
  \url{https://clutch.co/cloud}} or even building proprietary
infrastructures for running tests is a legitimate approach to mitigate
the regression testing problem.  However, for projects with low
budgets and yet relatively heavy testing workloads, this solution may
not be economically viable.  For these cases, the use of commodity
hardware, perhaps existing hardware, emerges as an attractive solution
for running tests.  The proliferation of multi-core CPUs and the
increasing popularization of testing frameworks with a rich support
for parallelization enables test suite speedups through increased CPU
usage~(see Section~\ref{sec:modes}).  This paper reports on an
empirical study we conducted to analyze the impact of low-level
parallelization to speedup testing in open-source projects.  This is a
relevant problem given the tremendous popularity of open-source
development and regression testing research.

The dimensions of analysis in this study are as follows.

\begin{itemize}
\item \textbf{Prevalence of long-running test suites.}
\item \textbf{Usage of test suite parallelization.}  
\item \textbf{Speedups of parallelization modes.}
\item \textbf{Issues with parallelization.}    
\end{itemize}  

%% papers also indicate the important of speeding up the testing process
%% in other industrial domain, e.g., in 
%% \Mar{Ask Milos if he can share stories on how prevalent regression
%%   testing is at Microsoft and Google, if he can explain how they run
%%   tests, and how prevalent is parallelization (in contrast to
%%   regression testing).}

%Motivated by this potential gap in regression testing,
%% More specifically, we studied how often long-running test suites occur
%% in this setting, how frequently parallelization features are used, if
%% not used why, and what are the associated drawbacks.

%% To run each test suite, we used a
%% 16GB memory Intel i7 machine\Comment{ i7-4790 Intel processor} with
%% eight virtual CPUs (four cores with two native threads each) and the
%% default Maven configuration of each project.

%% RQ1 + RQ2
We selected \numSubjs{} popular Java projects from \github{}
containing Maven build files~\cite{maven} to assess how prevalent
long-running test suites are.  Section~\ref{sec:eval} details our
methodology to select subjects and to isolate our experiments from
environmental noise.  In summary, results indicate that $\sim$21\% of
the test suites take at least 1m to run and $\sim$8\% of the test
suites take at least 5m to run.  Considering all the \numMedLong{}
projects in the $>$1m group (\percentMedLongRunning{} of
\numSubjs{})\Fix{$\leftarrow$this is not right. check macros.}, the
average execution time is \averageMedLongRunning{}.  Results also show
that individual tests are typically short-running, taking often less
than half a second to run.  Furthermore, we found that only in rare
cases few test cases monopolize execution cost associated with each
test suite.  Overall, these results suggest that test suite
parallelization has the potential to optimize these test suites.

%%RQ3
Given the high execution cost associated with a significant number of
\github{} projects, we decided to investigate practical usage of test
suite parallelization.  It is important to note that modern build
systems (\eg{}, Maven~\cite{maven}) offer parallelization options to
accelerate test execution (see Section~\ref{sec:modes}).  To find how
popular are these features in practice, we first analyzed build files
\emph{statically}.  Considering the +1m group of projects
(\numMedLong{} in total), we found that 49\% (41/\numMedLong{}) of
them use some parallelization option.  Considering those projects that
take more than 5m to run (\numLong{} in total), usage of parallelism
increases to 59\% (19/\numLong{}).  Furthermore, we found that
``forking'' virtual machines was the most prevalent parallel mode used
in practice.  This mode is used twice as much as any other mode used.
Important to note that ``forking VMs'' is perhaps the simplest of the
parallelization modes~--~it is provided by the build system as opposed
to the testing framework.  This result suggests that datarace
flakiness~\cite{luo-etal-fse2014} may be a concern to users as
``forking'' does not exploit thread-level concurrency within each VM,
which can create contention on data reachable from the static
area~\cite{bell-kaiser-icse2014,bell-etal-esecfse2015}.  Instead, it
creates independent processes (\ie{}, Java VM instances) and executes
tests sequentially within each process. \Mar{why 32-19 did not use
  parallelization?} \Mar{what about free services (jenkins)?  how much
  can you scale execution and how expensive would that be?}

\Fix{...Discuss impact...}

%% \Mar{missing connection. why parallelism?}  From these results we
%% investigated (2) how frequently the projects we considered use
%% parallelization to speedup test runs.  Our results indicate that
%% \percentParallelForLongRunning{} of long-running test suites do use
%% parallelism to speedup execution.  Given the popularity of this
%% solution, we further studied: (3) what are the causes of high
%% execution cost in individual tests (CPU vs. IO) and (4) how uniformly
%% cost is distributed across tests.  These questions help to assess the
%% impact of parallelization in test suite execution.  Intuitively,
%% IO-intensive test suites and test suites with uneven distribution of
%% cost per test are factors that limit the benefits of parallelization.
%% Results indicate that \Fix{...}  Finally, we investigated (5) what are
%% the limitations related to parallelization that could justify adoption
%% resistance.

%% To illustrate the importance of parallel test execution, \Jbc{TODO -
%% Write a paragraph motivating this paper.}

%% This paper reports an empirical study to evaluate the impact of
%% parallelization options on test execution speedup \Fix{...}\Jbc{TODO -
%% Summary of the contributions of this paper} .

%%  LocalWords:  parallelization multicore JUnit TestNG NUnit XXm YYm
%%  LocalWords:  Groupon parallelizing multi JVMs CPUs JVM Milos TODO
%%  LocalWords:  Ekstazi github CloudBuild PWA groupon criticality

 	
%%% Local Variables:
%%% TeX-master: "main"
%%% End:
%%  LocalWords:  jenkins
