\section{Introduction}

%% . which can disrupt the development
%% process~\cite{hilton-etal-ase2016}.\Comment{  This is often referred to as the
%% ``regression testing problem''.}

As software evolves it is expected that the number of tests and the
length of test runs increase.  Both components can add up to the
aggregate cost of running a test suite, which can lead to late reports
of test failures.  Dealing with high testing costs is critical at
large IT organizations, with massive code and test bases.  The Google
TAP system~\cite{google-tap,google-ci} and the Microsoft CloudBuild
system~\cite{prasad-shulte-ieee-microsoft-ci} provide evidence on the
criticality of the build and the testing processes at these
organizations.  Note, however, that the regression testing problem is
not specific to giants from the IT sector.  For example, a recent
paper from the automotive industry reported that test suites can take
days to run~\cite{artl-etal-icst2015}.

Several approaches have been proposed in the literature to address the
regression testing problem.  Research has focused mostly on test suite
minimization, prioritization, reduction, and
selection~\cite{yoo-harman-stvr2012}.  It is important to note,
however, that balancing the test workload, either within or across
machines, is a popular, sound, and practical approach to address
testing cost and can't be neglected.  Furtheremore, this approach is
complementary to those proposed in the literature. For example, as of
August 2013, the Groupon PWA system, which powers the
\url{groupon.com} website, included over 19K tests.  To run all those
tests under 10m, Groupon used a cluster of 4 computers with 24 cores
each~\cite{kim-etal-fse2013}.  As additional examples, Google's TAP
and Microsoft's CloudBuild distribute task workloads in the cloud.
Note, that the alternative of renting a cloud
service\footnote{\url{https://clutch.co/cloud}} or even building a
server farm (for increased security) is a legitimate approach to
mitigate the regression testing problem at large organizations.
However, this solution may not be economically viable for smaller
projects.

%% papers also indicate the important of speeding up the testing process
%% in other industrial domain, e.g., in 
%% \Mar{Ask Milos if he can share stories on how prevalent regression
%%   testing is at Microsoft and Google, if he can explain how they run
%%   tests, and how prevalent is parallelization (in contrast to
%%   regression testing).}

This paper focuses on optimizing regression testing of open-source
projects using commodity hardware.  This is an important problem
assuming that it is not rare for open-source projects to face high
testing costs.  \Fix{mention similar but different work} To the best
of our knowledge no prior work investigated this problem.

---- aqui

%% eventually move this to macros.tex
\newcommand{\numSubjs}{143}
\newcommand{\numLongRunning}{\Fix{17\%}}
\newcommand{\percentParallelForLongRunning}{\Fix{80\%}}

%Motivated by this potential gap in regression testing,

This paper focuses on the problem of understanding how critical the
regression testing problem is in open-source projects and what are the
potential benefits and shortcomings of using the parallelization
features of build systems and testing frameworks, which are very
popular today\Fix{cite cite cite}.  More precisely, we studied (1) how
prevalent long-running test suites are in open-source software
projects.  To conduct this experiment we selected \numSubjs{} Java
projects from github that are highly-popular and use the Maven build
system.  (Section~\ref{sec:eval} details our selection methodology.)
Our experimental results indicate that long-running test suites
(taking longer than 5m to complete\Fix{check if Ekstazi paper also
  used this value to indicate long runs and cite}) are not
rare~--~\numLongRunning{} of the projects we analyzed contain
long-running test suites.  \Mar{missing connection. why parallelism?}
From this result we investigated (2) how frequently the projects we
considered use parallelization to speedup test runs.  Our results
indicate that \percentParallelForLongRunning{} of long-running test
suites do use parallelism to speedup execution.  Given the popularity
of this solution, we further studied: (3) what are the causes of high
execution cost in individual tests (CPU vs. IO) and (4) how uniformly
cost is distributed across tests.  These questions help to assess the
impact of parallelization in test suite execution.  Intuitively,
IO-intensive test suites and test suites with uneven distribution of
cost per test are factors that limit the benefits of parallelization.
Results indicate that \Fix{...}  Finally, we investigated (5) what are
the limitations related to parallelization that could justify adoption
resistance.

\Fix{--------------}

To illustrate the importance of parallel test execution, \Jbc{TODO -
Write a paragraph motivating this paper.}

This paper reports an empirical study to evaluate the impact of
parallelization options on test execution speedup \Fix{...}\Jbc{TODO -
Summary of the contributions of this paper} .

%%  LocalWords:  parallelization multicore JUnit TestNG NUnit XXm YYm
%%  LocalWords:  Groupon parallelizing multi JVMs CPUs JVM Milos TODO
%%  LocalWords:  Ekstazi github CloudBuild PWA groupon
