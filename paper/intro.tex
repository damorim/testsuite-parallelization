\section{Introduction}
\label{sec:intro}

%% . which can disrupt the development
%% process~\cite{hilton-etal-ase2016}.\Comment{  This is often referred to as the
%% ``regression testing problem''.}

%% As software evolves it is expected that the number of tests and the
%% length of test runs increase.  Both components can add up to the
%% aggregate cost of running a test suite.

Dealing with high testing costs has been an important problem in
software engineering research and industrial practice.  Several
approaches have been proposed in the research literature to address
this problem, with the focus mainly on test suite minimization,
prioritization, reduction, and selection~\cite{yoo-harman-stvr2012}.
In industry, the focus has been mainly on distributing the testing
workload.  Evidence of this are the Google TAP
system~\cite{google-tap,google-ci} and the Microsoft CloudBuild
system~\cite{prasad-shulte-ieee-microsoft-ci}, which provide
distributed infrastructures to efficiently build massive amounts of
code and run tests.  Building in-house server clusters is also a
popular mechanism to distribute testing workloads.  For example, as of
August 2013, the test suite of the Groupon PWA system, which powers
the \url{groupon.com} website, included over 19K tests.  To run all
those tests under 10m, Groupon used a cluster of 4 computers with 24
cores each~\cite{kim-etal-fse2013}.

At large organizations, the alternative of renting cloud
services~\cite{cloud-services} or even building proprietary
infrastructures for running tests is a legitimate approach to mitigate
the regression testing problem.  However, for projects with modest
 or nonexistent budgets and yet relatively heavy testing workloads, this solution may
not be economically viable.  For these cases, the use of commodity
hardware\Comment{ (\eg{}, existing workstations)} is an attractive
solution for running tests.  The proliferation of
multi-core CPUs and the increasing popularization of testing
frameworks and build systems, which today provide mature support for
parallelization, enable\Comment{ test execution} speedups through increased CPU
usage~(see Section~\ref{sec:modes}).  These two elements~---~demand
for cost-effective test execution and supply of relatively inexpensive
testing infrastructures~---~inspired us to investigate 
test suite parallelization in practice.

%% \Mar{Fan Long
%%   (http://people.csail.mit.edu/fanl/) often uses a plot to show the
%%   exponential rise of projects in github.  Please check if you can
%%   find a reference to that or ask Fan Long what source he used (maybe
%%   browsed with public API).}

%% \begin{enumerate}
%% \item \textbf{Prevalence of long-running test suites.}
%% \item \textbf{Prevalence of test suite parallelization.}
%% \item \textbf{Developer's opinion (when parallelization not used).}    
%% \item \textbf{Speedups of parallelization (when used).}
%% \item \textbf{Issues versus benefits on using parallelization modes.}
%% \end{enumerate}

This paper reports on an empirical study we conducted to analyze the
usage and impact of low-level parallelization to speed up testing in
open-source projects.  This is a relevant problem given the tremendous
popularity of open-source development and regression testing
research~\cite{yoo-harman-stvr2012}.  Note that parallelization is
complementary to other approaches to mitigate testing costs such as
(safe) test
selection~\cite{Rothermel:1997:SER:248233.248262,gligoric-etal-issta2015}
and continuous integration~\cite{Saff:2003:RWD:951952.952340}.

The dimensions of analysis we considered in this study are (i)
feasibility, (ii) adoption, (iii) speedup, and (iv) tradeoffs.  The
dimension \emph{feasibility} measures the potential of parallelization
to reduce testing costs.  In the limit, parallelization would be
fruitless if all projects had short-running test suites or if the
execution cost was dominated by a single test case in the suite.  The
dimension \emph{adoption} evaluates how often existing open-source
projects use parallelization schemes and how developers involved in
costly projects (not using test suite parallelization) perceive this
technology.  It is important to measure resistance of practitioners to
the technology and to understand their reasons.  The dimension
\emph{speedup} evaluates the observed impact of
parallelization\Comment{~---~when used in selected open-source
  projects~---~} in running times.  Finally, the dimension
\emph{tradeoffs} evaluates the relationship between speedups obtained
with parallelization and issues that arise 
when running tests in parallel, including test
flakiness~\cite{luo-etal-fse2014,bell-etal-esecfse2015}.  We briefly summarize our findings
in the following.

%% Discovering Pareto-efficient~\cite{hwang-masud-1979}
%% configurations is important to guide practitioners and future
%% developments in this area.

%% papers also indicate the important of speeding up the testing process
%% in other industrial domain, e.g., in 
%% \Mar{Ask Milos if he can share stories on how prevalent regression
%%   testing is at Microsoft and Google, if he can explain how they run
%%   tests, and how prevalent is parallelization (in contrast to
%%   regression testing).}

%Motivated by this potential gap in regression testing,
%% More specifically, we studied how often long-running test suites occur
%% in this setting, how frequently parallelization features are used, if
%% not used why, and what are the associated drawbacks.

%% To run each test suite, we used a
%% 16GB memory Intel i7 machine\Comment{ i7-4790 Intel processor} with
%% eight virtual CPUs (four cores with two native threads each) and the
%% default Maven configuration of each project.


\noindent\emph{Feasibility.}~To assess how prevalent
long-running test suites are we selected \numSubjs{} popular Java
projects from \github{} containing Maven build files~\cite{maven}.
Section~\ref{sec:eval} details our methodology to select subjects and
to isolate our experiments from environmental noise.  Results indicate
that nearly \percentMedLongRunning{} of the projects take at least 1m
to run and \percentLongRunning{} of the projects
take at least 5m to run.  Considering the \numMedLong{}
projects with test suites taking longer than a minute to run, the average execution time of a test
suite was \averageMedLongRunning{}.  Results also show that test cases
are typically short-running, typically taking less than half a second
to run.  Furthermore, we found that only in rare cases few test cases
monopolize the overall time to run a test suite.


\noindent\emph{Adoption.}~ We considered two aspects in
measuring technology adoption.  First, we measured usage of
parallelism in open-source projects.  Then, we ran a survey with
developers to understand the reasons that could explain resistance to
using the technology.\Comment{ For the quantitative analysis we
  checked both statically (parsing build files) and dynamically
  (monitoring execution of build files) for the presence of
  parallelization.}  Considering only the projects whose test suites
take longer than a minute to run, we found that
only \percentParallelUpdated{} of them use parallelism.
%% To sum, results indicate that test suite parallelization is
%% underutilized.
%% only
%% \percentParallel{} of them use parallelism.  When correcting this
%% value with the feedback obtained a posteriori from developers, we
%% realized that the proportion should be higher, of
%% \percentParallelUpdated{}.
%%For the qualitative analysis
We also contacted 
developers from a selection of costly projects that did not use
parallelization to understand the reasons for not using parallelization.  Dealing with concurrency-related issues (\eg{}, the
extra work to organize test suite to avoid concurrency errors) and the
availability of continuous integration services were the most
frequently answered reasons for not considering parallelization.

%% \Mar{move to here} Intuitively, benefits are
%% proportional to the cost reduction that parallelization brings.  In
%% the limit, however, the overhead of spawning virtual machines or
%% threads can even result in added cost.

\noindent\emph{Speedups.}~We used two setups to measure
speedups.  In one setup we measured speedups obtained on
projects that run test suites in parallel by
default.  In the other setup, we evaluated how execution scales with
the number of available cores in the machine.\Comment{We did not
  change parallel configurations on that setup.  For comparison, we
  disabled parallel execution to obtain the running time of a
  sequential execution.}  Considering the first setup, results
indicate that the average speedup of parallelization was
\avgSpeedup{}x.  Although we found cases with very high speedups
(\eg{}, 28.8x for project Jcabi), we also found cases where the speedups were
not very significant.  Considering the scalability experiment, we
noticed, perhaps as expected, that parallelization obtained with forking JVMs scales with the number
of cores but the speedups are bounded by
long-running test classes.

%% %%RQ3
%% Given the high execution cost associated with a significant number of
%% \github{} projects, we decided to investigate usage of test
%% suite parallelization in practice.  It is important to note that modern build
%% systems (\eg{}, Maven~\cite{maven}) offer parallelization options to
%% accelerate test execution (see Section~\ref{sec:modes}).  To find how
%% popular these features are in practice, we first analyzed build files
%% \emph{statically}.  Considering the $>$1m group of projects
%% (\numMedLong{} in total), we found that 49\% (41/\numMedLong{}) of
%% them use some parallelization option.  Considering those projects that
%% take more than 5m to run (\numLong{} in total), usage of parallelism
%% increases to 59\% (19/\numLong{}).  Furthermore, we found that
%% ``forking'' virtual machines was the most prevalent parallel mode used
%% in practice.  This mode is used twice as much as any other mode used.
%% Important to note that ``forking VMs'' is perhaps the simplest of the
%% parallelization modes~--~it is provided by the build system as opposed
%% to the testing framework.  This result suggests that datarace
%% flakiness~\cite{luo-etal-fse2014} may be a concern to users as
%% ``forking'' does not exploit thread-level concurrency within each VM,
%% which can create contention on data reachable from the static
%% area~\cite{bell-kaiser-icse2014,bell-etal-esecfse2015}.  Instead, it
%% creates independent processes (\ie{}, Java VM instances) and executes
%% tests sequentially within each process. \Mar{why 32-19 did not use
%%   parallelization?}

\noindent\emph{Tradeoffs.}~Test flakiness is a central
concern when running tests in parallel.  Dependent tests
can be affected by different schedulings of test methods and classes.
This dimension of the study measures the impact of different parallel
configurations on test flakiness and speedup.  Overall, results
indicate that configurations that fork JVMs do not achieve
speedups as high as other more-agressive configurations, but they
manifest much lower flakiness ratios.

Our observations may trigger different actions:

\setlist[itemize]{leftmargin=1.1em}
\begin{itemize}
\item \emph{Incentivize forking.}~Forked JVMs manifest very low rates
  of test flakiness.  Developers of projects with long-running test
  suites should consider using that feature, which is available in
  modern build systems today (\eg{}, Maven).
\item \emph{Break test dependencies.}~Non-forked JVMs can achieve
  impressive speedups at the expense of sometimes impressive rates of
  flakiness.  Breaking test dependencies (with
  ElectricTest~\cite{bell-etal-esecfse2015}, for example) to avoid flakiness is advised for developers with
  greater interest in efficiency.
\item \emph{Refactor tests for load balancing.}~The configuration with
  forked JVMs scales
  better\Comment{ with the number of cores} when the test workload is balanced
  across testing classes.  Automated refactoring could help balance
  the workload in scenarios where developers are not willing to change
  test code but have access to machines with a high number of cores.
\item \emph{Improve debugging for build systems.}~While preparing our
  experiments, we found scenarios\Comment{, related to test
    parallelization,} where Maven's executions did not reflect
  corresponding JUnit's executions. (Docker reproduction scripts
  available.) Those issues can hinder developers from using parallel
  testing. Better debugging support for build systems could help
  on that. 
\end{itemize}

The artifacts we produced as result of this study are available from
the following web page \webpage{}.

%% \Mar{missing connection. why parallelism?}  From these results we
%% investigated (2) how frequently the projects we considered use
%% parallelization to speedup test runs.  Our results indicate that
%% \percentParallelForLongRunning{} of long-running test suites do use
%% parallelism to speedup execution.  Given the popularity of this
%% solution, we further studied: (3) what are the causes of high
%% execution cost in individual tests (CPU vs. IO) and (4) how uniformly
%% cost is distributed across tests.  These questions help to assess the
%% impact of parallelization in test suite execution.  Intuitively,
%% IO-intensive test suites and test suites with uneven distribution of
%% cost per test are factors that limit the benefits of parallelization.
%% Results indicate that \Fix{...}  Finally, we investigated (5) what are
%% the limitations related to parallelization that could justify adoption
%% resistance.

%% To illustrate the importance of parallel test execution, \Jbc{TODO -
%% Write a paragraph motivating this paper.}

%% This paper reports an empirical study to evaluate the impact of
%% parallelization options on test execution speedup \Fix{...}\Jbc{TODO -
%% Summary of the contributions of this paper} .

%%  LocalWords:  parallelization multicore JUnit TestNG NUnit XXm YYm
%%  LocalWords:  Groupon parallelizing multi JVMs CPUs JVM Milos TODO
%%  LocalWords:  Ekstazi github CloudBuild PWA groupon criticality VM

 	
%%% Local Variables:
%%% TeX-master: "main"
%%% End:
%%  LocalWords:  jenkins API VMs datarace tradeoffs tradeoff Paretto
%%  LocalWords:  schedulings posteriori priori bla blabla Jcabi
%%  LocalWords:  scalability parallelize Incentivize Refactor JUnit's
%%  LocalWords:  refactoring
