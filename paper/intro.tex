\section{Introduction}

%% . which can disrupt the development
%% process~\cite{hilton-etal-ase2016}.\Comment{  This is often referred to as the
%% ``regression testing problem''.}

As software evolves it is expected that the number of tests and the
length of test runs increase.  Both components can add up to the
aggregate cost of running a test suite, which can lead to late reports
of test failures.  Dealing with high testing costs is critical at
large IT organizations, with massive code and test bases.

Several approaches have been proposed in the literature to address the
regression testing problem.  Research has focused mostly on test suite
minimization, prioritization, reduction, and
selection~\cite{yoo-harman-stvr2012}.  These approaches reduce testing
cost by reducing the amount of tests for execution.  It is important
to bring to light that the \emph{complementary alternative} of
balancing the test workload across work units is sound (\ie{}, it
cannot miss fault-revealing tests) and very popular in industry.
Evidence of this are the Google TAP system~\cite{google-tap,google-ci}
and Microsoft's CloudBuild
system~\cite{prasad-shulte-ieee-microsoft-ci}, which provide
distributed infrastructures to efficienctly build massive amounts of
code and run tests.  Note, however, that the regression testing
problem is not specific to giants from the IT sector.  For example, as
of August 2013, the Groupon PWA system, which powers the
\url{groupon.com} website, included over 19K tests.  To run all those
tests under 10m, Groupon used a cluster of 4 computers with 24 cores
each~\cite{kim-etal-fse2013}.



%% As additional examples, Google's
%% TAP and Microsoft's CloudBuild distribute task workloads in the cloud.

At large organizations, the alternative of renting a cloud
service\footnote{\url{https://clutch.co/cloud}} or even building
proprietary infrastructures for running tests is a legitimate approach
to mitigate the regression testing problem.  However, for projects
with low budgets and yet moderate or heavy testing workloads, this
solution may not be economically viable.  The amount of serious
open-source project under development today is impressive.\Mar{give
  examples, provide stats.}  Optimizing existing computational
resources in this context saves money or time.  \Mar{what about free
  services (jenkins)?  how much can you scale execution and how
  expensive would that be?}

%% papers also indicate the important of speeding up the testing process
%% in other industrial domain, e.g., in 
%% \Mar{Ask Milos if he can share stories on how prevalent regression
%%   testing is at Microsoft and Google, if he can explain how they run
%%   tests, and how prevalent is parallelization (in contrast to
%%   regression testing).}

\sloppy This paper focuses on the problem of understanding how
critical the regression testing problem is in open-source projects.
To the best of our knowledge no prior work investigated this problem,
which is important given the tremendous popularity of open-source
development and the wide impact of regression testing optimization
techniques in this setting.

%Motivated by this potential gap in regression testing,
%% More specifically, we studied how often long-running test suites occur
%% in this setting, how frequently parallelization features are used, if
%% not used why, and what are the associated drawbacks.

More precisely, we studied (1) how prevalent long-running test suites
are in open-source software projects.  To that end, we selected from
\github{} \numSubjs{} highly-popular Java projects containing Maven
build files.  (Section~\ref{sec:eval} details our selection
methodology.)  To run this experiment we used the default Maven
configuration to run a project's test suite and a 16GB memory i7-4790
Intel processor machine, which we carefully isolated.  Our results
indicate that the test suite of nearly one in every five projects from
our set (\percentMedLongRunning) takes at least 5m to run.  On
average, one of those test suites take \averageMedLongRunning{} to
run.  These results suggest that open-source projects often contain
long running test suites.

\Mar{missing connection. why
  parallelism?}  From this result we investigated (2) how frequently
the projects we considered use parallelization to speedup test runs.
Our results indicate that \percentParallelForLongRunning{} of
long-running test suites do use parallelism to speedup execution.
Given the popularity of this solution, we further studied: (3) what
are the causes of high execution cost in individual tests (CPU vs. IO)
and (4) how uniformly cost is distributed across tests.  These
questions help to assess the impact of parallelization in test suite
execution.  Intuitively, IO-intensive test suites and test suites with
uneven distribution of cost per test are factors that limit the
benefits of parallelization.  Results indicate that \Fix{...}
Finally, we investigated (5) what are the limitations related to
parallelization that could justify adoption resistance.

\Fix{--------------}

To illustrate the importance of parallel test execution, \Jbc{TODO -
Write a paragraph motivating this paper.}

This paper reports an empirical study to evaluate the impact of
parallelization options on test execution speedup \Fix{...}\Jbc{TODO -
Summary of the contributions of this paper} .

%%  LocalWords:  parallelization multicore JUnit TestNG NUnit XXm YYm
%%  LocalWords:  Groupon parallelizing multi JVMs CPUs JVM Milos TODO
%%  LocalWords:  Ekstazi github CloudBuild PWA groupon criticality

 	
%%% Local Variables:
%%% TeX-master: "main"
%%% End:
