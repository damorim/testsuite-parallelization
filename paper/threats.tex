\section{Threats to Validity}

The main threats to validity of this study are the following.

\textit{External Validity:} Generalization of our findings is limited
to our selection of subjects, testing framework, and build system.  To
mitigate that issue, we selected subjects according to an objective
criteria, described in Section~\ref{sec:subjects}.  It remains to
evaluate the extent to which our observations would change when using
different testing frameworks and build systems.
Also, some of the selected subjects contain failing tests. Test
failures may reduce the testing time due to early termination
or even inflate the time (\eg, test waiting indefinitely for
an unavailable resources).
To mitigate this threat, we eliminated subjects with flaky tests and
filtered projects with at least 90\% of the tests passing.
Only 17\% of our subjects have failing tests.
We carefully inspected our rawdata to identify and ignore these
failures with JUnit's \CodeIn{@Ignore} annotation.

\textit{Internal Validity:} Our results could be influenced by
unintentional mistakes made by humans who interpreted survey data and
implemented scripts and code to collect and analyze the data.
For example, we developed JUnit runners to reproduce Maven's parallel
configurations and implemented several scripts to automate our
experiments (\eg, run tests and detect parallelism enabled by default
in the subjects).
All those tasks could bias our results.
To mitigate those threats, the first two authors of this paper
validated/inspected each other to increase chances of capturing
unintentional mistakes.

\textit{Construct Validity:} We considered a number of metrics in this
study that could influence some of our interpretations.  For example,
we measured number of test cases per suite, distribution of test costs
in a suite, time to run a suite, etc.  In principle, these metrics may
not reflect the main problems associated with test
efficiency.

%%  LocalWords:  QA dockerfile StackOverflow
