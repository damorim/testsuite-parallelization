Teste de software é um processo custoso mas mandatório no processo de
desenvolvimento. Sistemas complexos podem demandar horas para executar várias
suítes de testes. Para mitigar os custos, uma estratégia popular em grandes
corporações é o uso de um ambiente de execução distribuído em escala global.
Desta forma, a carga de testes é dividida entre diversos servidores e os testes
são executados simultaneamente em diferentes máquinas. Outra estratégia
explorada por pesquisadores e desenvolvedores é a aplicação de técnicas de teste
de regressão, como seleção e minimização de testes. Uma outra forma igualmente
relevante e complementar a estratégias existentes é a execução paralela de
suítes de testes em uma mesma máquina. Nos últimos anos, com a popularização de
processadores com múltiplos núcleos, ferramentas de build e bibliotecas de
teste adicionaram suporte a execução paralela de testes em diferentes
granularidades com o objetivo de prover a máxima utilização dos recursos
computacionais de uma máquina e diminuir o tempo execução. Desta forma, testes
podem ser executados em diferentes processos e até mesmo em múltiplas threads
em um mesmo processo.

Este trabalho reporta nossas descobertas sobre a utilização e o impacto de
execução paralela de testes em projetos de código aberto e provê recomendações
a programadores e desenvolvedores de ferramentas que pretendem utilizar ou
prover esta funcionalidade. Considerando um conjunto de \numSubjs{} projetos
populares escritos em Java, observamos que \percentMedLongRunning{} destes
projetos possuem suítes de teste custosas. Apesar disto, apenas
\percentParallelUpdated{} utilizam alguma forma de paralelismo para acelerar a
execução dos testes.
O principal motivo de resistência para adoção de execução paralela é o receio
de introduzir problemas de concorrência. Os resultados sugerem que, em média,
desenvolvedores preferem manter a previsibilidade de uma execução
determinística ao invés de alta performance. 

\begin{keywords}
Engenharia de Software, Teste de Software, Paralelismo
\end{keywords}
