Testing is a costly process but mandatory in the development process.
Complex systems may require hours to run several test suites.
To mitigate this cost, a popular strategy in large companies is the usage of a
distributed execution environment at global scale.
In this approach, the workload is divided among several servers, and tests
run simultaneously on different machines.
Another approach explored by researchers and practitioners is the usage of
regression testing techniques such as test selection and minimization.
Test suite parallelization is an important approach to address costly
executions and a complementary approach to existing techniques.
With the popularization of multi-core processors in the last years, build tools
and testing frameworks added support to parallel execution in different
granularities to speedup test execution by leveraging the underlying hardware.
In this approach, tests can run on different processes and even in multiple
threads within a process.

This work reports our findings on the usage and impact of test suite
parallelization in open-source projects, and it provides recommendations to
practitioners and tool developers.
Considering a set of \numSubjs{} popular Java projects, we observed that
\percentMedLongRunning{} of these projects contain costly test suites.
However, only \percentParallelUpdated{} of costly projects use some form of
parallelization to speedup test execution.
The main reported reason for adoption resistance was dealing with concurrency
issues.
Results suggest that, on average, developers prefer high predictability than
high performance in running tests.

\begin{keywords}
Software Engineering, Software Testing, Parallelism
\end{keywords}

